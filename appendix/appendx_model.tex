\chapter{Flux balance analysis}
\label{append:model}

\section{Modifications of chemical reaction in GECKO, general cases}
\label{append:model-gecko}

To illustrate how GECKO modifies chemical reactions, consider a simple example, a reaction $R_{j}$ catalysed by enzyme $E_{i}$:

\begin{equation}
  \ce{ A + B ->[E_{i}] C + D }
  \label{eq:model-gecko-catalysed}
\end{equation}

To apply the constraints described by Eq.\ \ref{eq:model-gecko-fba-constraints}, the chemical reaction in Eq.\ \ref{eq:model-gecko-catalysed} is modified by adding a term:

\begin{equation}
  \ce{ n_{ij}E_{i} + A + B -> C + D }
  \label{eq:model-gecko-catalysed-formalism}
\end{equation}
with the stoichiometric coefficient $n_{ij} = 1/k_{\mathrm{cat}}^{ij}$.
GECKO takes $k_{cat}$ values from BRENDA \parencite{changBRENDAELIXIRCore2021}.

This transformation, adding the enzyme as a pseudometabolite, is based on the intuition that the system uses some amount of enzyme at a specific time to catalyse the flux through the reaction.
Slightly different formalisms are applied to reversible reactions, isozymes, promiscuous enzymes, and enzyme complexes (Appendix~\ref{append:model-gecko-nonsimple}).

To constrain overall enzyme levels in the model, GECKO defines a pseudoreaction

\begin{equation}
  ER_{\mathrm{pool}}: \varnothing \ce{ -> E_{pool} }
  \label{eq:model-gecko-enzyme-pool}
\end{equation}

where $E_{\mathrm{pool}}$ is a pseudometabolite that represents the proteome pool available for enzymes.
This pseudoreaction has a flux

\begin{equation}
  \epool \leq (P_{\mathrm{total}} - P_{\mathrm{measured}}) \cdot f \cdot \sigma
  \label{eq:model-gecko-enzyme-pool-flux}
\end{equation}

in units of \SI{}{\gram~\gram_{DW}^{-1}}, where
$P_{\mathrm{total}}$ is the total protein fraction with respect to the dry weight of the cell,
$P_{\mathrm{measured}}$ is the protein fraction of proteins whose weight are accounted for in the model, based on proteomic data,
$f$ represents the fraction of proteins that are enzymes, and
$\sigma$ is a parameter that represents the average saturation of enzymes.
ecYeast8.6.0 assumes parameter values of $f = 0.5$, $P_{\mathrm{total}} = 0.5$, and $\sigma = 0.5$.
If no proteomic data is used, as is the case in this chapter, $P_{\mathrm{measured}} = 0$.

Defining such parameters is a judgement call, especially when the protein fraction varies across growth rates~\parencite{elsemmanWholecellModelingYeast2022}, but $f = 0.5$ is close to the protein mass fraction of ecYeast8.6.0.
Subsequently, GECKO changes the carbohydrate composition based on the assumption that a change in the amino acid composition is offset by the reverse change in the carbohydrate composition;
experimental data justifies this assumption \parencite{nissenFluxDistributionsAnaerobic1997}.

Then, for each enzyme $\ce{E_{i}}$, GECKO defines enzyme usage pseudoreactions of the form

\begin{equation}
  ER_{i}: \ce{ MW_{i} E_{\mathrm{pool}} -> E_{i} }
  \label{eq:model-gecko-enzyme-usage}
\end{equation}

where $\mathrm{MW}_{i}$ represents the molecular weight of the enzyme in units of \SI{}{\gram~\milli\mole^{-1}}.
The flux of enzyme usage pseudoreactions are defined in units of \SI{}{\mmolgdw}.
GECKO takes enzyme data from SWISSPROT \parencite{theuniprotconsortiumUniProtUniversalProtein2023} and KEGG \parencite{kanehisaKEGGTaxonomybasedAnalysis2023}, including molecular weight of proteins and associated pathways.

Taken together, the modelled cell thus has an enzyme pool in terms of a mass fraction of the cell's dry weight, and the modelled cell allocates certain fractions of this mass to the synthesis of each enzyme at steady-state.
The mass of each enzyme in the cell determines the amount (in moles) of each enzyme and therefore its catalytic activity.


\section{Modifications of chemical reactions in GECKO for non-simple cases}
\label{append:model-gecko-nonsimple}

GECKO modifies chemical reactions in a genome-scale model so that enzymes are expressed as metabolites that take part in reactions.
Section~\ref{subsec:model-yeast8-gecko} describes how this was performed for the simple case of a singular chemical reaction and one enzyme having a one-to-one association.
Slightly different formalisms are applied to reversible reactions, isozymes, promiscuous enzymes, and enzyme complexes.
Namely:
\begin{itemize}
  \item Reversible reactions are modelled as the forward and reverse reactions separately.
  \item For isozymes, the original reaction is copied multiple times corresponding to the number of reactions that the isozyme catalyses.
        Each has an isozyme catalysing the reaction.
        In addition, there is an `arm' reaction to act as an intermediate between the substrate and the products.
  \item No actions are needed for promiscuous enzymes.
  \item Enzyme complexes are modelled as one reaction that uses all subunit proteins that all share the same $k_{cat}$ value.
\end{itemize}


\section{Computing molecular weights of pseudometabolites in ecYeast8}
\label{append:model-molweights}

In genome-scale metabolic models, mass fractions are represented in the biomass reaction, set as the objective function, which is defined as:

\begin{equation}
  \ce{f_{1}M1 + f_{2}M2 + ... + f_{n}M_{n} -> B}
  \label{eq:model-biomass-reaction}
\end{equation}

where $M_{1} \ldots M_{n}$ represent the chemical species that make up the cell's biomass, the stoichiometric coefficients $f_{1} \ldots f_{n}$ represent the mass fraction of each species in units of \SI{}{\gram~\gram_{DW}^{-1}}, and $B$ represents biomass.
If a chemical species $M_{i}$ has a mass fraction $f_{i}$, then \SI{1}{\gram} of cell dry weight has $f_{i}$ \SI{}{\gram} of chemical species $M_{i}$.

In the ecYeast8 model, the mass fraction of biomass components cannot simply be obtained by taking the stoichiometric coefficients ($f_{i}$ in Eq.\ \ref{eq:model-biomass-reaction}) from the objective function.
This is because the objective function of Yeast8 does not conform to this format, and instead contains pseudometabolites.
This formalism can be expressed as:

\begin{equation}
  \begin{aligned}
    \ce{f_{1}M_{B,1} + ... + f_{n}M_{B,n} + f_{n+1}P1 + ... + f_{n+k}P_{k} &-> B}\\
    \ce{f_{P_{1},1}M_{P_{1},1} + ... + f_{P_{1},n}M_{P_{1},n} &-> P1}\\
    &\ldots\\
    \ce{f_{P_{k},1}M_{P_{k},1} + ... + f_{P_{k},n}M_{P_{k},n} &-> P_{k}}
    \label{eq:model-pseudometabolites}
  \end{aligned}
\end{equation}

where each $M$ is a chemical species with a defined molecular weight, each $P$ is a pseudometabolite, and each $f$ is a stoichiometric coefficient.
The objective function remains the reaction that produces $B$, but some chemical species $M_{B,1} \ldots M_{B,n}$ are retained in the objective function, while other chemical species are replaced by pseudometabolites $P_{1} \ldots P_{k}$.
The reactions that produce $P_{1} \ldots P_{k}$ are \textit{isa} reactions.
\textit{isa} reactions define pseudometabolites by having chemical species with known molecular weights as reactants, with their stoichiometric coefficients representing abundance in \SI{}{\mmolgdw}.
In Yeast8, the objective function is defined as:

\texttt{
  47.5883 atp\_c + 47.5883 h2o\_c + lipid\_c + protein\_c + carbohydrate\_c\\
  + dna\_c + rna\_c + cofactor + ion \\
  --> 47.5883 adp\_c + biomass\_c + 47.5883 h\_c + 47.5883 pi\_c
}

Here, there are seven pseudometabolites: lipid, protein, carbohydrate, DNA, RNA, cofactor, and ion.

As the ecYeast8 model does not specify the molecular weights of these pseudometabolites, in order to obtain the mass fraction of each biomass component represented by the pseudometabolites,
I treated each pseudometabolite as a chemical species and calculated its molecular weight by assuming mass balance \parencite{chanStandardizingBiomassReactions2017, dinhQuantifyingPropagationParametric2022, takhaveevTemporalSegregationBiosynthetic2023}.
Namely, I assumed that in reactions that produce the pseudometabolites, there is conservation of mass, and therefore:

% To this end, I first computed molecular weights for pseudometabolites that represent macromolecules and other biomass components in the ecYeast8 model, as
% the ecYeast8 model does not specify the molecular weights of these pseudometabolites.
% The pseudometabolites include: lipids, proteins, carbohydrates, RNA, DNA, cofactors, and ions.
% Namely, I assumed that in reactions that produce the pseudometabolites, there is conservation of mass, and therefore:

\begin{equation}
  \sum_{r = j}^{n_{r}}m_{r}c_{r} = \sum_{p = i}^{n_{p}}m_{p}c_{p}
\label{eq:conservation-of-mass}
\end{equation}

where
$s = 1, \ldots n_{s}$ represents substrates of the reaction in question,
$p = 1, \ldots n_{p}$ represents products.
$m_{r}$ represents molar mass of reactant $r$,
$m_{p}$ represents molar mass of product $p$,
$c_{r}$ represents stoichiometric coefficient of reactant $r$, and
$c_{p}$ represents stoichiometric coefficient of product $p$.

The resulting molecular weight will thus represent the mass fraction of each biomass component in units of \SI{}{\gram~\gram_{DW}^{-1}}.


\subsection{Carbohydrate, DNA, RNA, cofactor, and ion pseudometabolites}
\label{append:model-molweights-easy}

Computing the molecular weights of the carbohydrate, DNA, RNA, cofactor, and ion pseudometabolites is straightforward.
This is because the equations similarly have reactants with molecular weights specified in the model and only the pseudometabolite, the sole product, does not have a molecular weight specified.
In such cases, Eq.\ \ref{eq:conservation-of-mass} can be applied directly, i.e.\ the molecular weight of the pseudometabolite is equal to $\sum_{r = j}^{n_{r}}m_{r}c_{r}$, where $m_{r}$ values are taken directly from the molecular weights specified in the model.
The results for these pseudometabolites are shown in Table~\ref{tab:ecyeast8-easy-rxns}

\begin{table}[ht]
  \centering
    \begin{tabular}{llS}
      ID & Reaction & {\makecell{Computed molecular\\ weight (\SI{}{\gram~\mole^{-1}})}} \\
      \hline
      \texttt{r\_4048} & \makecell{\texttt{0.684535 (1->3)-beta-D-glucan} \\ \texttt{ + 0.228715 (1->6)-beta-D-glucan} \\ \texttt{+ 0.330522 glycogen + 0.650171 mannan} \\ \texttt{ + 0.126456 trehalose} \\ \texttt{--> carbohydrate}} & 350.37 \\
    \texttt{r\_4050} & \makecell{\texttt{0.0036 dAMP + 0.0024 dCMP + 0.0024 dGMP} \\ \texttt{ + 0.0036 dTMP} \\ \texttt{--> DNA}} & 3.90 \\
    \texttt{r\_4049} & \makecell{\texttt{0.0445348 AMP + 0.0432762 CMP} \\ \texttt{+ 0.0445348 GMP + 0.0579921 UMP} \\ \texttt{--> RNA}} & 64.04 \\
    \texttt{r\_4598} & \makecell{\texttt{0.00019 coenzyme A + 1e-05 FAD} \\ \texttt{ + 0.00265 NAD + 0.00015 NADH} \\ \texttt{ + 0.00057 NADP(+) + 0.0027 NADPH} \\ \texttt{ + 0.00099 riboflavin + 1.2e-06 TDP} \\ \texttt{ + 6.34e-05 THF + 1e-06 heme a} \\ \texttt{--> cofactor}} & 4.83 \\    \texttt{r\_4599} & \makecell{\texttt{3.04e-05 iron(2+) + 0.00363 potassium} \\ \texttt{ + 0.00397 sodium + 0.02 sulphate} \\ \texttt{ + 0.00129 chloride + 0.00273 Mn(2+)} \\ \texttt{ + 0.000748 Zn(2+) + 0.000217 Ca(2+)} \\ \texttt{ + 0.00124254 Mg(2+) + 0.000659 Cu(2+)} \\ \texttt{--> ion}} & 2.48 \\
    \end{tabular}
    \caption{Straightforward cases of computing pseudometabolite molecular weights from pseudoreactions in ecYeast8}
    \label{tab:ecyeast8-easy-rxns}
\end{table}


\subsection{Protein pseudometabolite}
\label{append:model-molweights-protein}

Other metabolites were less straightforward and required some judgement calls.
To compute the molecular weight of the protein pseudometabolite, I inspected reaction \texttt{r\_4047}:

\texttt{
  0.57284 Ala-tRNA(Ala) + 0.200644 Arg-tRNA(Arg) + 0.126979 Asn-tRNA(Asn)\\
  + ... + 0.330369 Val-tRNA(Val) \\
  --> 0.57284 tRNA(Ala) + 0.200644 tRNA(Arg) + 0.126979 tRNA(Asn) \\
  + ... + 0.330369 tRNA(Val) + protein
}

In Yeast8, aminoacyl-tRNA and tRNA species do not have molecular weights specified in the model.
This is because their chemical formulas are incompletely specified in the model as the model uses \texttt{R} to represent the tRNA.
For example, \texttt{Ala-tRNA(Ala)}, alanyl-tRNA, is represented as \texttt{C3H7NOR}, and \texttt{tRNA(Ala)} is represented as \texttt{RH}.
As a consequence, the molecular weights of these species cannot be directly computed from the chemical formula.
Because tRNAs are unmodified during translation, \texttt{R} can be ignored.
In other words, I treated \texttt{R} as a chemical element of atomic mass 0 when computing $m_{r}$ for each reactant and $m_{p}$ for each product, leaving only $m_{p}$ for \texttt{protein} undefined.
This $m_{p}$ can then be found by rearranging Eq.\ \ref{eq:conservation-of-mass}, thus giving the molecular mass of the protein pseudometabolite.


\subsection{Lipid pseudometabolite}
\label{subsubsec:model-molweights-lipid}

Finally, the lipid pseudometabolite is the least straightforward because the model does not specifiy the molecular weights of some of the reactants of the lipid pseudoreaction.
The lipid pseudoreaction is represented in reaction \texttt{r\_2108}:

\texttt{
  lipid backbone + lipid chain --> lipid
}

And both \texttt{lipid backbone} and \texttt{lipid chain} have no molecular weight specified.

Reaction \texttt{r\_4065} specifies a lipid chain pseudoreaction, in which \texttt{lipid chain} is generated:

\texttt{
  0.0073947 C16:0 chain + 0.0217019 C16:1 chain + 0.0020726 C18:0 chain \\
  + 0.000796243 C18:1 chain \\
  --> lipid chain
}

As all reactants have molecular weights defined in the model, the molecular weight of \texttt{lipid chain} can be computed from the mass balance of this reaction.

Reaction \texttt{r\_4063} specifies a lipid backbone pseudoreaction, in which \texttt{lipid backbone} is generated:

\texttt{
  0.000631964 1-phosphatidyl-1D-myo-inositol backbone\\
  + 0.00243107 ergosterol + 0.000622407 ergosterol ester backbone\\
  + 0.000135359 fatty acid backbone + ...\\
  --> lipid backbone
}


\begin{table}[ht]
  \centering
    \begin{tabular}{llS}
      ID & Reaction & {\makecell{Computed molecular\\ weight (\SI{}{\gram~\mole^{-1}})}} \\
      \hline
    \texttt{r\_3975} & \makecell{\texttt{palmitate} \\ \texttt{--> 0.255421 fatty acid backbone} \\ \texttt{0.256429 C16:0 chain}} & 742.54 \\
    \texttt{r\_3976} & \makecell{\texttt{palmitoleate} \\ \texttt{--> 0.253405 fatty acid backbone} \\ \texttt{0.254413 C16:1 chain}} & 744.56 \\
    \texttt{r\_3977} & \makecell{\texttt{stearate} \\ \texttt{--> 0.283475 fatty acid backbone} \\ \texttt{0.284483 C18:0 chain}} & 714.49 \\
    \texttt{r\_3978} & \makecell{\texttt{oleate} \\ \texttt{--> 0.281459 fatty acid backbone} \\ \texttt{0.282467 C18:1 chain}} & 716.51 \\
    \end{tabular}
    \caption{ecYeast8 reactions that generate the \texttt{fatty acid backbone} metabolite}
    \label{tab:ecyeast8-fatty-acid-backbone-rxns}
\end{table}

% TODO: Probably needs even more clarification
The model specifies molecular weights for all species in the reaction that generates \texttt{lipid backbone}, expect for \texttt{fatty acid backbone}.
To compute the molecular weight of \texttt{fatty acid backbone}, the reactions that produce this species must be used.
Because of SLIMEr, four reactions in the model produce \texttt{fatty acid backbone} (table~\ref{tab:ecyeast8-fatty-acid-backbone-rxns}).
The model specifies molecular weights for all species in these four reactions, except for \texttt{fatty acid backbone}.
Therefore, using each chemical equation, the molecular weight of \texttt{fatty acid backbone} can be solved for by rearranging Eq.\ \ref{eq:conservation-of-mass} with parameters defined to match each reaction.
However, the molecular weights computed from each equation are different.% , as shown in table~\ref{tab:ecyeast8-fatty-acid-backbone-rxns}.
Because the differences are slight, %, and ultimately I am making a back-of-the-envelope calculation,
I took the mean of the four weights to give \SI{729.53}{\gram~\mole^{-1}}.
Subsequently, the molecular weight of \texttt{lipid backbone} was computed from this mean value and the molecular weights of other species involved in the production of \texttt{lipid backbone}, giving \SI{21.31}{\gram~\mole^{-1}}.
With the molecular weights of \texttt{lipid backbone} and \texttt{lipid chain} defined, the molecular weight of \texttt{lipid} is thus the sum of the two.

A summary of molecular weights can be found in Table~\ref{tab:ecyeast8-mol-weights}.

Subsequently, the mass fraction of each biomass component is computed by dividing the molecular weight of the corresponding pseudometabolite by the molecular weight of biomass:

\begin{table}[ht]
  \centering
  \begin{tabular}{lS}
    Metabolite & {$f_{i}$} \\
    \hline
    Protein & 0.52453 \\
    Carbohydrate & 0.36438 \\
    RNA & 0.06660 \\
    Lipid & 0.03283 \\
    Cofactors & 0.00502 \\
    DNA & 0.00406 \\
    Ions & 0.00258 \\
  \end{tabular}
  \caption{
    $f_{i}$ values for each biomass component.
  }
  \label{tab:model-biomfrac}
\end{table}


\section{Mathematical explanation of the effect of restricting the enzyme pool}
\label{append:model-pool}

Let $\ratioabl$, given by Eq.\ \ref{eq:model-ratio}, depend on $x$:

\begin{equation}
  \ratioabl(x) = \left( \sum_i \frac{f_i}{\griabl(x)} \right) \cdot \frac{\gro(x)}{\biomfrac{protein}}
  \label{eq:model-ratio-x}
\end{equation}

where $x = \epool^{\prime}/\epool$.
The expression in Eq.\ \ref{eq:model-ratio-x} takes into account how $\gro$ and $\griabl$ values vary with $x$, and how $f_{i}$ values are constants.

We thus obtain:
\begin{equation}
  \begin{aligned}
  \ndif{\ratioabl(x)}{x} &= \frac{1}{\biomfrac{protein}} \ndif{}{x} \left[ \left( \sum_i \frac{f_i}{\griabl(x)} \right) \cdot \gro(x) \right]\\
  &= \frac{1}{\biomfrac{protein}} \left[ \left( \sum_i \frac{f_i}{\griabl(x)} \right) \cdot \ndif{\gro(x)}{x} + \gro(x) \ndif{}{x} \left( \sum_i \frac{f_i}{\griabl(x)} \right) \right]\\
  &= \frac{1}{\biomfrac{protein}} \left[ \left( \sum_i \frac{f_i}{\griabl(x)} \right) \cdot \ndif{\gro(x)}{x} - \gro(x) \sum_{i}\left( \frac{f_{i}}{\griabl(x)^{2}} \cdot \ndif{\griabl(x)}{x} \right) \right]
  \end{aligned}
  \label{eq:model-ratio-diff}
\end{equation}

To explain the increase in $\ratioabl$ as $\epool^{\prime}$ increases, I consider the behaviour of $\gro$ and $\griabl$ values with respect to $\epool^{\prime}$ in intervals.

With reference to Fig.\ \ref{fig:model-pool}, consider $0 \leq x \leq 0.5$.
In this region of $x$, based on the observations in the figure, we model $\gro = k_{0}x$ and $\griabl = k_{i}x$, where constants $k_{0}, k_{i} > 0$.
This models how these values initially increase linearly in Fig.\ \ref{fig:model-pool}.
Eq.\ \ref{eq:model-ratio-diff} thus becomes:
\begin{equation}
  \begin{aligned}
  \ndif{\ratioabl(x)}{x} &= \frac{1}{\biomfrac{protein}} \left[ \left( \sum_i \frac{f_i}{k_{i}x} \right) \cdot k_{0} - k_{0}x \sum_{i}\left( \frac{f_{i}}{(k_{i}x)^{2}} \cdot k_{i} \right) \right]\\
  &= \frac{1}{\biomfrac{protein}} \left[ \frac{k_{0}}{x} \sum_i \frac{f_i}{k_{i}} - k_{0}x \left( \sum_{i} \frac{f_{i}}{k_{i}x^{2}} \right) \right]\\
  &= \frac{1}{\biomfrac{protein}} \left[ \frac{k_{0}}{x} \sum_i \frac{f_i}{k_{i}} - \frac{k_{0}}{x} \sum_i \frac{f_i}{k_{i}} \right]\\
  &= 0
  \end{aligned}
  \label{eq:model-ratio-diff-smallx}
\end{equation}

And this explains the constant $\ratioabl$ in this region.

Now, consider $0.5 < x \leq 9$.
In this region, the trajectories of $\griabl$ with respect to time remain linear, but some with changes in slope.
In other words, in a sub-region where the slopes of all $\griabl$ are constant, we can let: $\gro = k_{0}x$ and $\griabl = m_{i}x + c_{i}$, where $k_{0}, m_{i}, c_{i} > 0$.
Eq.\ \ref{eq:model-ratio-diff} thus becomes:
\begin{equation}
  \begin{aligned}
  \ndif{\ratioabl(x)}{x} &= \frac{1}{\biomfrac{protein}} \left[ \left( \sum_i \frac{f_i}{m_{i}x+c_{i}} \right) \cdot k_{0} - k_{0}x \sum_{i}\left( \frac{f_{i}}{(m_{i}x+c_{i})^{2}} \cdot m_{i} \right) \right]\\
  &= \frac{k_{0}}{\biomfrac{protein}} \left[ \left( \sum_i \frac{f_i}{m_{i}x+c_{i}} \right) - x \left( \sum_{i} \frac{f_{i}m_{i}}{(m_{i}x+c_{i})^{2}} \right) \right]\\
  &= \frac{k_{0}}{\biomfrac{protein}} \sum_{i} \left[ \frac{f_i}{m_{i}x+c_{i}} - \frac{xf_{i}m_{i}}{(m_{i}x+c_{i})^{2}} \right]\\
  &= \frac{k_{0}}{\biomfrac{protein}} \sum_{i} \left[ \frac{f_{i}c_{i}}{(m_{i}x+c_{i})^{2}} \right]
  \end{aligned}
  \label{eq:model-ratio-diff-midx}
\end{equation}

As $f_{i}, c_{i}, m_{i} > 0$ for all biomass components $i$, and $k_{0} > 0$, we get $\ndif{\ratioabl(x)}{x} > 0$ regardless of the values that these constants take.
Because $k_{0}$ does not change over the region of $x$ considered, $m_{i}$, $c_{i}$, and $x$ values thus determine the magnitude of $\ndif{\ratioabl(x)}{x}$.
If within a region of $x$, $m_{i}$ and $c_{i}$ values remain constant for all $i$, then as $x$ increases, $\ndif{\ratioabl(x)}{x}$ should decrease --- this is certainly the case, as can be observed in Fig.\ \ref{fig:model-pool}.

Lastly, consider $x > 9$.
In this region, $\gro$ becomes constant, thus we let $\gro = k_{0}$.
We keep $\griabl = m_{i}x + c_{i}$, and as before, $k_{0}, m_{i}, c_{i} > 0$.
Eq.\ \ref{eq:model-ratio-diff} thus becomes:
\begin{equation}
  \begin{aligned}
  \ndif{\ratioabl(x)}{x} &= \frac{1}{\biomfrac{protein}} \left[ 0 - k_{0} \sum_{i}\left( \frac{f_{i}}{(m_{i}x+c_{i})^{2}} \cdot m_{i} \right) \right]\\
  &= -\frac{k_{0}}{\biomfrac{protein}} \sum_{i}\left[ \frac{f_{i}m_{i}}{(m_{i}x+c_{i})^{2}} \right]
  \end{aligned}
  \label{eq:model-ratio-diff-largex}
\end{equation}

This predicts \emph{decreasing} $\ratioabl$ as $x$ increases in this region.
Because $k_{0}$ is constant in this region, the rate of this decrease is thus controlled by $m_{i}$ and $c_{i}$ values.
As each $\griabl$ trajectory becomes flat as $x$ increases, each $\frac{f_{i}m_{i}}{(m_{i}x+c_{i})^{2}}$ term becomes zero, thus shrinking the magnitude of $\ndif{\ratioabl(x)}{x}$.
Finally, as all $\griabl$ trajectories become flat at $x > 15$, $\ndif{\ratioabl(x)}{x} = 0$.
