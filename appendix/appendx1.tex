\chapter{Insert appendix title here}
\label{append:model}

\section{Mathematical explanation of the effect of restricting the enzyme pool}
\label{append:model-pool}

Let $\ratioabl$, given by equation~\ref{eq:model-ratio}, depend on $x$:

\begin{equation}
  \ratioabl(x) = \left( \sum_i \frac{f_i}{\griabl(x)} \right) \cdot \frac{\gro(x)}{\biomfrac{protein}}
  \label{eq:model-ratio-x}
\end{equation}

where $x = \epool^{\prime}/\epool$.
The expression in Eq.\ \ref{eq:model-ratio-x} takes into account how $\gro$ and $\griabl$ values vary with $x$, and how $f_{i}$ values are constants.

We thus obtain:
\begin{equation}
  \begin{aligned}
  \ndif{\ratioabl(x)}{x} &= \frac{1}{\biomfrac{protein}} \ndif{}{x} \left[ \left( \sum_i \frac{f_i}{\griabl(x)} \right) \cdot \gro(x) \right]\\
  &= \frac{1}{\biomfrac{protein}} \left[ \left( \sum_i \frac{f_i}{\griabl(x)} \right) \cdot \ndif{\gro(x)}{x} + \gro(x) \ndif{}{x} \left( \sum_i \frac{f_i}{\griabl(x)} \right) \right]\\
  &= \frac{1}{\biomfrac{protein}} \left[ \left( \sum_i \frac{f_i}{\griabl(x)} \right) \cdot \ndif{\gro(x)}{x} - \gro(x) \sum_{i}\left( \frac{f_{i}}{\griabl(x)^{2}} \cdot \ndif{\griabl(x)}{x} \right) \right]
  \end{aligned}
  \label{eq:model-ratio-diff}
\end{equation}

To explain the increase in $\ratioabl$ as $\epool^{\prime}$ increases, I consider the behaviour of $\gro$ and $\griabl$ values with respect to $\epool^{\prime}$ in intervals.

With reference to Fig.\ \ref{fig:model-pool}, consider $0 \leq x \leq 0.5$.
In this region of $x$, based on the observations in the figure, we model $\gro = k_{0}x$ and $\griabl = k_{i}x$, where constants $k_{0}, k_{i} > 0$.
This models how these values initially increase linearly in figure~\ref{fig:model-pool}.
Equation~\ref{eq:model-ratio-diff} thus becomes:
\begin{equation}
  \begin{aligned}
  \ndif{\ratioabl(x)}{x} &= \frac{1}{\biomfrac{protein}} \left[ \left( \sum_i \frac{f_i}{k_{i}x} \right) \cdot k_{0} - k_{0}x \sum_{i}\left( \frac{f_{i}}{(k_{i}x)^{2}} \cdot k_{i} \right) \right]\\
  &= \frac{1}{\biomfrac{protein}} \left[ \frac{k_{0}}{x} \sum_i \frac{f_i}{k_{i}} - k_{0}x \left( \sum_{i} \frac{f_{i}}{k_{i}x^{2}} \right) \right]\\
  &= \frac{1}{\biomfrac{protein}} \left[ \frac{k_{0}}{x} \sum_i \frac{f_i}{k_{i}} - \frac{k_{0}}{x} \sum_i \frac{f_i}{k_{i}} \right]\\
  &= 0
  \end{aligned}
  \label{eq:model-ratio-diff-smallx}
\end{equation}

And this explains the constant $\ratioabl$ in this region.

Now, consider $0.5 < x \leq 9$.
In this region, the trajectories of $\griabl$ with respect to time remain linear, but some with changes in slope.
In other words, in a sub-region where the slopes of all $\griabl$ are constant, we can let: $\gro = k_{0}x$ and $\griabl = m_{i}x + c_{i}$, where $k_{0}, m_{i}, c_{i} > 0$.
Equation~\ref{eq:model-ratio-diff} thus becomes:
\begin{equation}
  \begin{aligned}
  \ndif{\ratioabl(x)}{x} &= \frac{1}{\biomfrac{protein}} \left[ \left( \sum_i \frac{f_i}{m_{i}x+c_{i}} \right) \cdot k_{0} - k_{0}x \sum_{i}\left( \frac{f_{i}}{(m_{i}x+c_{i})^{2}} \cdot m_{i} \right) \right]\\
  &= \frac{k_{0}}{\biomfrac{protein}} \left[ \left( \sum_i \frac{f_i}{m_{i}x+c_{i}} \right) - x \left( \sum_{i} \frac{f_{i}m_{i}}{(m_{i}x+c_{i})^{2}} \right) \right]\\
  &= \frac{k_{0}}{\biomfrac{protein}} \sum_{i} \left[ \frac{f_i}{m_{i}x+c_{i}} - \frac{xf_{i}m_{i}}{(m_{i}x+c_{i})^{2}} \right]\\
  &= \frac{k_{0}}{\biomfrac{protein}} \sum_{i} \left[ \frac{f_{i}c_{i}}{(m_{i}x+c_{i})^{2}} \right]
  \end{aligned}
  \label{eq:model-ratio-diff-midx}
\end{equation}

As $f_{i}, c_{i}, m_{i} > 0$ for all biomass components $i$, and $k_{0} > 0$, we get $\ndif{\ratioabl(x)}{x} > 0$ regardless of the values that these constants take.
Because $k_{0}$ does not change over the region of $x$ considered, $m_{i}$, $c_{i}$, and $x$ values thus determine the magnitude of $\ndif{\ratioabl(x)}{x}$.
If within a region of $x$, $m_{i}$ and $c_{i}$ values remain constant for all $i$, then as $x$ increases, $\ndif{\ratioabl(x)}{x}$ should decrease --- this is certainly the case, as can be observed in figure~\ref{fig:model-pool}.

Lastly, consider $x > 9$.
In this region, $\gro$ becomes constant, thus we let $\gro = k_{0}$.
We keep $\griabl = m_{i}x + c_{i}$, and as before, $k_{0}, m_{i}, c_{i} > 0$.
Equation~\ref{eq:model-ratio-diff} thus becomes:
\begin{equation}
  \begin{aligned}
  \ndif{\ratioabl(x)}{x} &= \frac{1}{\biomfrac{protein}} \left[ 0 - k_{0} \sum_{i}\left( \frac{f_{i}}{(m_{i}x+c_{i})^{2}} \cdot m_{i} \right) \right]\\
  &= -\frac{k_{0}}{\biomfrac{protein}} \sum_{i}\left[ \frac{f_{i}m_{i}}{(m_{i}x+c_{i})^{2}} \right]
  \end{aligned}
  \label{eq:model-ratio-diff-largex}
\end{equation}

This predicts \emph{decreasing} $\ratioabl$ as $x$ increases in this region.
Because $k_{0}$ is constant in this region, the rate of this decrease is thus controlled by $m_{i}$ and $c_{i}$ values.
As each $\griabl$ trajectory becomes flat as $x$ increases, each $\frac{f_{i}m_{i}}{(m_{i}x+c_{i})^{2}}$ term becomes zero, thus shrinking the magnitude of $\ndif{\ratioabl(x)}{x}$.
Finally, as all $\griabl$ trajectories become flat at $x > 15$, $\ndif{\ratioabl(x)}{x} = 0$.
