\chapter{Collocation Method at the Wake Edges}
\label{append:collocation}

\setcounter{equation}{0}

A collocation based gradient method is formulated in \citet[sec. 5]{Gray:2004:SJoSC}. In that
paper it is proved that the method, although applicable to a general Laplace problem, is not
applicable to the crack problem. The difficulty arises from an additional constant in the shape
functions for the most singular point. However, the zero value of 

% To finalise the applicability of this method, prior to its derivation, and in respect to the
% necessity of an internal and external domain, it is proposed that the method does not suffer the
% same difficulties at an edge as the Galerkin method.
% 
% To prove this consider a general point of interest lying on either the top and bottom
% surfaces of the wake, $S^{+}$ and $S^{-}$. Once more, the average of the velocities of the top and
% bottom are saught. This is written as
% \begin{multline}
% \frac{1}{2} \left( \pdif{\phi^{+}}{E_{k}} + \pdif{\phi^{-}}{E_{k}} \right) = \left\{ \lim_{\pint
% \rightarrow P} - \lim_{\pext \rightarrow P}
% \right\} \int_{S} \phi \frac{\partial^{2}G}{\partial\E_{k}\partial\n} \: dS \\
% = \left\{ \lim_{\pint \rightarrow P} - \lim_{\pext \rightarrow P}
% \right\} \int_{S^{+}} \phi^{+} \frac{\partial^{2}G}{\partial\E_{k}\partial\n^{+}}
% \: dS + \int_{S^{-}} \phi^{-} \frac{\partial^{2}G}{\partial\E_{k}\partial\n^{-}}
% \: dS
% \end{multline}
% Moving on to considering a point lying on the division between $S^{+}$ and $S^{-}$, the above
% becomes
% \begin{equation}\label{eq:edge_collocation}
% \pdif{\phi}{E_{k}} = \left\{ \lim_{\pint \rightarrow P} - \lim_{\pext \rightarrow P}
% \right\}
% \int_{S^{+}} \phi^{+} \frac{\partial^{2}G}{\partial\E_{k}\partial\n^{+}} \: dS +
% \int_{S^{-}} \phi^{-} \frac{\partial^{2}G}{\partial\E_{k}\partial\n^{-}} \: dS
% \end{equation}
% given that $\phi^{+} + \phi^{-} = \phi$. The difference between the internal and external limits
% will cancel all of the non-singular integrals, hence the remaining integrals are just the singular
% ones in $S^{+}$ and $S^{-}$. The right hand side of equation~\eqref{eq:edge_collocation} can
% easily be combined into a single surface by

The development of the collocation method is very similar to that of the edge adjacent integral,
expect that only one polar transformation is available. To begid
effect of
differ


% END OF CHAPTER