\section{Towards a coarse-grained, phenomenological model of the yeast metabolic cycle}
\label{sec:concl-coarse}

Insert content here.

% ---- FROM BIOLOGY CHAPTER ----

% [MOVED FROM BEGINNING]
In this study, I show that yeast cells exhibit oscillations in the fluorescence of flavins that are consistent with previous descriptions of the yeast metabolic cycle.
These oscillations synchronise with the cell division cycle in permissive conditions.
Such flavin oscillations showed different behaviour in respiratory conditions.
Additionally, I found that cells were able to autonomously reset the phase of their metabolic cycle, independently of the cell division cycle, in response to abrupt nutrient changes.
Finally, I showed that deletion strains generated flavin oscillations that exhibited different behaviour from dissolved oxygen oscillations from chemostat conditions.

My results confirm that the metabolic cycle is an autonomous oscillator, independent of the cell division cycle, and responds to perturbations in the environment\ldots{}

\subsection{Broad implications}
\label{sec:biology-discussion-implications}

\begin{itemize}
\item Confirms utility of using flavin as an indicator of YMC.  Confirms utility of our microfluidics platform.  Our technology produces vast amounts of data, so our computational tools are especially useful.  Confirms utility of our analysis methods.
\item Confirms model in which autonomous, single-cell YMC responds to nutrient conditions and adjusts period/phase accordingly.  YMC-CDC synchrony holds in a window of frequencies -- outside this, YMCs still occur but coupling changes.  Points to model that YMCs govern cell resources and only if they are adequate does the cell initiate a CDC.  Suggestion that deletion strain impair this model.
\item Autonomy of YMCs (cell-autonomous and independence from CDC), starvation in chemostat, and presence of subpopulations may explain differences between chemostat and single-cell based studies.  My study makes the case for studying this system in single-cell as it gives more info than chemostat.
\item Something fundamental to control of biological rhythms?  Links to other processes?
\begin{itemize}
\item pH and growth rate oscillations (\textcite{luziaPHDependenciesGlycolytic})
\end{itemize}
\end{itemize}



\section{Further avenues of study}
\label{sec:concl-further}
