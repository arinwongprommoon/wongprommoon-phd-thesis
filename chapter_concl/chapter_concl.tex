\chapter{Conclusions}
\label{ch:concl}

Although biological rhythms are crucial for living organisms to control their physiological processes in response to external conditions, not all biological rhythms are well-characterised.
In contrast to the circadian rhythm and the cell division cycle, our knowledge of the biochemical basis of the yeast metabolic cycle is incomplete.
It is unclear which mechanism drives cell division-independent cycling of biosynthesis observed in the yeast metabolic cycle, and it is also unclear what mechanism is responsible for the adaptation of the yeast metabolic cycle to changing demands \parencite{zylstraMetabolicDynamicsCell2022}.
Additionally, chemostat-based and single-cell experiments led to conflicting conclusions about the yeast metabolic cycle because each type of experiment creates different culture conditions and have different types of measurements.

The primary goal of this thesis was thus to develop an explanation to reconcile chemostat and single-cell studies on the yeast metabolic cycle.
Specifically, I developed such explanations through testing whether specific characteristics of the yeast metabolic cycle as observed in the chemostat could be recapitulated in single-cell microfluidics.
In addition, this thesis aimed to show whether proteomic constraints could explain why the yeast cell temporally segregates biosynthetic events as it progresses through the yeast metabolic cycle.
This secondary goal provided a coarse-grained explanation of a model of the yeast metabolic cycle as a fundamental metabolic adaptation to physiological constraints.


\section{Microfluidics and fluorescence microscopy for cellular metabolic cycles}
\label{sec:concl-biology}

In Chapter~\ref{ch:biology}, I used the ALCATRAS \parencite{craneMicrofluidicSystemStudying2014} single-cell microfluidics platform to physically separate budding yeast cells and fluorescence microscopy to monitor the yeast metabolic cycle and the cell division cycle.
I showed that yeast cells independently generated flavin-based single-cell metabolic cycles.
In addition, a specific phase of such cycles likely gated the cell division cycle, as evidenced by decoupling between the metabolic and cell division cycles during starvation.
I further showed that the metabolic cycle was retained in nutrient perturbations and in deletion strains.
In particular, I showed that cells generated such cycles in potassium-deficient conditions, contrary to \textcite{oneillEukaryoticCellBiology2020}.
I also showed that that \textit{zwf1$\Delta$} and \textit{tsa1$\Delta$ tsa2$\Delta$} cells generated flavin cycles whose waveforms differed from cycles of dissolved oxygen previously observed in the chemostat \parencite{tuCyclicChangesMetabolic2007,caustonMetabolicCyclesYeast2015}.

My results suggest that the yeast metabolic cycle is likely an intrinsic cycle in budding yeast that oscillates within a range of natural frequencies, but the cell is able to adjust this frequency to respond to nutrient conditions.
If conditions are permissive, the metabolic cycle provides windows of opportunities for the cell division cycle to be initiated.
Otherwise, if conditions are not permissive, the metabolic cycle continues while the cell division cycle is halted at a gap phase (G\textsubscript{1} or G\textsubscript{2}/M).

My results further suggest that the presence of sub-populations in the yeast culture could explain the discrepancy between single-cell and chemostat observations.
The idea of sub-populations that stagger their entry into the yeast metabolic cycle has been suggested before to explain chemostat observations \parencite{burnettiCellCycleStart2016}, and the presence of genetically identical sub-populations that respond differently to the same nutrient perturbation has later been shown \parencite{bagameryPutativeBetHedgingStrategy2020}.
Additionally, genetically identical sub-populations that have different levels of sensitivities to an inhibitor may also explain chemostat oscillations \parencite{smithTheoryChemostatDynamics1995}.

To provide more clarity to the role of nutrient storage in the yeast metabolic cycle, future work may include experiments with lipid synthesis-deficient strains.
Additionally, a feast-and-famine experimental set-up which better emulates chemostat conditions could lead to a clearer explanation for previous chemostat-based studies.
The glucose pulses imposed by this set-up may lead to a mathematical model of coupled oscillations that links the intrinsic yeast metabolic cycle to extrinsically-imposed oscillations.


\section{Analysis of oscillatory time series in the yeast metabolic cycle}
\label{sec:concl-analysis}

Because the ALCATRAS platform produces large datasets of time series, in Chapter~\ref{ch:analysis}, I developed a series of time series analysis methods.
These methods clean data, visualise groups in a dataset, detect rhythmicity, estimate periodicity of signals, and detect synchrony between two types of signals.
I showed that a high-pass filter offered good control over the frequency domain of time series.
Subsequently, I showed that dimension-reduction (UMAP) and clustering (modularity clustering) methods agreed on a division between oscillatory and non-oscillatory time series in a dataset.
Following this, I demonstrated that a statistical method based on the power spectrum and a support vector classifier offer modest performances in rhythmicity detection.
Additionally, I showed that the autocorrelation function could be used to estimate periodicity and noise parameters from synthetic data.
However, my current implementation of the autocorrelation function has limited ability in characterising noise parameters from real data.
Finally, I showed that the cross-correlation function could be used to quantify the shift of one type of time series relative to another, across a population of paired time series

% MOVE TO CHAPT 4 DISCUSSION?  IT LOOKS OUT OF PLACE HERE.
Rhythmicity detection is complicated by its different definitions depending on the approach --- reflected in the variety of rhythmicity detection methods compared in Chapter~\ref{ch:analysis}.
From a signal-processing perspective, it can be defined as finding a strong signal within a range of expected frequencies \parencite{zielinskiStrengthsLimitationsPeriod2014}.
However, from a data science perspective, rhythmicity detection can be seen as identifying the values of a set of time series features that best discriminate between non-oscillatory and oscillatory time series.

To improve the usefulness of the time series analysis methods, further refinement is needed.
To make the clustering methods and the support vector classifier generalisable, we require a large enough dataset of signals that includes a variety of oscillation types and shapes, and hyperparameter tuning.
Furthermore, to improve the ability of the autocorrelation function to infer noise properties of real data, a broader range of noise parameters should be simulated.
Such simulations would provide addition information that leads to give a more precise relationship between noise parameters and the shape of the autocorrelation function.
A precise way to detect of noise parameters can then be useful to compare the noise from different environmental conditions and imaging methods.
With the improvements in place, the methods developed in Chapter~\ref{ch:analysis} can form a powerful time series analysis pipeline for oscillatory signals from any natural phenomenon.


\section{Modelling yeast biosynthesis strategies under constraints}
\label{sec:concl-model}

Finally, in Chapter~\ref{ch:model}, I used an enzyme-constrained genome-scale model of budding yeast and flux balance analysis to address whether a limited proteome pool leads to a preference of sequential biosynthesis over parallel biosynthesis.
In this chapter, I used the novel approach of ablating components of the biomass reaction to simulate temporal segregation of biosynthesis, and devised a time ratio that indicates whether sequential or parallel biosynthesis was more advantageous.
I showed that sequential scheduling of biosynthesis was advantageous across deletion strains, and became more advantageous if the proteome pool was smaller.
However, I also showed that parallel scheduling of biosynthesis became advantageous when both carbon and nitrogen sources were limiting.
This observation may be explained by the synthesis pathways across different biomass components sharing enzymes.

The advantage of sequential biosynthesis may explain why the yeast cell temporally partitions biosynthesis of biomass components across phases of the yeast metabolic cycle, even when such partitioning is not needed to coordinate events of the cell division cycle --- e.g.\ when the metabolic cycle proceeds without cell division during starvation.
Furthermore, the advantage of parallel biosynthesis in some conditions suggests that the metabolic cycle may cease to occur if nutrient conditions are too harsh, in concordance with studies that suggest disappearance of the metabolic cycle in extreme conditions \parencite{oneillEukaryoticCellBiology2020}.
% IDEA: Do the grid plot for the deletion strains?
To improve model predictability, this study could be extended by using derivations of flux balance analysis that account for compartmentalisation or temporality, such as dynamic flux balance analysis.


\section{Summary and broader context of thesis}
\label{sec:concl-summary}

Put together, single-cell analysis of flavin-based yeast metabolic cycles and modelling of the metabolism of budding yeast may provide a mechanistic explanation for such an under-characterised biological rhythm.
I envisage a biochemical explanation for the autonomous generation of the yeast metabolic cycle and for its response to nutrient conditions.
The biochemistry of the yeast metabolic cycle could then be modelled using techniques such as flux balance analysis.
In addition, robust time series analysis methods would be able to discover classes of oscillations within a microfluidics experiment that could correspond to sub-populations in the culture.
Identification of such sub-populations could then potentially reconcile results of single-cell and chemostat experiments.

Biological rhythms are an important physiological adaptation of all living organisms.
This thesis, in sum, shows the robustness of the yeast metabolic cycle and relates it to resource allocation strategies, thus potentially shedding light on what could be a fundamental biological process.

The presence of the yeast metabolic cycle leads to questions about the benefits of biological oscillations that justify their existence.
The benefits of the circadian rhythm and the cell division cycle are clear: circadian rhythms synchronise physiological processes to the light-dark cycle, while the cell division cycle coordinates resource-intensive processes with the presence of nutrient stores and maintains genetic fidelity.
In the yeast metabolic cycle, it has been proposed that temporal partitioning of biosynthesis optimises the use of the limited proteome for cell growth \parencite{oneillEukaryoticCellBiology2020,takhaveevTemporalSegregationBiosynthetic2023} so that biomass components are synthesised `just-in-time' for when they are required for phase of the cell division cycle \parencite{zylstraMetabolicDynamicsCell2022}.
However, the continued presence of the yeast metabolic cycle when the cell division cycle is halted implies that the metabolic cycle has functions other than control of the cell division cycle.
Alternatively, the metabolic cycle may function as a environment-sensitive background rhythm to ensure that just-in-time biosynthesis for the cell division cycle can be switched on rapidly, as supported by my observation of rapid responses of cells to restoration of glucose after starvation.

Furthermore, the common control features of the yeast metabolic cycle and other biological rhythms may reflect a common evolutionary or functional origin.
The yeast metabolic cycle may be mediated via post-translational modifications, as evidenced by how the proteome shows far less cyclic variation than the transcriptome \parencite{felthamTranscriptionalChangesAre2020} or the metabolome \parencite{oneillEukaryoticCellBiology2020}.
Specifically, studies have highlighted the role of peroxiredoxin oxidation cycles in the integrity of the yeast metabolic cycle in chemostats \parencite{caustonMetabolicCyclesYeast2015,amponsahPeroxiredoxinsCoupleMetabolism2021}, while other studies highlight the importance of chromatin remodelling in this biological rhythm \parencite{nocettiNucleosomeRepositioningUnderlies2016,gowansINO80ChromatinRemodeling2018}.

Continued peroxiredoxin oxidation cycles have also been observed in circadian rhythms of transcription-disabled \textit{Ostreococcus tauri} cells \parencite{oneillCircadianRhythmsPersist2011} and DNA-lacking human red blood cells \parencite{oneillCircadianClocksHuman2011}.
Furthermore, human red blood cells also exhibit NADH/NADPH and ATP cycles, independent of the glycolytic cycle \parencite{oneillCircadianClocksHuman2011}, similar to the yeast metabolic cycle \parencite{papagiannakisAutonomousMetabolicOscillations2017}.
In addition, the glycolytic cycle is regulated through solely biochemical means \parencite{ghoshOscillationsGlycolyticIntermediates1964,higginsChemicalMechanismOscillation1964}, so this cycle may represent a basal mechanism of regulation of biological rhythms.
Such similarities, across a range of biological kingdoms, thus strongly suggest a non-genetic common origin of biological rhythms.
