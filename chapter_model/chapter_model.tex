\chapter{Modelling the yeast metabolic cycle}
\label{ch:model}

Discuss difficulty of having a fine-grained model: too many unknowns with this biological system.
Especially that we don't know much about the molecular mechanisms (link to introduction) and the main read-outs of single-cell studies (NAD(P)H, flavins) are 'aggregate' measures of enzymatic activity and realistically only function to monitor the cellular redox state.

\section{Modelling temporal partitioning of biosynthesis}
\label{sec:model-temporal}
Research question: Given a finite amount of enzymes and the nutrient conditions yeast cells are subject to, is it more efficient for the cells to temporally partition synthesis of macromolecules (lipid, carbohydrates, amino acids), or to synthesise all of them at the same time?
If so, does the time scale fit with that of the yeast metabolic cycle?

There have been studies that attempt to address similar questions.
% add notes from org-roam about this publication
\textcite{takhaveevTemporalSegregationBiosynthetic2023} showed that in different stages of the cell division cycle, the level of synthesis of each class of macromolecule is different.
They used single-cell NAD(P)H responses (metabolic cycles) to inhibitors that block synthesis of each class of macromolecule to infer the activity of biosynthesis of classes of macromolecules over a cell division cycle.
Then, they used these activities as coefficients for FBA on a modified thermodynamic-stoichiometric metabolic model at each time point to deduce biomass production rates.
Does not suggest that synthesis of each class of macromolecule excludes all others (which makes sense).
Importantly, this study does not predict timescale.
% please clarify this sentence
Rather, confirms that timescale of observed synthesis events matches simulations.

My approach is to use a genome-scale model of /Saccharomyces cerevisiae/ and perform flux balance analysis (FBA).
Specifically, I use the enzyme-constrained Yeast8 (ecYeast8) model \parencite{luConsensusCerevisiaeMetabolic2019}.
Traditional genome-scale models assume that the uptake rate of carbon source limits production; however, levels of each enzyme also restrict reaction fluxes.
Models like \textcite{sanchezImprovingPhenotypePredictions2017} and \textcite{elsemmanWholecellModelingYeast2022} -- the latter of which imposes a ribosome capacity constraint and additionally imposes compartment constraints -- constrain the total sum of fluxes based on a defined total amount of enzyme.
Using an enzyme-constrained model also fits the assumption that there is a fixed number of amino acids the cell has to distribute \parencite{weisseMechanisticLinksCellular2015}.

ecYeast8 is derived from Yeast8 by applying the GECKO formalism.
In a conventional genome-scaled model, metabolic fluxes through reactions are constrained between a lower bound and an upper bound.
This is to narrow down the solution space when optimising the objective function.
The GECKO formalism imposes an additional constraint on the metabolic fluxes based on the concentration of the enzyme that catalyses the reaction.
As a simple example, for a reaction $R_{j}$ catalysed by enzyme $E_{i}$ (and $E_{i}$ only), the formalism imposes:

\begin{equation}
  v_{j} \le k_{cat}^{ij} \cdot [E_{i}]
\end{equation}

In other words, the flux through the reaction must not exceed $v_{max}$.
Slightly different relationships are imposed for other types of enzymes, i.e. isozymes, promiscuous enzymes, and enzyme complexes.

To apply this constraint, GECKO modifies reactions in the genome-scaled model it is applied to.
For example, if the model defines a reaction $R_{j}$ catalysed by $E_{i}$:
\begin{equation}
  \ce{ A + B ->[E_{i}] C + D }
\end{equation}

GECKO adds a term to the equation to make it:
\begin{equation}
  \ce{ n_{ij}E_{i} + A + B -> C + D }
\end{equation}
with $n_{ij} = \frac{1}{k_{cat}^{ij}}$.
This transformation, adding the enzyme as a pseudo-reactant, is based on the interpretation that the system uses some amount of enzyme at a specific time to catalyse the flux going through the reaction.

GECKO then also creates an overall enzyme pseudo-reaction $EU_{i}$: $\ce{ -> E_{i}}$, in order to preserve the mass balance of enzymes.  The flux $e_{i}$ of this pseudo-reaction is thus constrained: $0 \le e_{i} \le [E_{i}]$.

Slightly different formalisms are used for reversible reactions, isozymes, promiscuous enzymes, and complexes.  Namely:
\begin{itemize}
  \item Reversible reactions are modelled as the forward and reverse reactions separately.
  \item For isozymes, the original reaction is copied $n$ times and each has as isozyme catalysing the reaction.
  In addition, there is an `arm' reaction to act as an intermediate between the substrate and the products.
  \item No actions are needed for promiscuous enzymes.
  \item Complexes: one reaction that uses all subunit proteins that all share the same $k_{cat}$ value.
\end{itemize}

GECKO repeats this logic for all enzyme-catalysed reactions in the genome-scale model to create an enzyme-constrained model.
GECKO takes $k_{cat}$ values from BRENDA [CITATION NEEDED] and enzyme data from SWISSPROT and KEGG [CITATIONS NEEDED], including molecular weight of protein and associated pathways.
To constrain enzyme levels in the model, GECKO goes through each enzyme and defines upper bounds that correspond to literature values of enzyme properties.
For the proteins that do not have literature values, the constraint is based on the protein mass fraction and a saturation factor.
Then, GECKO scales the amino acid composition so as to reflect the total protein content.
And then changes the carbohydrate composition based on the assumption that a change in the amino acid composition is offset by the reverse change in the carbohydrate composition;
experimental data justifies this assumption.

Alternatively, rather than constraining each enzyme, GECKO can impose a global constraint -- i.e. behave as if none of the proteins have literature values.
GECKO accomplishes this by introducing a pseudo-reaction $ER_{pool}$: $\ce{ -> E_{pool} }$, with a flux $e_{pool} \le (P_{total} - P_{measured}) \cdot f \cdot \sigma$, where $P$ represents protein fraction, $f$ the mass fraction of proteins, and $\sigma$ a parameter that represents the average saturation of enzymes.
And replacing all other pseudo-reactions and replacing them with a global pseudo-reaction $ER_{i}$: $\ce{ MW_{i} E_{pool} -> E_{i} }$.
This approach is similar to FBA with molecular crowding.

In my approach, I exclude each class of macromolecule in turn in the objective function (biomass-generating reaction) and optimise the model.
Specifically, given the objective function:

\texttt{
  55.3 atp\_c + 55.3 h2o\_c + lipid\_c + protein\_c + carbohydrate\_c + dna\_c + rna\_c + cofactor + ion \\
  --> 55.3 adp\_c + biomass\_c + 55.3 h\_c + 55.3 pi\_c
}

There are five classes of macromolecules: lipids, proteins, carbohydrates, DNA, and RNA.
To have the cell prioritise biosynthesis of lipids, I set the stoichiometries of all five classes except for lipids to zero in the above equation, giving:

\texttt{
  55.3 atp\_c + 55.3 h2o\_c + lipid\_c + cofactor + ion \\
  --> 55.3 adp\_c + biomass\_c + 55.3 h\_c + 55.3 pi\_c
}

And I repeated this for the other four macromolecules.

For each version of the model, I computed the objective flux and the flux through each of the five pseudoreactions for the synthesis of the five classes of macromolecules, summarised here:

[INSERT BAR CHART]

% Edit this when I improve my logic, with better molecular weights
Then, using the optimised flux of the objective function, I estimates the time it takes for the cell to synthesise all macromolecules of that class enough to create a new cell.
% TODO: improve wording of this sentence to make it more clear
I then compared the estimated timescale for to the timescale estimated by the unmodified biomass reaction.

For this to be possible, I needed to compute molecular weights for bulk metabolites that represent macromolecules in the ecYeast8 model.
The ecYeast8 model does not specify the molecular weights of these bulk metabolites.
The bulk metabolites includes: lipids (\texttt{s\_1096}), proteins (\texttt{s\_3717}), carbohydrates (\texttt{s\_3718}), RNA (\texttt{s\_4049}), and DNA (\texttt{s\_3720}).
Additionally, I computed molecular weights for bulk metabolites that represent cofactors (\texttt{s\_4205}), and ions (\texttt{s\_4206}), as they are part of the objective function too.
I applied the procedure that \textcite{takhaveevTemporalSegregationBiosynthetic2023} used to compute the molecular weight of biomass in their model.
Namely, I assumed that in reactions that produce the bulk metabolites, there is conservation of mass, and therefore,

\begin{equation}
\label{eq:conservation-of-mass}
    \sum_{s}(\text{molar mass}_{s})(\text{stoichiometric coefficient}_{s}) - \sum_{p}(\text{molar mass}_{p})(\text{stoichiometric coefficient}_{p})
\end{equation}

where $s = 1, ... (\text{number of substrates})$ represents substrates and $p = 1, ... (\text{number of products})$ represents products of the reaction in question.

This procedure must be applied because the ecYeast8 model does not necessarily imply that each molecule of macromolecule corresponds to a single molecule in a real cell.
Certainly, it is not possible to create a model with such an implication because each species of a macromolecule represented by the bulk metabolite have different molecular weights, and because most macromolecules are polymers -- the model exhibits the average molecular weight of a monomer in such instances.

To compute the molecular weight of the carbohydrate metabolite, I inspected reaction \texttt{r\_4048}:

\texttt{
  0.748514964334505 (1->3)-beta-D-glucan + 0.25009165448919 (1->6)-beta-D-glucan + 0.361414527575564 glycogen + 0.710939625370425 mannan + 0.138275712087142 trehalose \\
  --> carbohydrate
}

Here, the molecular weights of all species except for \texttt{carbohydrate}, the bulk metabolite, are represented in the model.
Thus, equation \ref{eq:conservation-of-mass} can be applied to compute the molecular weight of the carbohydrate metabolite.

The same process can be applied to compute the molecular weights of the DNA, RNA, cofactor, and ion metabolites as the equations similarly have reactants with molecular weights represented in the model and only the bulk metabolite, the sole product, as the metabolite with an unspecified molecular weight.
The DNA molecular weight was computed from reaction \texttt{r\_4050}:

\texttt{
  0.00359999993816018 dAMP + 0.00240000011399388 dCMP + 0.00240000011399388 dGMP + 0.00359999993816018 dTMP
  \\ --> DNA
}

The RNA molecular weight was computed from reaction \texttt{r\_4049}:

\texttt{
  0.0445348319234424 AMP + 0.043276239184487 CMP + 0.0445348319234424 GMP + 0.0579920969091588 UMP
  \\ --> RNA
}

The cofactor molecular weight was computed from reaction \texttt{r\_4598}:

\texttt{
  0.000190000006114133 coenzyme A + 9.99999974737875e-06 FAD + 0.00264999992214143 NAD + 0.000150000007124618 NADH + 0.000569999974686652 NADP(+) + 0.00270000007003546 NADPH + 0.000989999971352518 riboflavin + 1.20000004244503e-06 TDP + 6.34000025456771e-05 THF + 9.99999997475243e-07 heme a
  \\ --> cofactor
}

And the ion molecular weight was computed from reaction \texttt{r\_4599}:
\texttt{
    3.04000004689442e-05 iron(2+) + 0.00362999993376434 potassium + 0.003969999961555 sodium + 0.0199999995529652 sulphate + 0.00129000004380941 chloride + 0.00273000006563962 Mn(2+) + 0.000747999991290271 Zn(2+) + 0.000216999993426725 Ca(2+) + 0.0012425429886207 Mg(2+) + 0.000659000012092292 Cu(2+)
    \\ --> ion
}

Other metabolites were less straightforward and required some judgement calls.
To compute the molecular weight of the protein metabolite, I inspected reaction \texttt{r\_4047}:

\texttt{
  0.527012401964609 Ala-tRNA(Ala) + 0.184592178971158 Arg-tRNA(Arg) + 0.116820320233205 Asn-tRNA(Asn) + ... + 0.303939602869103 Val-tRNA(Val)
  \\ --> 0.527012401964609 tRNA(Ala) + 0.184592178971158 tRNA(Arg) + 0.116820320233205 tRNA(Asn) + ... + 0.116820320233205 tRNA(Val) + protein
}

In this reaction, the aminoacyl-tRNA reactants are represented in the form of the atoms that make up the aminoacyl residues plus \texttt{R} to represent the tRNA, and the tRNA products are represented as \texttt{RH}.
For example, \texttt{Ala-tRNA(Ala)}, alanyl-tRNA, is represented as \texttt{C3H7NOR}.
The protein pseudoreaction shows how different proportions of each aminoacyl-tRNA combine to form the cell's proteins, so it is safe to discard the \texttt{R} formalism that corresponds to the tRNA from the reaction.
On doing so, the mass balance represented by \ref{eq:conservation-of-mass} can be applied to compute the molecular weight of the protein metabolite.

Finally, the lipid metabolite is the least straightforward because some of the reactants do not have molecular weights specified.
The lipid pseudoreaction is represented in reaction \texttt{r\_2108}:

\texttt{
  lipid backbone + lipid chain --> lipid
}

And both \texttt{lipid backbone} and \texttt{lipid chain} have no molecular weight specified.

Reaction \texttt{r\_4065} specifies a lipid chain pseudoreaction, in which \texttt{lipid chain} is generated:

\texttt{
  0.00808584038168192 C16:0 chain + 0.0237302090972662 C16:1 chain + 0.00226631597615778 C18:0 chain + 0.00870663858950138 C18:1 chain
  \\ --> lipid chain
}

As all reactants have molecular weights defined in the model, the molecular weight of \texttt{lipid chain} can be computed from the mass balance of this reaction.

Reaction \texttt{r\_4063} specifies a lipid backbone pseudoreaction, in which \texttt{lipid backbone} is generated:

\texttt{
  0.00691029997542501 1-phosphatidyl-1D-myo-inositol backbone + 0.026582900248468 ergosterol + 0.0068058000644669 ergosterol ester backbone + 0.00148009998956695 fatty acid backbone + ...
  \\ --> lipid backbone
}

Within this reaction, all reactants have defined molecular weights except for \texttt{fatty acid backbone}.
Four reactions in the model produce \texttt{fatty acid backbone}.
\begin{table}[ht]
  \centering
    \begin{tabular}{llS}
      ID & Reaction & {\makecell{Computed molecular\\ weight (g/mol)}} \\
      \hline
    \texttt{r\_3975} & \makecell{\texttt{palmitate} \\ \texttt{--> 0.25542094 fatty acid backbone + 0.25642888 C16:0 chain}} & 744.56 \\
    \texttt{r\_3976} & \makecell{\texttt{palmitoleate} \\ \texttt{--> 0.25340506 fatty acid backbone + 0.254413 C16:1 chain}} & 714.49 \\
    \texttt{r\_3977} & \makecell{\texttt{stearate} \\ \texttt{--> 0.2834747 fatty acid backbone + 0.28448264 C18:0 chain}} & 742.55 \\
    \texttt{r\_3978} & \makecell{\texttt{oleate} \\ \texttt{--> 0.28145882 fatty acid backbone + 0.28246676 C18:1 chain}} & 716.51 \\
    \end{tabular}
    \caption{ecYeast8 reactions that generate the \texttt{fatty acid backbone} metabolite}
    \label{tab:ecyeast8-fatty-acid-backbone-rxns}
\end{table}

All species in these reactions except for \texttt{fatty acid backbone} have molecular weights specified, and thus I applied mass balance to find the molecular weight of \texttt{fatty acid backbone} from each reaction.
However, the molecular weights computed from each equation are different, as shown in the table above.
Since the differences are slight, and ultimately I am making a back-of-the-envelope calculation, I took the average of the four weights (736.04 g/mol) as the molecular weight of \texttt{fatty acid backbone} for the purposes of computing the molecular weight of \texttt{lipid backbone} (which then yields 24.39 g/mol).

With the molecular weights of \texttt{lipid backbone} and \texttt{lipid chain} defined, I can now compute the molecular weight of \texttt{lipid} using mass balance.

In summary, the molecular weights of the bulk metabolites are:
\begin{table}[ht]
  \centering
  \begin{tabular}{lS}
    Metabolite & {Computed molecular weight (g/mol)} \\
    \hline
    Protein & 464.02 \\
    Carbohydrate & 383.12 \\
    RNA & 64.04 \\
    Lipid & 48.78 \\
    Cofactors & 4.83 \\
    DNA & 3.90 \\
    Ions & 2.48
  \end{tabular}
  \caption{Computed molecular weights of bulk metabolites in ecYeast8}
  \label{tab:ecyeast8-mol-weights}
\end{table}

And thus the molecular weight of biomass is 971.18 g/mol, close to the 966 g/mol in \textcite{takhaveevTemporalSegregationBiosynthetic2023} computed from a different genome-scale model.
As additional validation, the ratio between the molecular weights are similar to the ratio between the mass of each class of macromolecule in the yeast cell dry weight shown by Grigiatis [TODO: Add proper citation].

On simulating the ecYeast8 model using parsimonious flux balance analysis, I set the growth rate to 0.1, the carbon source to glucose (i.e. leaving glucose uptake flux unrestricted), and leaving flux through the biomass reaction unrestricted.
% TODO: format units, use SIunitX or otherwise
Simulating the unmodified enzyme-constrained model gives a flux through the biomass reaction of 0.0889 mmol/(gDW h).
To convert this value to a time it takes to synthesise a new cell, I estimated the molecular weight of biomass as 0.966 g/mol \parencite{takhaveevTemporalSegregationBiosynthetic2023} and the dry mass of the cell as 15 pg [CITATION NEEDED -- LOOK AT THE CELL ECONOMICS PROJECT AND TRACE FROM THERE].
% should this be in the methods?
My logic is:
\begin{enumerate}
   \item Flux through biomass reaction = 0.0889 mmol/(gDW h), i.e. 1 gDW of cell produces 0.0889 mmol biomass in an hour.
   \item The molecular weight of biomass is 0.966 g/mmol.  Therefore, 1 gDW of cell produces (0.0889 * 0.966) g biomass in an hour.
   \item The dry mass of a cell is 15 pg.  Therefore, 1 cell produces (15e-12)(0.0889 * 0.966) g biomass in an hour.
   \item When the cell wants to divide, it produces 1 cells' worth of biomass, i.e. 15 pg.  If the cell produces (15e-12)(0.0889 * 0.966) g biomass in an hour, it takes (15e-12)/(15e-12 * 0.0889 * 0.966) hours.  This evaluates to 11.6 hours.
\end{enumerate}
And this is close to the 10 hours in theory given the growth rate of 0.1 h-1.

For ablated biomass reactions, I followed a similar logic.
Using carbohydrates as an example:
\begin{itemize}
    \item Flux through carbohydrate pseudoreaction is 0.1322 mmol/(gDW * h).  This means that each gDW of cell produces 0.1322 mmol of carbohydrate in an hour.
    \item Cell dry mass is 15e-12 g.  This means that each cell produces (15e-12)*(0.1322) mmol of carbohydrate in an hour.
    \item The molecular weight of the carbohydrate bulk metabolite is 383.12 g/mol, which is 383.12e-3 g/mmol.  This means that each cell produces (383e-3)*(15e-12)*(0.1322) g of carbohydrate in an hour
    \item Cell has (3450+75)e-15 = 3525e-15 g carbohydrates.  This means that each cell produces all the new carbohydrates it needs in (3525e-15) / ((383e-3)*(15e-12)*(0.1322)) hours.  This yields 4.64 hours.
\end{itemize}

Here, I used the flux from the simulated model, the molecular weight as computed earlier, and the mass of the macromolecule per dry cell as described by [CITATION NEEDED: GRIGIATIS].

Repeating this process for the other macromolecules yields:

\begin{table}[ht]
  \centering
  \begin{tabular}{lSS}
    Priority & {\makecell{Flux through\\ ablated objective function (mmol/(gDW h))}} & {Estimated time (h)} \\
    \hline
    Lipids & 0.1082 & 11.3679 \\
    Proteins & 0.0987 & 11.1337 \\
    Carbohydrates & 0.1322 & 4.6397 \\
    DNA & 0.1482 & 8.6527 \\
    RNA & 0.1401 & 12.2571
  \end{tabular}
  \caption{Ablated objective function predicts timescales for biosynthesis prioritising each class of macromolecule.}
  \label{tab:ecyeast8-ablated-timescales}
\end{table}

% TODO: Fix conclusion
Therefore, I conclude that given a finite amount of enzymes, it is more efficient to partition synthesis of macromolecules temporally, and the time scale is at the same order of magnitude as the yeast metabolic cycle.

Upon closer inspection of the times and fluxes... [INSERT RESULTS AND DISCUSSION HERE]
% - How long does it take to replicate the genome?  It is biosynthesis of nucleotides + process of polymerising them.  There has to be super basic cell division cycle literature about this...
% - Fatty acids: cell may use pentose phosphate pathawy and gluconeogenesis to route flow in a cycle to generate masses of NAD(P)H.  Check if the fluxes suggest this.

Discussion: Although it is unrealistic to assume that synthesis of one class macromolecule excludes all others, this approach is still instructive as it gives a back-of-the-envelope calculation to support the notion that the cell partitions biosynthesis temporally, and gives weight to the idea that this may be one of the rationale of the existence yeast metabolic cycle.
Furthermore, the timescale gleaned from this investigation may explain the discrepancy between the shorter (on the range of 1 hour) single-cell metabolic cycles and the longer (on the range of 4 hours) dissolved oxygen oscillations in chemostats along with the longer (on the range of 10 hours) cell division cycles in chemostats.
% give some idea of explanation here
From a biochemical perspective, [DISCUSS IMPLICATIONS FROM LOOKING AT FLUXES]

% AM I STILL DOING THIS??
\section{Modelling chemostat-based studies}
\label{sec:model-chemostat}
% - Metabolic responses to environment
% - Cell communication
% - Relating them to the single-cell yeast metabolic cycle

Cite \textcite{jonesCyberneticModelGrowth1999}.

\section{Coarse-grained, phenomenological model}
\label{sec:model-coarse}

% leads nicely to conclusion
