
There are multiple approaches to evaluate the hypothesis of whether a restriction of the proteome pool favours sequential biosynthesis of biomass components.
Here, I discuss three commonly-used approaches --- parsimonious FBA, regularised FBA, and constraining the sum of absolute values of fluxes --- before justifying the use of directly varying a parameter in the ecYeast8 model to take advantage of GECKO.

Parsimonious FBA \parencite{lewisOmicDataEvolved2010} first uses FBA to compute the optimal growth rate, fixes this value, and then minimises the sum of gene-associated reaction fluxes while maintaining optimal growth.
Depending on the software package, this minimisation either minimises the sum of fluxes (COBRA, for MATLAB), the sum of the absolute values of each flux (\textit{cobrapy}, for the Python programming language) or squared sum (COBREXA, for the Julia programming language).
I reject this approach because it fixes the growth rate, but I aim to see how constraints affect cell strategies, including the growth rate, as the constraints vary along a spectrum.
Additionally, parsimonious FBA relies on reducing the subset of genes that contribute to the solution.
In other words, it modifies the model and it also relies on good gene-protein annotations, with the latter not always guaranteed.

Regularised FBA is defined as adding a regularisation parameter to the objective function.
\textcite{vijayakumarHybridFluxBalance2020} describe a quadratic program for solving a regularised two-level FBA:

\begin{equation}
  \max g^\intercal v - \frac{\sigma}{2}v^\intercal v
  \label{eq:model-regularised-fba}
\end{equation}

where $g$ is the objective function, $v$ is the flux vector, and $\sigma$ is a regularisation parameter than can be tuned.

I reject this approach because it requires a quadratic solver, making it difficult and computationally expensive to implement.
% Add figure to show this unexpected behaviour?
Additionally, the behaviour is not as expected when I implemented it: the growth rate does not change as the regularisation parameter $\sigma$ varies.
I expected a trade-off between growth and reaction fluxes and the balance between these two quantities to change as this parameter varies.

Constraining the sum of absolute values of fluxes is simply defined as fixing

\begin{equation}
  \sum_{i} |v_{i}| < c
  \label{eq:model-constrain-sumfluxes}
\end{equation}

where $v_{i}$ represents each flux of each reaction, and $c$ is a constant to be varied.

This is reasonable for the original Yeast8 model without the enzyme constraint (as opposed to ecYeast8).
% ADD PLOTS TO SHOW THIS?
As $c$ decreases to 0, the original growth rate $\gro$ and ablated growth rates $\griabl$ decrease to 0 because at $c$ values near 0, flux values, including that of the biomass reaction, can only take small values.
However, if this approach is applied to ecYeast8,
there is double imposition of constraints: on
(a) constraints on enzyme usage imposed on the enzyme-usage pseudoreactions created by GECKO, and on
(b) constraints on sum of the absolute values of fluxes.
This will confuse interpretation.

Therefore, I decide to vary the enzyme-available proteome pool to study proteomic constraints.
This takes advantage of a GECKO formalism that is easy to modify and interpret.
