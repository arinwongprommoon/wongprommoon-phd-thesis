
There are multiple approaches to evaluate the hypothesis of whether a restriction of the proteome pool favours sequential biosynthesis of biomass components.
Here, I discuss three commonly-used approaches --- parsimonious FBA, regularised FBA, and constraining the sum of absolute values of fluxes --- before justifying the use of directly varying a parameter in the ecYeast8 model to take advantage of GECKO.

Parsimonious FBA \parencite{lewisOmicDataEvolved2010} first uses FBA to compute the optimal growth rate, fixes this value, and then minimises the sum of gene-associated reaction fluxes while maintaining optimal growth.
Depending on the software package, this minimisation either minimises the sum of fluxes (COBRA, for MATLAB), the sum of the absolute values of each flux (\textit{cobrapy}, for the Python programming language) or squared sum (COBREXA, for the Julia programming language).
I reject this approach because it fixes the growth rate, but I aim to see how constraints affect cell strategies, including the growth rate, as the constraints vary along a spectrum.
Additionally, parsimonious FBA relies on reducing the subset of genes that contribute to the solution.
In other words, it modifies the model and it also relies on good gene-protein annotations, with the latter not always guaranteed.

Regularised FBA is defined as adding a regularisation parameter to the objective function.
\textcite{vijayakumarHybridFluxBalance2020} describe a quadratic program for solving a regularised two-level FBA:

\begin{equation}
  \max g^\intercal v - \frac{\sigma}{2}v^\intercal v
  \label{eq:model-regularised-fba}
\end{equation}

where $g$ is the objective function, $v$ is the flux vector, and $\sigma$ is a regularisation parameter than can be tuned.

I reject this approach because it requires a quadratic solver, making it difficult and computationally expensive to implement.
% Add figure to show this unexpected behaviour?
Additionally, the behaviour is not as expected when I implemented it: the growth rate does not change as the regularisation parameter $\sigma$ varies.
I expected a trade-off between growth and reaction fluxes and the balance between these two quantities to change as this parameter varies.

Constraining the sum of absolute values of fluxes is simply defined as fixing

\begin{equation}
  \sum_{i} |v_{i}| < c
  \label{eq:model-constrain-sumfluxes}
\end{equation}

where $v_{i}$ represents each flux of each reaction, and $c$ is a constant to be varied.

This is reasonable for the original Yeast8 model without the enzyme constraint (as opposed to ecYeast8).
% ADD PLOTS TO SHOW THIS?
As $c$ decreases to 0, the original growth rate $\gro$ and ablated growth rates $\griabl$ decrease to 0 because at $c$ values near 0, flux values, including that of the biomass reaction, can only take small values.
However, if this approach is applied to ecYeast8,
there is double imposition of constraints: on
(a) constraints on enzyme usage imposed on the enzyme-usage pseudoreactions created by GECKO, and on
(b) constraints on sum of the absolute values of fluxes.
This will confuse interpretation.

Therefore, I decide to vary the enzyme-available proteome pool to study proteomic constraints.
This takes advantage of a GECKO formalism that is easy to modify and interpret.




For computational use, the values of $R_{i}$ are equally-spaced discrete values from a vector with an interval $\Delta R_{i}$.
The differential $\pdif{y(\exchrate{glc}, \exchrate{amm})}{R_{i}}$ is thus approximated by central differences:

\begin{equation}
  \begin{aligned}
  \pdif{y(\exchrate{glc}, \exchrate{amm})}{\exchrate{glc}} &\approx \frac{y(\exchrate{glc} + \Delta \exchrate{glc}, \exchrate{amm}) - y(\exchrate{glc} - \Delta \exchrate{glc}, \exchrate{amm})}{2\Delta \exchrate{glc}}\\
  \pdif{y(\exchrate{glc}, \exchrate{amm})}{\exchrate{amm}} &\approx \frac{y(\exchrate{glc}, \exchrate{amm} + \Delta \exchrate{amm}) - y(\exchrate{glc}, \exchrate{amm} - \Delta \exchrate{amm})}{2\Delta \exchrate{amm}}
  \end{aligned}
  \label{eq:model-central-difference}
\end{equation}

Susceptibility gives an advantage over simply computing the differential of the quantity of interest with respect to the exchange rate, i.e. $\pdif{y}{R_{i}}$, because it accounts for the changing magnitude of $R_{i}$.
At greater values of $R_{i}$, the susceptibility is decreased by a greater value with respect to $\pdif{y}{R_{i}}$.
This accounts for how at such high values of $R_{i}$, the proportional change in $R_{i}$ as it is increased or decreased by one step $\Delta R_{i}$ is smaller.
It also accounts for the different step sizes of $\exchrate{glc}$ and $\exchrate{amm}$, owing to the different saturation concentrations of the two nutrients.





To explain why $\ratioabl > 1$ when ammonium exchange is under saturation and glucose is over saturation, consider equation~\ref{eq:model-ratio}.
If we assume that $\frac{f_i}{\griabl}$ for $i$ other than carbohydrate and protein changes negligibly --- justified by the small $f_{i}$ values for these other biomass components --- we write:

\begin{equation}
  \ratioabl \approx (k + \frac{\biomfrac{protein}}{\grabl{protein}} + \frac{\biomfrac{carbohydrate}}{\grabl{carbohydrate}}) \cdot \frac{\gro}{\biomfrac{protein}}
  \label{eq:model-ratio-assume}
\end{equation}

where $k$ is a constant.

In the glucose-limiting region, as $\exchrate{glc}$ increases, under saturation,
\begin{itemize}
  \item $\gro$ increases,
  \item $\grabl{carbohydrate}$ increases, and
  \item $\grabl{protein}$ increases, while
  \item $\ratioabl$ stays constant.
\end{itemize}

In the nitrogen-limiting region, as $\exchrate{amm}$ increases, under saturation,
\begin{itemize}
  \item $\gro$ increases,
  \item $\grabl{carbohydrate}$ stays constant, and
  \item $\grabl{protein}$ increases, while
  \item $\ratioabl$ increases.
\end{itemize}

Based on the glucose-limiting region, the increases in $\grabl{carbohydrate}$ and $\grabl{protein}$ in the carbon-limiting case are just enough to balance the increase in $\gro$.
Then, in the nitrogen-limiting case, if there is no increase in $\grabl{carbohydrate}$, then the combined effect of $\grabl{carbohydrate}$ and $\grabl{protein}$ no longer balance the increase in $\gro$.
This thus explains why $\ratioabl$ increases in this case.

If both nutrients are limiting, as $(\exchrate{glc}, \exchrate{amm})$ varies along the curve from $(0, 0)$ to $(\exchrate{glc, saturation}, \exchrate{amm, saturation})$ that goes through such conditions, both $\grabl{carbohydrate}$ and $\grabl{protein}$ increase, explaining why $\ratioabl$ remain roughly at the same high values greater than 1.
Thus, the region where $\ratioabl > 1$ where both glucose and ammonium are near-limiting can be seen as where the behaviour of the glucose-limiting and ammonium-limiting cases meet.





The model does not account for varying protein fractions and other parameters that affect the size of $\epool$ during growth and division, and across different conditions.
For instance, \textcite{elsemmanWholecellModelingYeast2022} state that growth rate affects proteome fractions ($f$) and the saturation factor ($\sigma$)
However, because I find that $\epool$ affects the growth rate, there may be a circular relationship, and there is no guarantee that parameter values will converge if I tune the $\epool^{\prime}$ to obtain a $\gro$ that in turn affects $\epool^{\prime}$.
Furthermore, the data on biomass component fractions are old, sparse, and coarse --- as a back-of-the-envelope calculation, my investigation is perhaps sufficient.





% FIXME: Make this less like a lecture and link interpretations to hypotheses/etc.
Figure~\ref{fig:model-pool} shows the results of my investigation.
% In full: 0.103697326777848
I denote $\epool$ as the default enzyme-available proteome pool (\SI{0.104}{\gram~\gram_{DW}^{-1}}) and $\epool^{\prime}$ as the enzyme-available proteome pool I set in the model.
When $0 \leq \epool^{\prime} \leq 2\epool$, the model gives realistic growth rates \SIrange{0}{0.8}{\hour^{-1}} (figure~\ref{fig:model-pool-growthrate}).
In this region, growth rate increases linearly as $\epool$ increases.
With higher $\epool^{\prime}$ values --- that is, less of a constraint on the enzyme pool --- the $\ratioabl$ ratio increases, thus indicating that sequential biosynthesis gives less of an advantage (figure~\ref{fig:model-pool-ratio}).
As $\epool^{\prime}$ increases, ablated growth rates $\griabl$ increases linearly at low $\epool^{\prime}$ (figure~\ref{fig:model-pool-ablated}).
But then, at higher $\epool^{\prime}$, these linear relationships decrease in gradient, all going to a plateau at very high $\epool^{\prime}$ (figure~\ref{fig:model-pool-ablated-20}).
This behaviour shows that the relationship between $\griabl$ and $\epool^{\prime}$ is independent of the growth rate $\gro$ and of each other.
% TODO: Come up with a better (biological) interpretation -- consult some org notes.
The $\griabl$ of different components plateau at different $\epool^{\prime}$ sizes, with carbohydrate reaching a plateau first, followed by protein.
This may indicate that the enzyme-available proteome pool is limiting for these components, which may be because the cell needs more enzyme mass to catalyse the reactions needed for the synthesis of these components.




A restricted proteome-available protein pool, as long as the growth rate remains realistic, makes sequential biosynthesis more advantageous.
In addition, the advantage is retained in auxotrophs and deletion strains, confirming the robustness (presence) of sequential biomass synthesis as a resource allocation strategy, and thus may explain why the metabolic cycle is still present in such strains.

Nitrogen source availability affects protein synthesis time in the sequential case, while carbon source availability affects both carbohydrate and protein synthesis times.
Both additively promote (wild type) growth rate.
These effects explain why parallel biosynthesis is advantageous when carbon and nitrogen sources are both near-limiting or when the nitrogen source is near saturation while the glucose source is over saturation.
Such patterns are dependent on the growth saturation curves on each nutrient source.

When the cell prioritises biosynthesis of each biomass component, it allocates its proteome to enzymes differently, but the allocation is similar when the cell prioritises carbohydrates, DNA, RNA, cofactors, and ions.
This could explain why parallel biosynthesis is advantageous in some conditions.
However, the multiplicity of FBA solutions complicates analysis based of fluxes.

How the cell changes its allocation of its proteome to subsystems may explain some experimental observations.
It is possible that the cell synthesises carbohydrate, DNA, and RNA around the same time so that use of  oxidative phosphorylation occur at around the same time.
This may explain cycles in dissolved oxygen concentrations.
In addition, lipid biosynthesis often requires different enzymes than the other components, and this may explain cycling of lipid stores.



The extent to which sequential biosynthesis is advantageous in different nutrient conditions and the synthesis timescales predicted may explain why yeast cells show metabolic cycles in certain nutrient conditions.
The model may provide a weak explanation of why yeast cells continue to have metabolic cycles when abruptly starved of glucose: if the glucose exchange is near zero, the ratio is less than one.
Though, a better investigation would account for switching between nutrient conditions, which the most basic forms of FBA is not built for --- it only looks at steady state, and does not `remember' past states.

In summary, my results show that the yeast cell may synthesise its biomass components in sequence or in parallel, subject to proteomic and nutrient availability constraints.
Although it is unrealistic to assume that synthesis of one class macromolecule excludes all others, this approach is still instructive.
It gives a back-of-the-envelope calculation to support the notion that the cell partitions biosynthesis temporally
It also gives weight to the idea that this may be one of the rationales of the existence of the yeast metabolic cycle.

% Upon closer inspection of the times and fluxes... [INSERT RESULTS AND DISCUSSION HERE]
% - How long does it take to replicate the genome?  It is biosynthesis of nucleotides + process of polymerising them.  There has to be super basic cell division cycle literature about this...
% - Fatty acids: cell may use pentose phosphate pathway and gluconeogenesis to route flow in a cycle to generate masses of NAD(P)H.  Check if the fluxes suggest this.

% <comment> Moved from end of sec:model-fba.  More appropriate as discussion re extensions to my work
However, FBA, in its most basic form, has several limitations.
It only gives a steady-state picture of metabolism, and therefore cannot be used to describe changes in fluxes over time.
Although, dynamic FBA has been developed to solve this problem \parencite{mahadevanDynamicFluxBalance2002}.
Additionally, it does not account for concentrations of metabolites.
Nevertheless, there are many derivatives of FBA to overcome these limitations and extend the method to answer additional modelling questions.
However, these derivatives are outside the scope of this thesis.
% </comment>
