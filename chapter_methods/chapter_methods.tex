% ROUGH DRAFT, based on 10-month report for now
% PROBABLY BEST WRITTEN AFTER BIOLOGICAL RESULTS CHAPTER DRAFT DONE
% TODO:

\chapter{Methods}
\label{ch:methods}

\section{Strains and media}
\label{sec:methods-strains_media}

The \emph{S. cerevisiae} strains used in this thesis are described in table~\ref{tab:methods-strains}.
% Potential additions: CEN.PK & Causton strains, if they go into the biological chapter.

\begin{table}
  \footnotesize
  \centering
  \begin{tabularx}{\linewidth}{bbbbb}
    \toprule
    Name & Background & Genotype & Origin & Notes\\
    \midrule
    FY4 & FY4 & - & EUROSCARF & \textcite{winstonConstructionSetConvenient1995} \\
    htb2::mCherry & FY4 & HTB2::mCherry & In-house, CRISPR & - \\
    BY4741 & BY4741 & \emph{MAT}a \emph{his3$\Delta$1 leu2$\Delta$0 met15$\Delta$0 ura3$\Delta$0} & EUROSCARF & \textcite{brachmannDesignerDeletionStrains1998}\\
    zwf1$\Delta$ & BY4741 & zwf1$\Delta$::KAN & Edinburgh Genome Foundry & Yeast deletion collection \\
    BY4742 & BY4742 & \emph{MAT}a \emph{his3$\Delta$1 leu2$\Delta$0 lys2$\Delta$0 ura3$\Delta$0} & Bruce Morgan & \textcite{calabreseHyperoxidationMitochondrialPeroxiredoxin2019}\\
    tsa1$\Delta$ tsa2$\Delta$ & BY4742 & tsa1$\Delta$::natNT2 tsa2$\Delta$::kanMX4 & Bruce Morgan & \textcite{calabreseHyperoxidationMitochondrialPeroxiredoxin2019} \\
    \bottomrule \\
  \end{tabularx}
  \caption{Strains used in this thesis.}
  \label{tab:methods-strains}
\end{table}

The minimal medium described by \parencite{verduynEffectBenzoicAcid1992} was used unless otherwise stated.
This minimal medium does not contain riboflavin, thus minimising its effect on flavin autofluorescence imaging, and its composition is known and easily-controlled.
Specifically, the composition of the carbon source-limiting medium are described in tables~\ref{tab:methods-media-delft}--\ref{tab:methods-media-delft-vitamins}, and the media pH was adjusted to 6.0 before use using potassium hydroxide, or sodium hydroxide for potassium-free media.
For auxotrophic strains, supplements were added according to table~\ref{tab:methods-media-auxotroph}.
Then, a carbon source is added as appropriate to create the growth medium.

% TODO: align numbers by their decimal point
\begin{table}
  \footnotesize
  \centering
  \begin{tabularx}{\linewidth}{bbb}
    \toprule
    Reagent & Concentration & Remarks\\
    \midrule
    \ce{KH2PO4} & \SI{3.}{\gram~\litre^{-1}} & \\
    \ce{MgSO4.7H2O} & \SI{0.5}{\gram~\litre^{-1}} & \\
    \ce{(NH4)2SO4} & \SI{5.}{\gram~\litre^{-1}} & \\
    \ce{Trace metals} & \SI{1.}{\milli\litre~\litre^{-1}} & See table~\ref{tab:methods-media-delft-metals} \\
    \ce{Vitamins} & \SI{1.}{\milli\litre~\litre^{-1}} & See table~\ref{tab:methods-media-delft-vitamins}.  Add upon use. \\
    \ce{Carbon source} & variable & Add upon use. \\
    \bottomrule \\
  \end{tabularx}
  \caption{
    Composition of base minimal medium.
    For potassium-free media, replace \ce{KH2PO4} with \SI{2.65}{\gram~\litre^{-1}} \ce{NaH2PO4}, which gives the same molarity.
  }
  \label{tab:methods-media-delft}
\end{table}

\begin{table}
  \footnotesize
  \centering
  \begin{tabularx}{\linewidth}{bbb}
    \toprule
    Reagent & Formula & Concentration [\SI{}{\gram~\litre^{-1}}]\\
    \midrule
    EDTA & \ce{C10H14N2Na2O8.2H2O} & 15.00 \\
    Zinc sulfate & \ce{ZnSO4.7H2O} & 4.50 \\
    Manganese (II) chloride & \ce{MnCl2.2H2O} & 0.84 \\
    Cobalt (II) chloride & \ce{CoCl2.6H2O} & 0.30 \\
    Copper (II) sulfate & \ce{CuSO4.5H2O} & 0.30 \\
    Sodium molybdate & \ce{Na2MoO4.2H2O} & 0.40 \\
    Calcium chloride & \ce{CaCl2.2H2O} & 4.50 \\
    Iron (II) sulfate & \ce{FeSO4.7H2O} & 3.00 \\
    Boric acid & \ce{H3BO3} & 1.00 \\
    Potassium iodide & \ce{KI} & 0.10 \\
    \bottomrule \\
  \end{tabularx}
  \caption{
    Composition of trace metal mix for minimal media described in table~\ref{tab:methods-media-delft}.
  }
  \label{tab:methods-media-delft-metals}
\end{table}

\begin{table}
  \footnotesize
  \centering
  \begin{tabularx}{\linewidth}{bbb}
    \toprule
    Reagent & Formula & Concentration [\SI{}{\gram~\litre^{-1}}]\\
    \midrule
    D-(+)-biotin & \ce{C10H16N2O3S} & 0.05 \\
    D-panthothenic acid calcium salt & \ce{Ca(C9H16NO5)2} & 1.00 \\
    Nicotinic acid & \ce{C6H5NO2} & 1.00 \\
    \emph{myo}-Inositol & \ce{C6H12O6} & 25.00 \\
    Thiamine chloride hydrochloride & \ce{C12H15ClN4OS.HCl} & 1.00 \\
    Pyridoxal hydrochloride & \ce{C8H12ClNO3} & 1.00 \\
    4-aminobenzoic acid & \ce{C7H7NO2} & 0.20 \\
    \bottomrule \\
  \end{tabularx}
  \caption{
    Composition of vitamin mix for minimal media described in table~\ref{tab:methods-media-delft}.
  }
  \label{tab:methods-media-delft-vitamins}
\end{table}

\begin{table}
  \footnotesize
  \centering
  \begin{tabularx}{\linewidth}{bbb}
    \toprule
    Reagent & Concentration [\SI{}{\milli\gram~\litre^{-1}}] \\
    \midrule
    histidine & 125. \\
    leucine & 500. \\
    tryptophan & 75. \\
    methionine & 100. \\
    uracil & 150. \\
    \bottomrule \\
  \end{tabularx}
  \caption{
    Supplements to minimal media for BY4741-background auxotrophic strains, compositions derived from \textcite{pronkAuxotrophicYeastStrains2002}.
    For BY4742-background strains, replace methionine with \SI{100}{\milli\gram~\litre^{-1}} lysine-HCl.
  }
  \label{tab:methods-media-auxotroph}
\end{table}

\section{Single-cell microfluidics}
\label{sec:methods-microfluidics}

Cells were grown in a minimal media formulation appropriate for the experiment, supplements appropriate for the strain's auxotrophy, and a carbon source (glucose or pyruvate) appropriate for the experiment (see section~\ref{sec:methods-strains_media}).
% Kevin's OD measurements?
After \SI{14}{hr} overnight growth, the cells were diluted so that the resulting culture had an OD\textsubscript{600} of 0.10--0.20, and then were incubated for \SI{4}{hr} at \SI{30}{\celsius}.

% Refer to Jove paper if it comes out
ALCATRAS microfluidics \parencite{craneMicrofluidicSystemStudying2014}  chambers were filled with media supplemented with 2\% w/v glucose and 0.05\% w/v bovine serum albumin.
Cells were then loaded into the ALCATRAS chambers.
Syringe pumps containing media were programmed to produce a constant flow of \SI{4}{\micro\litre} into the chambers.
The cells and ALCATRAS chambers were located in an incubation chamber (Oko-labs) that was maintained at \SI{30}{\celsius}.

% Update fluorescence settings
Microscopy was performed using a 60 $\times$ 1.4 NA oil immersion objective (Nikon), and the Nikon Perfect Focus System was used to ensure consistent focus.
Five z-slices were taken for brightfield images, with a spacing of \SI{0.6}{\nano\metre} between slices.
Fluorescence imaging was performed with an OptoLED light source (Cain Research), and LED voltage was optimised for maximum signal intensity without LED cut-off prior to experiments.
For flavin imaging, the excitation frequency was \SI{430}{\nano\metre}, the emission filter was set to \SI{455}{\nano\metre}, and exposure time varied: \SI{0}{\milli\second}, \SI{60}{\milli\second}, \SI{120}{\milli\second}, and \SI{180}{\milli\second}.
For mCherry imaging, the excitation frequency was \SI{590}{\nano\metre} and the exposure time was \SI{100}{\milli\second}.
Z-slices were taken for the mCherry images, with the same number of slices and spacing as for the brightfield images.
Image acquisition duration varied for each experiment.
For all channels, images were taken every \SI{5}{min}.

\section{Segmentation, extraction, post-processing}
\label{sec:methods-segmentation}

% TODO: Most of this is obsolete.  Update for aliby.

The brightfield z-stack images were segmented using custom neural network-based software based in MATLAB.
The software identified cell boundaries, birth events, and cell lineages.

Because the aperture of the microscope causes optical aberration, aperture-fitting and flat-field corrections were performed for the flavin and mCherry channels independently, using custom MATLAB scripts.
First, the average brightfield image across all imaging positions at three random time points was generated (figure [FIGURE REMOVED], left).
Second, a circular aperture mask (figure [FIGURE REMOVED], centre) was fitted for each channel based on the saturated average image.
Cells outside this aperture were excluded as they exhibited low signal intensity.
Third, a flat-field correction for cells inside the aperture was generated (figure [FIGURE REMOVED], right) by fitting a Gaussian filter to the average image.

Flavin autofluorescence was quantified by calculating the mean intensity of pixels within the cell boundaries and then subtracting the background fluorescence. % This is `imBackground` - elaborate if needed.
Whi5p localisation was quantified by calculating the \texttt{nucEstConv} measure for each z-slice and then finding the maximum projection across the five z-slices.
The \texttt{nucEstConv} measure was based on convolution with a Gaussian filter of a size specific to the nucleus size \parencite{geronHandsOnMachineLearning2017}.

%%% Local Variables:
%%% mode: latex
%%% TeX-master: "../thesis.tex"
%%% End:
