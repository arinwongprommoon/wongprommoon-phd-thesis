%%
%% precntnt.tex - LaTeX2e thesis class
%%
%% Copyright (C) 2010-2021 Mathew Topper <damm_horse@yahoo.co.uk>
%%
%%
%%   ABOUT
%%
%% This is frontmatter prior to the contents for a Latex2e template which
%% corresponds to the regulations regarding layout of a thesis submitted
%% within the University of Edinburgh. 
%%
%% INPUT THIS FILE USING THE /makefrontmatter{} COMMAND OR THE FORMATTING
%% WON'T WORK PROPERLY

%%%% DEDICATION

\dedication{%
\begin{normalsize}To my family,\end{normalsize}\\[0.2cm]%
Mum and Dad, my brother Arun, and aunt Pinnapa.%
}

%%%% ABSTRACT
\abstract{
    The yeast metabolic cycle (YMC) is a biological rhythm present in budding yeast.
    This biological rhythm is linked to the cell division cycle and entails oscillations in oxygen consumption, intracellular metabolite concentrations, and cellular events.
    Most studies on the YMC have been based on bulk-culture experiments, and the behaviour of the YMC in individual cells is unclear.
    In particular, it is unclear whether such cycles arise from interactions between cells or are generated by individual cells.
    Furthermore, the molecular mechanism of the YMC is unclear.
    I aim to characterise the YMC in single cells and its response to nutrient and genetic perturbations.
    Specifically, I use microfluidics to trap individual yeast cells and record the intensity of flavin fluorescence, a component of the YMC.

    My results show that yeast cells show oscillations in the fluorescence of flavins consistent with the YMC.
    These oscillations coincide with phases of the cell division cycle.
    I found that the metabolic cycle changed its behaviour in respiratory conditions.
    Additionally, I found that cells were able to individually reset the phase of their metabolic cycle, independently of the cell division cycle, in response to abrupt nutrient changes.
    Finally, I showed that deletion strains generated flavin oscillations that exhibited different behaviour from dissolved oxygen oscillations from chemostat conditions, thus warranting a model that reconciles such differences.
}


%%%% LAY SUMMARY
\summary{See the web page
\href{http://www.dcc.ac.uk/resources/how-guides/write-lay-summary}{"How to Write a Lay Summary"}
for a guide.}

%%%% ACKNOWLEDGEMENTS

\acknowledgements{%
  This work would not have been possible without the support of many people and organisations.

  I would like to thank the financial support provided by School of Biological Sciences, University of Edinburgh, and the Edinburgh Global Scholarship.

  I would like to express my deepest gratitude towards my supervisors Dr Diego Oyarz\'{u}n and Prof Peter Swain for the unique opportunity of joining two research teams, and for creating such a great environment for me to learn and work in Edinburgh.
  I am also very grateful your advice, patience, and mental support -- especially in the very isolating days of the COVID-19 pandemic.

  I am most grateful for Dr Kevin Correia for his mentorship in the early parts of my PhD: especially teaching me how to me ask questions, bringing in laboratory best practices from his experience, and even philosophical insights in science and life.

  I am very grateful for the main software developers of the Swain lab: Dr Julian Pietch, Dr Ivan Clark, Dr Diane Adjavon, and Dr Al\'{a}n Mu\~{n}oz Gonzalez.
  I would like to thank Julian in particular for his patience as I grasped the ropes of the old software pipeline.
  Thank you, Diane and Al\'{a}n, for including me in your very ambitious project to re-write the pipeline -- this was an immense learning experience for me.
  Al\'{a}n -- we've been through thick and thin, and I'm most grateful to have you as a partner in crime (?), and for \texttt{doom emacs} too.

  I am very grateful for Dr Ivan Clark and Iseabail Farquhar for imparting their wet-lab expertise and their patience during my experiments.
  Ish, I'd like to thank you for your patience and perseverance training me CRISPR during the chaotic times of the COVID-19 lockdown.

  I would also like to thank my other colleagues at the Swain Lab: Dr Fran\c{c}ois El-Daher, Dr Julien Hurbain, Dr Nahuel Manzanaro Moreno, Dr Yu Huo, and Sof\'{i}a Esteban Serna.

  I would like to thank my colleagues at the Biomolecular Control Group for creating a fantastic social environment and continuously sparking ideas.
  I am grateful for Dr Vanessa Smer-Barreto and Dr Babita Verma for their mentorship and feedback on my research.
  I would like to thank Dr Mona Tonn and Denise Thiel for our weekly movie nights and continued moral support.
  And I would like to thank Evangelos Marios-Nikolados for a great collaboration opportunity, and (too) regular pub nights.
  Let's drink to health!

  I would like to thank my friends from university -- Christos Nicolaou, Esther Ng, Patawee Jintana, among others -- for their moral support.
  I would like to extend special thanks to my best friends Jessica Wang and Maetavee Asavabhokhin for their unwavering emotional and moral support during all ups and downs.
  And I would also like to thank my high school friends, scattered across the globe, Attawit Chaiyaroj, Paloch Vasudhara, and Dr Pravarit Athiprayoon for their support too.

  Last but not least, I would like to thank my family.
  I express my deepest gratitude for my parents who have taught me the wonder of science, the importance of education, and the value of working hard.
  None of this would have been possible without you.
  Aunt Pinnapa, I now have a PhD!
  And lastly, I thank my brother Arun Wongprommoon and cousin Pornsom Ruengpirasiri for their continuous support in this long journey.
}

%%%% DECLARATION

%% Use a custom declaration

% \declaration{I did it.}

%% Use the standard regulation declaration. Enter your
%% name for the signature line.

\standarddeclaration{Arin Wongprommoon}

%%% Local Variables:
%%% mode: latex
%%% TeX-master: "../thesis"
%%% End:
