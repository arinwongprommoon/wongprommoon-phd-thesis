\documentclass{article}

\XeTeXlinebreaklocale "th"

\usepackage{setspace}

\usepackage{lipsum}
\usepackage[thai,english]{babel}
\usepackage[utf8x]{inputenc}
\usepackage{fontspec}
\setmainfont{TH Sarabun New}

\begin{document}

\setstretch{1.30}

วัฏจักรเมตาบอลิซึมในยีสต์ (ภาษาอังกฤษ: ``yeast metabolic cycle'' หรือตัวย่อ YMC) เป็นวัฏจักรทางชีวภาพในยีสต์ทำขนมปัง (\textit{Saccharomyces cerevisiae}) ในวัฏจักรนี้ ความเข้มข้นกับสถานะรีดอกซ์ของเมตาบอไลต์และระดับกรดไรโบนิวคลีอีกภายในเซลล์มีจังหวะขึ้นลง มีการแบ่งช่วงเวลาในการสังเคราะห์สารในเซลล์ และการใช้ออกซิเจนมีจังหวะขึ้นลงที่สังเกตได้จากอุปกรณ์คีโมสแตท (chemostat)

โดยส่วนมาก การศึกษาวัฏจักรเมตาบอลิซึมในยีสต์ใช้คีโมสแตทเป็นเครื่องมือ จึงไม่เป็นที่ชัดเจนว่าวัฏจักรเมตาบอลิซึมในยีสต์เกิดจากปฏิสัมพันธ์ระหว่างเซลล์หรือถูกสร้างอย่างเป็นอิสระในแต่ละเซลล์ วิทยานิพนธ์นี้จึงมุ่งแสดงคุณลักษณะของวัฏจักรเมตาบอลิซึมในยีสต์ในเซลล์เดี่ยวและมุ่งศึกษาผลตอบสนองของวัฏจักรนี้ต่อการเปลี่ยนแปลงของสารอาหารและสารพันธุกรรม โดยจำเพาะ ข้าพเจ้าใช้เครื่องมือไมโครฟลูอิดิกส์ (microfluidics) เพื่อดักและแยกเซลล์ยีสต์ และวัดความเรืองแสงของฟลาวิน (flavin) ตามกาลเวลา เป็นการเฝ้าสังเกตวัฏจักรเมตาบอลิซึมในยีสต์

ไมโครฟลูอิดิกส์เซลล์เดี่ยวสร้างข้อมูลอนุกรมเวลาเป็นจำนวนมาก เนื่องจากอนุกรมเวลาที่สร้างโดยการทดลองเชิงชีววิทยามีลักษณะสั้นและมีสัญญาณรบกวน เครื่องมือทางคอมพิวเตอร์ที่สามารถใช้ในการวิเคราะห์จึงถูกจำกัด ดังนั้นข้าพเจ้าพัฒนาวิธีกรองอนุกรมเวลา พัฒนาโมเดลปัญญาประดิษฐ์เพื่อแยกแยะอนุกรมกรมเวลาที่มีจังหวะซ้ำ และพัฒนาวิธีสหสัมพันธ์เชิงอนุกรมเพื่อประมาณความถี่ของจังหวะในข้อมูลอนุกรมเวลา

ผลการทดลองของข้าพเจ้าแสดงให้เห็นว่าเซลล์ยีสต์แสดงความเรืองแสงของฟลาวินเป็นจังหวะ คู่ขนานกับวัฏจักรการแบ่งตัวของเซลล์ โดยในสภาพกลูโคสสูงประชากรเซลล์สร้างจังหวะนี้อย่างไม่พร้อมกัน ข้าพเจ้าแสดงให้เห็นว่าเซลล์สามารถตั้งเฟสของวัฏจักรเมตาบอลิซึมใหม่เมื่อสารอาหารเปลี่ยนแปลงอย่างกะทันหัน และอย่างเป็นอิสระจากวัฏจักรการแบ่งตัวของเซลล์ นอกจากนี้ข้าพเจ้าแสดงให้เห็นว่าพันธุ์เซลล์ที่เกิดจากการตัดยีนมีจังหวะการเรืองแสงของฟลาวินที่มีพฤติกรรมที่แตกต่างจากจังหวะของความเข้มข้นของออกซิเจนในคีโมสแตท

ข้าพเจ้าใช้ flux balance analysis เพื่อตอบคำถามว่าข้อจำกัดในโปรติโอมนำไปสู่ความได้เปรียบของการแยกเวลาการสังเคราะห์สารในเซลล์หรือไม่ และตอบคำถามว่าการแยกเวลาการสังเคราะห์สารนี้สามารถอธิบายระยะเวลาของวัฏจักรเมตาบอลิซึมได้หรือไม่ ผลการศึกษาของข้าพเจ้าแสดงให้เห้นว่าข้อจำกัดในโปรติโอมนำไปสู่การแยกเวลาการสังเคราะห์สารในเซลล์เนื่องจากการแยกเวลานี้ย่นระยะเวลาการสังเคราะห์สารในเซลล์โดยรวม แต่ทว่าความได้เปรียบนี้ลดต่ำลงหากแหล่งคาร์บอนและไนโตรเจนมีปริมาณต่ำ

กล่าวโดยสรุป วิทยานิพนธ์นี้ยืนยันว่าจังหวะการเรืองแสงของฟลาวินถูกสร้างโดยอิสระในเซลล์ และเสนอว่าวัฏจักรเมตาบอลิซึมในยีสต์ตอบสนองต่อสภาพสารอาหารแล้วกำหนดเวลาการแบ่งตัวของเซลล์ วิทยานิพนธ์นี้เน้นว่ากลุ่มเซลล์ย่อยในการเพาะเลี้ยงเซลล์อาจอธิบายผลการสังเกตวัฏจักรเมตาบอลิซึมในยีสต์ในคีโมสแตท นอกจากนี้วิทยานิพนธ์นี้ปูทางสำหรับการใช้วิธีทางคอมพิวเตอร์ในการวิเคราะห์ชุดข้อมูลอนุกรมเวลาขนาดใหญ่ ซึ่งมีประโยชน์กับสาขาวิชาอื่นนอกจากวัฏจักรเมตาบอลิซึมในยีสต์

\end{document}
