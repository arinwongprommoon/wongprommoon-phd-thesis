% ROUGH DRAFT, based on 10-month report for now
% TODO:
% - Fact-checking
% - Add content

\chapter{Introduction}

\section{Motivation of thesis}

The motivation of this thesis is to understand how an organism adapts its metabolism and cellular processes in response to external conditions.
I do so by through using the the yeast metabolic cycle as a framework for other biological oscillators.
My reasons are twofold: (a) biological oscillators are important for coordination of responses and are present across kingdoms, (b) there are many unknowns about the yeast metabolic cycle, in particular the mechanistic basis and what happens in individual cells.
% might as well mention flavins here if i have to mention this in my description of chapter 1 in the list below.

Therefore, the general aim of my project is to study YMC regulation in isolated cells, especially mechanisms that ensure synchrony with the cell cycle, in different nutrient conditions.

% - describe the logic of the rest of the chapters
% TODO: replace chapter numbers with \ref{}
This thesis is divided into six chapters:
\begin{enumerate}
  \item Chapter 1 discusses the background behind the yeast metabolic cycle and the logic of using flavin autofluorescence as a way to monitor the yeast metabolic cycle.
  \item Chapter 2 discusses the methods used in my project, i.e. single-cell microfluidics of yeast cells, followed by an automated image analysis pipeline.
  \item Chapter 3 discusses the analysis of oscillatory time series; given the size of the datasets and the challenges of analysing noisy low-resolution time series, this deserves discussion in its own right.
  This chapter will step through the process of analysis and provide a review \& justification of the computational methods at each stage.
  \item Chapter 4 presents the biological results of my investigation of single-cell flavin-based yeast metabolic rhythms, employing the analysis methods discussed in chapter 3.
  In brief, I show that the metabolic cycle and cell division cycle are autonomous and synchronise in permissive conditions, while perturbations affect the relationship between these two biological oscillators.
  \item Chapter 5 discusses approaches to model the yeast metabolic cycle mathematically (so that predictions can be made) -- specifically addressing the questions of whether temporal partitioning of biosynthesis explains the timing of the yeast metabolic cycle and whether models of chemostat-based studies can be adapted to our understanding of single-cell metabolic cycles.
  \item Finally, chapter 6 ties together previous understanding of the metabolic cycle, experimental observations, and mathematical models to propose a coarse-grained, phenomenological model of the yeast metabolic cycle.
  And suggests further avenues of study.
\end{enumerate}


\section{Yeast metabolic cycle}
\label{sec:intro-ymc}

\subsection{Introduction to biological rhythms}
\label{subsec:intro-ymc-biological_rhythms}

\subsubsection{Biological basis of biological rhythms}
\label{subsubsec:intro-ymc-biological_rhythms-biological_basis}
% Physiological importance of biological rhythms
% including the circadian rhythm, cell division cycle, yeast metabolic cycle
% Basis, e.g. biochemical oscillators

% Literature:
% Mellor (2016) -- provides a good overview.
%I'm sure there are other citations I have lying around.

Biological rhythms are ... [INSERT SOME DEFINITION HERE].

Biological rhythms compartmentalise cellular processes and ensure that the cell prepares for sequential events.
Genetic oscillators, biochemical oscillators, and metabolic oscillators, all linked to a cellular redox cycle, govern biological rhythms \citep{mellorMolecularBasisMetabolic2016}.
Such biological rhythms include the circadian rhythm and the cell division cycle.

\subsubsection{Theoretical basis of biological rhythms}
\label{subsubsec:intro-ymc-biological_rhythms-theoretical_basis}
% - Mathematics of systems of coupled oscillators.

% Single oscillations
% Useful here: cell division cycle modelling literature, e.g. Novak/Tyson
Biological rhythms are modelled as ....
For example, the cell division cycle is modelled based on the molecular mechanisms (negative feedback loops) -- the most up-to-date and comprehensive one [REALLY?? CHECK CDC LITERATURE.  BUT DON'T GO TOO DEEP BECAUSE MY PHD ISN'T STRICTLY ABOUT CDCS] being \textcite{novakMitoticKinaseOscillation2022}, which employ ordinary differential equations to model negative feedback loops that sustain oscillations of kinases [MAKE SURE THIS SENTENCE IS ACTUALLY FACTUAL].
But people can get away with doing that with the cell division cycle because the molecular basis is well-characterised.
Less well-characterised biological rhythms such as glycolytic oscillations lead to more coarse-grained models, such as one driven by a positive feedback loop \parencite{goldbeterMultisynchronizationOtherPatterns} [EXPLAIN THE MODEL A BIT MORE HERE]. % Also add more examples

% Forced oscillators, coupled oscillators
% Useful here: \textcite{tysonTimekeepingDecisionmakingLiving} and related reviews
As biological rhythms are often coupled with each other, there is interest in modelling forced and coupled oscillators.
If an oscillator is forced, it means that it has a natural oscillation frequency when left to its own devices, but it is in a situation in which an external force is applied at a regular interval so as to force the oscillator to a certain frequency.
An example is the circadian clock, which is forced by being entrained to the light-dark cycle \parencite{goldbeterMultisynchronizationOtherPatterns}.
Closer to home, yeast glycolytic oscillations can also be entrained via a periodic input of substrate.
The concept of forced oscillators is closely linked to coupled oscillators, in which two oscillators are coupled to each other by certain activation/deactivation events.
% This is a good place to put an Arnold tongue figure to illustrate the point I'm trying to make in the next 3 sentences.
What tends to happen is that the two coupled oscillators oscillate at a compromise frequency if the natural frequencies of each are close enough to each other.
Otherwise, what can happen are complex oscillations in which the oscillators lock to a rational ratio of frequencies -- i.e. one oscillator goes through $p$ periods while the other goes through $q$ periods; the exact ratio depends on the ratio of the natural frequencies.
And in certain cases, chaos can happen.
There is a mathematical basis [CITATION NEEDED], and experimental observations support this [CITATIONS NEEDED, AND GO INTO SPECIFIC BIOLOGICAL EXAMPLES].
Certainly, the yeast metabolic cycle has been modelled as a system of coupled oscillators before \citep{papagiannakisAutonomousMetabolicOscillations2017,ozsezenInferenceHighLevelInteraction2019}, based on how it is linked to the cell division cycle -- but I will discuss this later when I finish defining what the metabolic cycle is.
% I can see this becoming relevant again in the modelling chapter

\subsection{Definition and description of the yeast metabolic cycle}
\label{subsec:intro-ymc-definition}
% - Context of discovery: i.e. in chemostats, after glucose starvation, primary read-out being dissolved oxygen, start from Benjamin Tu in 2005.
% - Overview of sequence of events: high oxygen consumption and low oxygen consumption phases, and what happens in each.  I will get into more detail in the following bullet points.
% - Biochemical and metabolic aspects
%   - Metabolites: NADPH, ATP, flavins, etc.
%   - Mobilisation of macromolecules, storage molecules (carbohydates, lipids).
% - Transcription aspects
% - Cell biology aspects, including cell division cycle
% - Introduce chemostat vs single-cell discussions


% Worth re-reading: Mellor 2016, Lloyd 2019

% Definition issue:
% Oscillations individually vs synchronised.
% Laxman et al. (2010) only calls it YMC if synchronised, while Papagiannakis et al. (2017)
% uses the term with single-cell NAD(P)H/ATP cycles.  I use the latter.
% There is this dispute in the scientific literature, which is worth discussing, it's central
% to my project, after all.
The yeast metabolic cycle (YMC) is one such biological rhythm.
% 'aerobic' -- Oct 2020, Kevin did experiments in SC+pyruvate, degassed.
%   (I think the fluorescence killed cells)
It is observed in \emph{Saccharomyces cerevisiae} cells cultured at high density and aerobic,
% 'nutrient-limited' -- Not necessarily.  Cycles are seen in SC+20 g/L glucose.
%   Glucose limitation is somewhere between 10--100 mg/L (Julian's data).
nutrient-limited conditions.
It comprises of oscillations in oxygen consumption, metabolite concentrations, and cellular events -- all coordinated by genome-wide transcriptional oscillations.

Several aspects of the YMC have been observed over decades.
\citet{nosohSYNCHRONIZATIONBUDDINGCYCLE1962} discovered that synchronised \emph{S. cerevisiae} cultures show oscillatory oxygen consumption.
\citet{kasparvonmeyenburgEnergeticsBuddingCycle1969} showed that gas metabolism and energy generation increase upon budding, while and \citet{mochanRespiratoryOscillationsAdapting1973} described a high-amplitude respiratory oscillation following a substrate shift from glucose to ethanol.
\citet{satroutdinovOscillatoryMetabolismSaccharomyces1992} were the first to describe the metabolic components of the short-phase YMC for cells in continuous culture.
\citet{tuLogicYeastMetabolic2005} first incorporated transcript cycling in the description of the YMC and defined the YMC events.

% Are these two sides of the same coin?
% 'Shitty reactor' idea: no short/long cycles in high-quality chemostats
The YMC has been observed in bulk cultures and manifests itself as a
% Murray camp
40-minute short-phase cycle or
% Tu camp
a 4- to 5-hour long-phase cycle synchronised to the cell cycle \citep{mellorMolecularBasisMetabolic2016}.
% [START COMMENTED-OUT MAIN TEXT]
%The short-phase cycle occurs in the polyploid IFO0233 strain cultured under 2\% glucose w/v, while the long-phase cycle occurs in the diploid CEN.PK strain cultured under 1\% glucose w/v \citep{tu_chapter_2010}.
% [END COMMENTED-OUT MAIN TEXT]
% or is it cells independently responding to chemostat conditions?
% My experimental evidence (Apr 2021) strongly suggests this.  And also see Laxman et al. (2010)
These cycles are synchronised across cells in culture.
Furthermore, the YMC is robust, with oscillations persisting for weeks to months \citep{lloydRedoxRhythmicityClocks2007}.
% 'insensitive' -- potentially vague, but can easily clarify
Additionally, the oscillation period is insensitive to temperatures from \SI{25}{\celsius} to \SI{35}{\celsius} and media pH values from
% also see O'Neill et al. (2020)
2.9 to 6.0 \citep{lloydUltradianMetronomeTimekeeper2005}.

% (In more detail... Describe the main features with brief references to key papers and experiments, as evidence.)

The YMC can be divided into three phases: the oxidative phase, the reductive-building phase, and the
% Debate whether this phase actually exists or is an artefact of the chemostat
% conditions in experiments thus far.  RC may be an adaptation of 'unhappy' cells
% to chemostat.  Think of feast-and-famine or glucose limitation.
% There are many strings to pull on this phase.
% Though Lloyd/Murray camp also describe these 3 phases
% However, all publications agree with OX/HOC and RED/LOC as major phases
reductive-charging phase.
The oxidative and reductive phases are characterised by the rate of oxygen consumption \citep{mellorMolecularBasisMetabolic2016}.
Using unsupervised clustering, \citet{tuLogicYeastMetabolic2005} described the following events in each phase:
% [IT LOOKS A BIT OUT OF PLACE...] \citet{krishnaMinimalPushPull2018} interpret the oxidative phase as a growth state, while the reductive phase is a quiescent state.

\begin{enumerate}
  \item \emph{Oxidative phase:} Cells consume oxygen at a high rate as respiration, fermentation, and
        % Potentially may be the reason auxotrophs (hypothetically) don't have YMCs --
        % but I showed that they do
        energy-demanding processes
        like biosynthesis and gene expression occur.
    As the oxidative phase transitions to the reductive-building phase, ethanol and acetate concentrations in the medium peak as respiration finishes \citep{tuLogicYeastMetabolic2005}.
    % and so does flavins? (FMN and FAD)
    % makes sense if they are oxidised when respiration occurs
    `Redox state' metabolites, including NADH, NADPH, and glutathione, become most oxidised in this phase \citep{lloydUltradianMetronomeTimekeeper2005}.
Here, 70\% of metabolite concentrations peak with the combined autofluorescence of NADH and NADPH \citep{murrayRegulationYeastOscillatory2007}.
\item \emph{Reductive-building phase:} Cells consume oxygen at a low rate, and activities linked to mitochondrial growth occur.
% Idea put forth: temporal separation prevents ROS damage to DNA.  But disputed,
% as shown here.
  There is evidence to suggest that activities linked to cell proliferation -- such as initiation of the cell division cycle, DNA replication, and spindle pole activity -- are gated to the reductive-building phase for both the short-period and long-period YMC.
  Such evidence includes budding activity and the pattern of the expression of \emph{YOX1}, which encodes a cell division cycle repressor \citep{tuLogicYeastMetabolic2005}.
% But Heinemann lab publications suggest that YMC gates early & late cell cycle independently.
% Not sure if they changed the growth rate of the cells though -- probably quite difficult
% to do in microfluidics setting.
  However, measuring DNA content and oxygen consumption in cells grown at different growth rates \citep{slavovCouplingGrowthRate2011} showed that the S phase of the cell cycle may occur in the oxidative phase if the cells have a slow growth rate.
% What would be the function of such flexible gating?  Seems like community
% has no meaningful consensus yet.  Probably allocation of metabolites.
  This evidence indicates that the gating between the YMC and the cell cycle is flexible.
\item \emph{Reductive-charging phase:} Cells consume oxygen at a low rate.
  Non-respiratory metabolism and degradation processes occur to prepare the cell for the oxidative phase.
  This non-respiratory metabolism includes glycolysis, ethanol and fatty acid metabolism, and nitrogen metabolism.
  With these metabolic modes, under the regulation of the transcription factors Msn2p and Msn4p \citep{kuangMsn2RegulateExpression2017}, acetyl CoA accumulates for ATP production in the oxidative phase \citep{tuLogicYeastMetabolic2005}.
After acetyl CoA levels reach a threshold, it promotes histone acetylation and thus induces the oxidative phase.
These metabolic pathways also optimise production of NADPH -- based on the induction of \emph{GND2} -- to buffer against oxidative stress in the oxidative phase.
Genes associated with protein degradation, ubiquitinylation, peroxisomes, vacuoles, and the proteosome also peak in the reductive-charging phase.
\end{enumerate}

% (Describe features that may differ in different strain and nutrient backgrounds, i.e. 'what can change the YMC?'.)
% This REALLY needs more detail: review literature related to the R/C phase ----- I don't think there's a lot though.

\subsection{Metabolic cycles in other organisms}
\label{subsec:intro-ymc-other_organisms}
% - Other yeasts
% - Other organisms: bacteria, mammalian cells

\subsection{Yeast metabolic cycles under perturbations}
\label{subsec:intro-ymc-perturbations}
% - Nutrient perturbations
  % - Changing concentration or compositions of carbon sources.
  % - Changing concentration or compositions of nitrogen sources.
  % - Key deletion strains shed light on mechanism

The phase difference between the YMC and cell cycle varies in different conditions % how?
% Confirmation that the YMC responds to nutrient availability?  How does phase
% resetting help the Cell Division Cycle then?
\citep{ewaldYeastCyclinDependentKinase2016}. % ok, but have I seen this phase shift with the different media (i.e. SC vs SM)?  What did Ewald have to say about this phase shift??
% 'bulk carbon source' -- could be adaptation to changing nutrient concentrations in general?
Adding a bulk carbon source such as acetate, ethanol, or acetaldehyde can reset the phase of the YMC \citep{kuangMsn2RegulateExpression2017, krishnaMinimalPushPull2018}.
The long-phase cycle may vary from 1.4 to 14 hours depending on the strain and culture conditions including the chemostat dilution rate \citep{caustonMetabolicRhythmsFramework2018}. % how?
Specifically, increasing the growth rate increases the duration of the oxidative phase relative to the reductive phases \citep{slavovCouplingGrowthRate2011}.
It has also been found that media quality affects the period of the YMC.
% 20 g/L --> 0.5 g/L doesn't seem to significantly prolong it, but maybe
% a greater effect will be seen if we get to the region of glucose limitation
Specifically, decreasing glucose concentration prolongs the YMC \citep{mellorMolecularBasisMetabolic2016,papagiannakisAutonomousMetabolicOscillations2017},
% Maybe because long OX stage as it's the stage for biomass building
and decreasing nitrogen source concentration does so too \citep{baumgartnerFlavinbasedMetabolicCycles2018}.
Furthermore, the YMC can oscillate multiple times per cell cycle, or even disappear in some conditions \citep{baumgartnerFlavinbasedMetabolicCycles2018}. % which conditions?  can these be tested?
% --> also see Causton et al. (2015)

\subsection{Big picture/Hypothesis: a nutrient sensor than entrains the cell division cycle?}
\label{subsec:intro-ymc-hypothesis}
% - Idea that the yeast metabolic cycle is a autonomous biological oscillator that operates at a range of frequencies in response to a range of (permissive) growth conditions.  This oscillator acts as a timing mechanism for cellular processes, most importantly the cell division cycle.  The yeast metabolic cycle creates windows of opportunities for the cell to commit to START if conditions are favourable: e.g. good stores.
In sum, the YMC is a robust biological rhythm in \emph{S. cerevisiae} that responds to nutrient availability.
% **Logical inconsistency**: previously I mentioned that this coupling is flexible &
% this idea is disputed.  This is an error!
In turn, if the cell divides, it temporally partitions cell cycle events in relation to YMC events to prevent oxidative damage.
In bulk culture, the YMC synchronises between cells for an unclear reason.

\subsection{Implications of the metabolic cycle}
\label{subsec:intro-ymc-implications}
%- Fundamental mechanistic basis for biological oscillators, such as the cell division cycle.  Could change what we know about the cell division cycle.
% (Significance of study)

Metabolic oscillations may be the origins of biological timekeeping mechanisms.
Indeed, \citet{lloydRedoxRhythmicityClocks2007} assert that ultradian oscillations form the basis of longer-period biological oscillators like the circadian rhythm or the cell cycle.
Circadian rhythms can occur in cells of multicellular eukaryotes without transcription \citep{oneillCircadianRhythmsPersist2011}, refuting the idea that gene circuits are responsible for these rhythms.
Additionally, the eukaryotic cell cycle evolved before cyclin-dependent kinases \citep{papagiannakisAutonomousMetabolicOscillations2017}, so metabolic oscillations may have served to regulate the cell cycle before cyclin-dependent kinases evolved.
Furthermore, YMCs share mechanisms with the circadian oscillator \citep{caustonMetabolicCyclesYeast2015,arataQuantitativeStudiesCellDivision2019}, suggesting a common evolutionary origin.
Thus, studying YMCs may shed light on the evolution of biological rhythms.

\subsection{Disputes and unresolved questions}
\label{subsec:intro-ymc-unresolved}
% - Limitations of chemostat: obscures contributions from sub-populations of cells or individual cells, is not realistic compared to natural habit of yeast, starvation.
% - Do cells individually generate the metabolic cycle or is a diffusible chemical responsible for synchrony?
% - Do single-cell flavin-based metabolic cycles from deletion strains recapitulate dissolved oxygen-based metabolic cycles in chemostats?  If not, what would be a likely explanation?

% (State the main unknowns of the yeast metabolic cycle:
% - molecular mechanisms, specifically...? -> deletions to investigate
% - whether things are the same in batch vs bulk)

There are unknowns in the molecular mechanism that drives YMCs.
Genome-wide transcript cycling has two superclusters that correspond to the oxidative and reductive-building phases \citep{machneYinYangYeast2012}.
However, there has been no genome-wide analysis of genes that influence cycling \citep{mellorMolecularBasisMetabolic2016}, though some genes seem to have key roles.
For example, \emph{MSN2} and \emph{MSN4} have been shown to regulate acetyl CoA accumulation in the reductive-charging phase, as evidenced by the lack of YMCs in deletion strains \citep{kuangMsn2RegulateExpression2017}.
Moreover, deletion of \emph{GTS1} shortens the oscillation periods \citep{lloydUltradianMetronomeTimekeeper2005},
% There are more: GSY2, GPH1 (O'Neill et al., 2020)
and deletion of \emph{ZWF1} removes the oscillations \citep{tuCyclicChangesMetabolic2007}.

However, there are a number of questions remaining.
For example, without proteome analysis, it is unclear how protein levels and post-translation modifications are affected.
How YMCs are coordinated between cells is also unclear, though experimental studies \citep{murrayRegulationYeastOscillatory2007} and mathematical models
% Actually, they aren't *that* explicit on H2S mediating cell-cell communication
% Disputed: may be an artefact of chemostat activation.
% No synchronisation in single-cell conditions, so here's where simulating chemostat in microfluidics
% comes in (feast-and-famine).
% The different aspects found from bulk culture & single-cell experiments isn't written so
% clearly.  A proper review will be better.  These are just some aspects pasted together.
\citep{krishnaMinimalPushPull2018} suggest that acetaldehyde and hydrogen sulfide may mediate cell-to-cell communication.
The reason for this synchronisation between cells is still unclear. % citation needed

Current knowledge of the YMC derives mostly from studies done on batch or continuous yeast cultures, but it is unclear whether aspects of the YMC discovered in bulk culture apply to single cells.
% This does seen like a key difference seen in single-cell, but the fact wasn't made clear
% in writing here.
\citet{papagiannakisAutonomousMetabolicOscillations2017} revealed that YMCs are an intrinsic feature of single cells and are autonomous with respect to the cell cycle, based on measurements of the combined level of NADH and NADPH in single cells in microfluidic devices.
% Also, Murray et al. (2011) -- flavins in BULK culture.
Furthermore, by measuring the level of flavins in the cell, \citet{baumgartnerFlavinbasedMetabolicCycles2018} demonstrated that YMCs persist in mutants deficient in oxidative phosphorylation, and that the cell cycle inhibitor rapamycin desynchronises the YMC and the cell cycle.
Additionally, \citet{ozsezenInferenceHighLevelInteraction2019} suggested that the YMC controls the early and late cell cycle independently, based on modelling the oscillators as a system of Kuramoto oscillators.
However, the effects of perturbations to specific metabolic networks in single-cell conditions are yet to be determined.

% And there is Laxman et al. (2010), which is an early attempt at microfluidics
% to address the bulk vs single-cell issue.  However, it lacks quantitative
% time-series analysis of any sort.

% [BEGIN COMMENTED MAIN TEXT]
%Systems biology approaches have been used to develop models to explain features of biological rhythms.
%\citet{ozsezenInferenceHighLevelInteraction2019} use a deterministic Kuramoto model to explain the interaction between one metabolic oscillator and three cell cycle oscillators.
%Additionally, a data-driven stochastic modelling approach was used to describe the relationship between the circadian and cell cycle oscillators without assuming a prior relationship \citep{droin_low-dimensional_2019}.
%This strategy is yet to be applied to the metabolic and cell cycle oscillators.
%However, \citet{ozsezenInferenceHighLevelInteraction2019} assumed the same, specific coupling function for all interactions between the four oscillators.
% [END COMMENTED MAIN TEXT]

\section{Flavins and flavoproteins}
\label{sec:intro-flavin}
% USE ORG NOTE ABOUT FLAVINS/FLAVOPROTEINS

\subsection{Introduction to cellular autofluorescence}
\label{subsec:intro-flavin-autofluo}
% - (Refer to reviews, there are good ones in the 2000s)

\subsection{Biochemical basis of flavins and flavoproteins}
\label{subsec:intro-flavin-biochem}
% - Flavins are molecules with a certain aromatic moiety than can undergo redox reactions.  Show chemical structures.  These include FAD and FMN.
% - Flavoproteins have FAD and FMN as co-factors.

\subsubsection{Description of key flavoproteins and their roles}
\label{subsubsec:intro-flavin-biochem-descriptions}
% - (Sort by abundance)
% - Most in biosynthesis and redox.

\subsection{Studies on global flavin changes in response to perturbations}
\label{subsec:intro-flavin-perturbations}
% - How oxygen and nutrient changes affect flavins and other autofluorescence (there are papers that specifically study this)

\subsection{Flavins and flavoproteins in the yeast metabolic cycle}
\label{subsec:intro-flavin-ymc}
% - Autofluorescence is usually something people would avoid in experiments, but we are taking advantage of it.
% - Easy-to-measure metabolic readout (does not require additional engineering of strains), linked to many biochemical processes

\subsubsection{Review of chemostat and single-cell studies that use flavin}
\label{subsubsec:intro-flavin-ymc-precedent}
% - To justify my use, e.g. \textcite{murrayRedoxRegulationRespiring2011}, \textcite{baumgartnerFlavinbasedMetabolicCycles2018}

\subsubsection{Do flavin cycles suggest cycling of lipid stores?}
\label{subsubsec:intro-flavin-ymc-lipid_cycling}
% e.g. \textcite{campbellBuildingBlocksAre2020}

%%% Local Variables:
%%% mode: latex
%%% TeX-master: "../thesis.tex"
%%% End:
