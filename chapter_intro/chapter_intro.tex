% ROUGH DRAFT, based on 10-month report for now
% TODO:
% - Fact-checking
% - Add content (some from my org notes, and I refer to them in comments)

\chapter{Introduction}

\section{Motivation of thesis}

The motivation of this thesis is to understand how an organism adapts its metabolism and cellular processes in response to external conditions.
I do so by through using the the yeast metabolic cycle as a framework for other biological oscillators.
My reasons are twofold: (a) biological oscillators are important for coordination of responses and are present across kingdoms, (b) there are many unknowns about the yeast metabolic cycle, in particular the mechanistic basis and what happens in individual cells.
% might as well mention flavins here if i have to mention this in my description of chapter 1 in the list below.

Therefore, the general aim of my project is to study YMC regulation in isolated cells, especially mechanisms that ensure synchrony with the cell cycle, in different nutrient conditions.

% - describe the logic of the rest of the chapters
% TODO: replace chapter numbers with \ref{}
This thesis is divided into six chapters:
\begin{enumerate}
  \item Chapter 1 discusses the background behind the yeast metabolic cycle and the logic of using flavin autofluorescence as a way to monitor the yeast metabolic cycle.
  \item Chapter 2 discusses the methods used in my project, i.e. single-cell microfluidics of yeast cells, followed by an automated image analysis pipeline.
  \item Chapter 3 discusses the analysis of oscillatory time series; given the size of the datasets and the challenges of analysing noisy low-resolution time series, this deserves discussion in its own right.
  This chapter will step through the process of analysis and provide a review \& justification of the computational methods at each stage.
  \item Chapter 4 presents the biological results of my investigation of single-cell flavin-based yeast metabolic rhythms, employing the analysis methods discussed in chapter 3.
  In brief, I show that the metabolic cycle and cell division cycle are autonomous and synchronise in permissive conditions, while perturbations affect the relationship between these two biological oscillators.
  \item Chapter 5 discusses approaches to model the yeast metabolic cycle mathematically (so that predictions can be made) -- specifically addressing the questions of whether temporal partitioning of biosynthesis explains the timing of the yeast metabolic cycle and whether models of chemostat-based studies can be adapted to our understanding of single-cell metabolic cycles.
  \item Finally, chapter 6 ties together previous understanding of the metabolic cycle, experimental observations, and mathematical models to propose a coarse-grained, phenomenological model of the yeast metabolic cycle.
  And suggests further avenues of study.
\end{enumerate}


\section{Yeast metabolic cycle}
\label{sec:intro-ymc}

\subsection{Introduction to biological rhythms}
\label{subsec:intro-ymc-biological_rhythms}

\subsubsection{Biological basis of biological rhythms}
\label{subsubsec:intro-ymc-biological_rhythms-biological_basis}
% Physiological importance of biological rhythms
% including the circadian rhythm, cell division cycle, yeast metabolic cycle
% Basis, e.g. biochemical oscillators

% Literature:
% Mellor (2016) -- provides a good overview.
% and that collection of review articles probably give good definitions

[FIGURE: SHOW COMPONENTS OF A BIOLOGICAL OSCILLATOR, ADAPTED FROM MELLOR 2016]

Biological rhythms can be defined as physiological or cellular processes that repeat after a certain period of time.
Genetic oscillators, biochemical oscillators, and metabolic oscillators, all linked to a cellular redox cycle, govern biological rhythms \citep{mellorMolecularBasisMetabolic2016}.
Biological rhythms can occur at different time scales, from short (seconds, e.g. glycolytic cycle), to ultradian cycle (i.e. cycles that are more frequent than 24 hours), to circadian rhythms (i.e. 24 hours).
Biological rhythms are important in biological timekeeping -- in other words, separating physiological processes, including cellular processes, to different times to coincide with other processes so that things work the way they should.
This can be instrumental in responding to external conditions, e.g. nutrient conditions, growth requirements, or the day-night cycle.
Thus, this means that biological rhythms can vary according to conditions -- they aren't necessarily rigid clocks, but some are more rigid than others (e.g. the circadian rhythm).

[FIGURE(S) MAY BE USEFUL IN DEMONSTRATING THE CELL DIVISION CYCLE HERE.  SHOW G1/S/G2/M PHASES, SHOW OVERVIEW OF CDKs, ETC.]

% VERY useful citation: adlerYeastCellCycle2022

Such biological rhythms include the circadian rhythm and the cell division cycle.
To demonstrate the definition of biological rhythm, I will discuss here the cell division cycle, which is a very well-characterised biological rhythm.
The cell division cycle in budding yeast is governed by a series of gene regulatory networks that interact in a feedback loop, resulting in oscillatory expression of regulatory proteins, namely, cyclin-CDK complexes that regulate cellular events in a temporal manner \citep{adlerYeastCellCycle2022, orlandoGlobalControlCellcycle2008, murrayRecyclingCellCycle2004}.
This cycle also includes biochemical and metabolic oscillators, for example, biosynthesis during S phase.
As with other biological rhythms, the cell division cycle also includes a system to control it, i.e. so that DNA replication occurs once every cell division cycle and so that the cell only divides when necessary.
The importance of such control systems are highlighted by disorders when these systems go wrong, e.g. chromosome aberrations and cancer.% [NEED MORE DETAIL, POTENTIALLY CITATIONS].

The yeast metabolic cycle is a type of biological rhythm because it has all the above properties that define a biological rhythm.
Namely, it has gene-expression oscillators as evidenced by transcript cycling in its phases,
it has biochemical oscillators as evidenced by changes in dissolved oxygen in the chemostat,
and it has metabolite oscillations as evidence by changes in the levels of compounds that undergo redox reactions like NADH/NADPH.
I discuss further this evidence in the context of the known progression of the yeast metabolic cycle in section \ref{subsubsec:intro-ymc-definition-phases}.
However, in contrast to the cell division cycle, the control mechanisms of the yeast metabolic cycle are less well characterised -- I discuss this further in section \ref{subsec:intro-ymc-unresolved}.

\subsubsection{Theoretical basis of biological rhythms}
\label{subsubsec:intro-ymc-biological_rhythms-theoretical_basis}
% - Mathematics of systems of coupled oscillators.

% Single oscillations
% Useful here: cell division cycle modelling literature, e.g. Novak/Tyson,
% Adler et al. (2022) -- comprehensive review of cell division cycle models
% Goldbeter (2022) -- models of several examples
There are many approaches in mathematically modelling biological oscillations.
The theoretical basis of biological rhythms originated in work in the 1960s, which included simple systems of ordinary differential equations to describe negative feedback control circuits \citep{goodwinOscillatoryBehaviorEnzymatic1965, griffithMathematicsCellularControl1968}.
Experimental observations have then informed the development of models with finer detail.
And furthermore, synthetic genetic circuits have also been modelled and developed \citep{elowitzSyntheticOscillatoryNetwork2000}.

As an example to illustrate, I'm going to talk about modelling the cell division cycle.
The cell division cycle has inspired a long history of models with a variety of approaches.
Early models are based on a negative feedback loop of key components as identified by experimental studies -- for example, \textcite{goldbeterMinimalCascadeModel1991} assumed a minimal model of one cyclin, one kinase, and one protease to construct a negative feedback loop with a delay, giving rise to stable oscillations.
Such a strategy forms the basis of later models that incorporate more detail, including additional control points of the cell division cycle \parencite{chenIntegrativeAnalysisCell2004}, responses to perturbations such as osmotic stress \parencite{adroverTimeDependentQuantitativeMulticomponent2011}, and relationship with other oscillators like the circadian rhythm \parencite{gerardEntrainmentMammalianCell2012, charvinForcedPeriodicExpression2009, droinLowdimensionalDynamicsTwo2019}. \parencite{adlerYeastCellCycle2022}.
More recent, comprehensive models include \textcite{adlerYeastCellCycle2022} which is based on a system of ordinary differential equations adapted for the modelling to pheromone and osmotic shock responses, and \textcite{novakMitoticKinaseOscillation2022}, which models the cell division cycle as a series of switches between two stable steady states whose behaviour is regulated by the CDK oscillator.

But people can get away with doing that with the cell division cycle because the molecular basis is well-characterised.
The glycolytic oscillation is a type of ultradian biochemical oscillator, characterised by oscillations in NADH levels (and other cofactors) in budding yeast cells at the time scale of 10-ish seconds \parencite{doddLiveCellImaging2017, lloydSaccharomycesCerevisiaeOscillatory2019, olsenOscillationsYeastGlycolysis2021}, when they are in high-glucose conditions.
Because the glycolytic oscillation is less well-characterised, models that describe it are more coarse-grained, such as one driven by positive feedback loops as a system of two coupled instability-generating mechanisms \parencite{goldbeterMultisynchronizationOtherPatterns}. % Also add more examples

% Forced oscillators, coupled oscillators
% Useful here: \textcite{tysonTimekeepingDecisionmakingLiving} and related reviews
As biological rhythms are often coupled with each other, there is interest in modelling forced and coupled oscillators.
If an oscillator is forced, it means that it has a natural oscillation frequency when left to its own devices, but it is in a situation in which an external force is applied at a regular interval so as to force the oscillator to a certain frequency.
An example is the circadian clock, which is forced by being entrained to the light-dark cycle \parencite{goldbeterMultisynchronizationOtherPatterns}.
Closer to home, yeast glycolytic oscillations can also be entrained via a periodic input of substrate.
The concept of forced oscillators is closely linked to coupled oscillators, in which two oscillators are coupled to each other by certain activation/deactivation events.
% This is a good place to put an Arnold tongue figure to illustrate the point I'm trying to make in the next 3 sentences.
What tends to happen is that the two coupled oscillators oscillate at a compromise frequency if the natural frequencies of each are close enough to each other.
Otherwise, what can happen are complex oscillations in which the oscillators lock to a rational ratio of frequencies -- i.e. one oscillator goes through $p$ periods while the other goes through $q$ periods; the exact ratio depends on the ratio of the natural frequencies.
And in certain cases, chaos can happen.
There is a mathematical basis in Arnold tongues \citep{heltbergTaleTwoRhythms2021}, and experimental observations support this.
For example, \citep{charvinForcedPeriodicExpression2009} showed that externally forcing cell division cycles via glucose pulsing leads to phase-locking of the cell division cycle oscillator only within a range of extrinsic periods.

Certainly, the yeast metabolic cycle has been modelled as a system of coupled oscillators before \citep{papagiannakisAutonomousMetabolicOscillations2017,ozsezenInferenceHighLevelInteraction2019}, based on how it is linked to the cell division cycle -- but I will discuss this in section \ref{subsec:intro-ymc-model} when I finish defining what the metabolic cycle is.
% I can see this becoming relevant again in the modelling chapter

\subsection{Definition and description of the yeast metabolic cycle}
\label{subsec:intro-ymc-definition}
% Worth re-reading: Mellor 2016, Lloyd 2019

Heads-up: people don't really agree on what the yeast metabolic cycle (YMC) is, so I'm going to go with the bits that everyone seems to agree on first, and then discuss what people disagree on, and why.
I will discuss these disputes in the literature in section \ref{subsec:intro-ymc-unresolved}.

The yeast metabolic cycle (YMC) is one such biological rhythm.

% GOOD MAIN IDEA-THEN-EVIDENCE STRUCTURE HERE
It is an ultradian oscillator.
It comprises of oscillations in oxygen consumption, metabolite concentrations, and cellular events -- all coordinated by genome-wide transcriptional oscillations.
Importantly, the cellular events the metabolic cycle has consistently been linked to is the cell division cycle.
%   Glucose limitation is somewhere between 10--100 mg/L (Julian's data).
nutrient-limited conditions;
although, single-cell studies have then suggested that these oscillations occur in nutrient-rich conditions too (see further: section \ref{subsec:intro-ymc-unresolved}).

% Not too sure if the subsubsections in this subsection are a good idea.
\subsubsection{History of evidence for the yeast metabolic cycle}
\label{subsubsec:intro-ymc-definition-history}

Several aspects of the YMC have been observed over decades.
\citet{nosohSYNCHRONIZATIONBUDDINGCYCLE1962} discovered that synchronised \emph{S. cerevisiae} cultures show oscillatory oxygen consumption.
\citet{kasparvonmeyenburgEnergeticsBuddingCycle1969} showed that gas metabolism and energy generation increase upon budding, while and \citet{mochanRespiratoryOscillationsAdapting1973} described a high-amplitude respiratory oscillation following a substrate shift from glucose to ethanol.
\citet{satroutdinovOscillatoryMetabolismSaccharomyces1992} were the first to describe the metabolic components of the short-phase YMC for cells in continuous culture.
\citet{tuLogicYeastMetabolic2005} based on a chemostat-based investigation of growth of budding yeast on glucose-starved conditions,
first incorporated transcript cycling in the description of the YMC and defined the YMC events.

It is important to contrast the yeast metabolic cycle to similar biological oscillator: the glycolytic oscillation.
The glycolytic oscillation has a periodicity on the scale of 40 seconds \citep{olsenRegulationGlycolyticOscillations2009}.
In contrast, the yeast metabolic cycle has been described, using various definitions, to either exhibit a 40-minute short-phase cycle \citep{lloydUltradianMetronomeTimekeeper2005, liRapidGenomescaleResponse2006, lloydRedoxRhythmicityClocks2007}, or a long-phase cycle, which is most commonly described to be 4-5 hours \citep{tuLogicYeastMetabolic2005, tuCyclicChangesMetabolic2007}, but also ranges between 1.4 hours to 14 hours, depending on the chemostat dilution rate \citep{beuseEffectDilutionRate1998, oneillEukaryoticCellBiology2020}.

Glycolytic oscillations have been observed as highly damped, but yeast metabolic oscillations have been observed to be robust -- experiments have been conducted that maintain these oscillations for weeks \parencite{lloydRedoxRhythmicityClocks2007}.
Additionally, glycolytic oscillations have been observed in anaerobic conditions \citep{lloydSaccharomycesCerevisiaeOscillatory2019}, but yeast metabolic cycles have been observed in aerobic conditions.
Moreover, glycolytic oscillations are characterised by fluctuations in NADH fluorescence, but yeast metabolic cycles consist of longer-period fluctuations in NADH fluorescence as well as fluctuations in other compounds like ATP and flavins.

\subsubsection{Phases of the yeast metabolic cycle}
\label{subsubsec:intro-ymc-definition-phases}
% INTEGRATE THIS STRUCTURE IN EACH PARAGRAPH THAT DISCUSSES EACH PHASE
% 1. Overarching theme of phase
% 2. Cellular events (cell division cycle, mitochondria)
% 3. Metabolic events (metabolite concentrations, redox)
% 4. Transcript/genetic events.
% Or any that make sense, e.g. 3 main events and all the evidence from different
% parts of biochemistry.
% Structure should then link well with the definition of biological rhythms in
% a previous subsection.
% ---------------------------------------------------------------------
The YMC can be divided into two major phases: an oxidative, high-oxygen consumption (OX/HOC) phase and a reductive, low-oxygen consumption (RED/LOC) phase.
% Though Lloyd/Murray camp also describe these 3 phases
Many authors \citep{slavovMetabolicCyclingCell2011, murrayRedoxRegulationRespiring2011, caustonMetabolicRhythmsFramework2018} use oxygen consumption rates, evidenced by the change of dissolved oxygen concentrations over time, as a basis to refer to the YMC as a two-phase cycle.
Though, there are authors \citep{machneYinYangYeast2012} that base their two-phase model on the clustering of gene expression patterns.
\citet{krishnaMinimalPushPull2018} interpret the oxidative phase as a growth state, while the reductive phase is a quiescent state.
In contrast to the two-phase model, some authors identify a three-phase model with a reductive-building (RB) phase and a reductive-charging (RC) phase within the reductive phase, especially within the long-phase (4-5 hours) yeast metabolic cycle.
This three-phase model is primarily based on cellular events, including clustering of transcript trajectories \citep{tuLogicYeastMetabolic2005} and of metabolite concentration trajectories \citep{tuCyclicChangesMetabolic2007}.
In addition, single-cell studies \citep{papagiannakisAutonomousMetabolicOscillations2017, baumgartnerFlavinbasedMetabolicCycles2018} do not discuss phases at all as the single-cell microfluidic set-up does not allow live monitoring of transcription, and oxygen consumption rate is an emergent property from chemostat cultures.
% ---- I don't think any publication explicitly says this -- it's mostly just Kevin IIRC.  I suggest phrasing it differently, below
%However, others [INSERT CHAIN OF PUBLICATIONS HERE] debate the existence of the reductive-charging phase and argue that this phase is an artefact of cellular adaptation to glucose limitation or feast-and-famine conditions.
There is a possibly that the two- or three-phase response results from cellular adaptation to glucose limitation in chemostat cultures, and for the most part, it is unknown whether these dynamics hold true in glucose-rich conditions \citep{slavovCouplingGrowthRate2011}, which cannot be created in a chemostat.
That said, I will discuss the reductive phase in terms of the reductive-building and reductive-charging phases, then discuss the evidence in favour of merging these phases.

[ADD DIAGRAM HERE -- MELLOR HAS A GOOD ONE, ADAPT HERS]

In the oxidative phase, cells consume oxygen at a high rate as respiration, fermentation, and
% Potentially may be the reason auxotrophs (hypothetically) don't have YMCs --
% but I showed that they do
energy-demanding processes
like biosynthesis and gene expression occur.
In chemostats, this manifests as a rapid decrease in measured dissolved oxygen.
Occurrence of biosynthesis and gene expression is confirmed by increased transcripts of [INSERT NAME OF TRANSCRIPTS HERE] \parencite{tuLogicYeastMetabolic2005}.
As the oxidative phase transitions to the reductive phase, ethanol and acetate concentrations in the medium peak as respiration finishes \citep{tuLogicYeastMetabolic2005}.
`Redox state' metabolites, including NADH, NADPH, glutathione \citep{lloydUltradianMetronomeTimekeeper2005}, and flavins (FMN and FAD) % check if true for flavins
\parencite{murrayRedoxRegulationRespiring2011} become most oxidised in this phase.
Here, 70\% of metabolite concentrations peak with the combined autofluorescence of NADH and NADPH \citep{murrayRegulationYeastOscillatory2007}.

In the reductive phase, cells consume oxygen at a low rate, showing up as an increase in dissolved oxygen in chemostats.
During the so-called reductive-building phase, and activities linked to mitochondrial growth occur.
% as shown here.
There is evidence to suggest that activities linked to cell proliferation -- such as initiation of the cell division cycle, DNA replication, and spindle pole activity -- are gated to the reductive-building phase for both the short-period and long-period YMC.
Such evidence includes budding activity and the pattern of the expression of \emph{YOX1}, which encodes a cell division cycle repressor \citep{tuLogicYeastMetabolic2005}.
The hypothesis is that these links create a temporal separation between oxidative biochemical processes (in OX) and the cell division cycle so as to prevent reactive oxygen species generated by oxidative process from damaging DNA.
However, measuring DNA content and oxygen consumption in cells grown at different growth rates \citep{slavovCouplingGrowthRate2011} showed that the S phase of the cell cycle may occur in the oxidative phase if the cells have a slow growth rate.
% Not sure if they changed the growth rate of the cells though -- probably quite difficult
% to do in microfluidics setting.
This may be explained by the YMC gating the early and late cell cycle independently, as evidenced by single-cell evindence in [DESCRIBE THE EXACT LINE OF EVIDENCE HEINEMANN USED HERE] \parencite{papagiannakisAutonomousMetabolicOscillations2017}.
% What would be the function of such flexible gating?  Seems like community
% has no meaningful consensus yet.  Probably allocation of metabolites.
This evidence indicates that the gating between the YMC and the cell cycle is flexible,
though there is no meaningful consensus as to the function of such flexible gating.
% Discuss allocation of metabolites later -- and this can be linked to the modelling chapter of thesis.

% There are many strings to pull on this phase.
Finally, during the so-called reductive-charging phase,
non-respiratory metabolism and degradation processes occur to prepare the cell for the oxidative phase.
This non-respiratory metabolism includes glycolysis, ethanol and fatty acid metabolism, and nitrogen metabolism.
With these metabolic modes, under the regulation of the transcription factors Msn2p and Msn4p \citep{kuangMsn2RegulateExpression2017}, acetyl CoA accumulates for ATP production in the oxidative phase \citep{tuLogicYeastMetabolic2005}.
After acetyl CoA levels reach a threshold, it promotes histone acetylation and thus induces the oxidative phase.
These metabolic pathways also optimise production of NADPH -- based on the induction of \emph{GND2} -- to buffer against oxidative stress in the oxidative phase.
Genes associated with protein degradation, ubiquitinylation, peroxisomes, vacuoles, and the proteosome also peak in the reductive-charging phase.

\subsubsection{Variations in the yeast metabolic cycle} % Need a better name
\label{subsubsec:intro-ymc-definition-variation}

Furthermore, the YMC is robust, with oscillations persisting for weeks to months \citep{lloydRedoxRhythmicityClocks2007}.
Additionally, the oscillation period is insensitive to temperatures from \SI{25}{\celsius} to \SI{35}{\celsius} and media pH values from
% also see O'Neill et al. (2020)
2.9 to 6.0 \citep{lloydUltradianMetronomeTimekeeper2005}.

% (Discuss the evidence in favour of merging the reductive-building and reductive-charging phases)
% This REALLY needs more detail: review literature related to the R/C phase ----- I don't think there's a lot though.
% Are these two sides of the same coin?
% I think I either have my own notes on this or notes from a meeting with Kevin Correia
In bulk culture,
the cycles are synchronised across cells in culture.
In such conditions, the YMC has been observed as a
% Murray camp
40-minute short-phase cycle [INSERT MURRAY CAMP CITATIONS HERE] or
% Tu camp
a 4- to 5-hour long-phase cycle synchronised to the cell cycle [INSERT TU CAMP CITATIONS HERE]
%\citep{mellorMolecularBasisMetabolic2016}.

% [START COMMENTED-OUT MAIN TEXT]
%The short-phase cycle occurs in the polyploid IFO0233 strain cultured under 2\% glucose w/v, while the long-phase cycle occurs in the diploid CEN.PK strain cultured under 1\% glucose w/v \citep{tu_chapter_2010}.
% [END COMMENTED-OUT MAIN TEXT]
% or is it cells independently responding to chemostat conditions?
% My experimental evidence (Apr 2021) strongly suggests this.  And also see Laxman et al. (2010)
%
% 'Shitty reactor' idea: no short/long cycles in high-quality chemostats.
% I have an org note on this, and some related citations.
However, there is some debate in the literature as to what is responsible for the two phase lengths observed.
One possible explanation is that such short- or long-phase cycles are an artefact of chemostat culture conditions and truly high-quality chemostats do not produce such cycles [CITATION/SUPPORTING EVIDENCE NEEDED, especially that i'm making an argument possibly counter to the literature].
For example, accumulation of hydrogen sulfide \citep{oneillCircadianRhythmsPersist2011} and limitations imposed by [WHAT IS IT?? I CAN'T REMEMBER -- READ MY ORG NOTE].
This begs the question of whether yeast cells autonomously and individually generate these cycles -- a question bulk culture in chemostats cannot answer.  % probably good to tie all unsolved questions at the end and say 'look, this is why i do single-cell'.

\subsection{Yeast metabolic cycles under perturbations}
\label{subsec:intro-ymc-perturbations}
% - Nutrient perturbations
  % - Changing concentration or compositions of carbon sources.
  % - Changing concentration or compositions of nitrogen sources.
  % - Key deletion strains shed light on mechanism

Nutrient and genetic perturbations can affect the length of the metabolic cycle and its relationship with other cellular events.
The long-phase cycle may vary from 1.4 to 14 hours depending on the strain and culture conditions including the chemostat dilution rate \citep{caustonMetabolicRhythmsFramework2018};
lower dilution rates give lower growth rates and therefore longer-period metabolic oscillations.
Additionally, several studies \citep{slavovCouplingGrowthRate2011,oneillEukaryoticCellBiology2020} show that the duration of the low oxygen consumption phase increases while the duration of the high oxygen consumption phase holds constant if the metabolic cycle duration increases due to these reasons.
Such an observation may hold clues to the regulation of the metabolic cycle.

\subsubsection{Nutrient perturbations}
\label{subsubsec:intro-ymc-perturbations-nutrient}

The main nutrient perturbations people have studied so far are perturbations in carbon sources and in nitrogen sources.

Perturbations in carbon sources are well-documented.

% change glucose concentration
Lower glucose concentrations prolong the metabolic cycle, as evidenced by both chemostat [CITATION NEEDED, I currently just have Mellor 2016] and single-cell studies \parencite{papagiannakisAutonomousMetabolicOscillations2017}.
% 20 g/L --> 0.5 g/L doesn't seem to significantly prolong it, but maybe
% a greater effect will be seen if we get to the region of glucose limitation
% Though I am aware that this is my experimental observations -- need to see if literature confirms.
For example, increasing the growth rate increases the duration of the oxidative phase relative to the reductive phases \citep{slavovCouplingGrowthRate2011}, thus prolonging the metabolic cycle.
This effect is pronounced in the region of glucose limitation;
increasing the glucose concentration beyond a certain point does not produce an effect.
This experimental observations could by explained by models such as \textcite{jonesCyberneticModelGrowth1999}, which proposes that [INSERT WHAT THE MODEL PREDICTS IN TERMS OF GLUCOSE CONCENTRATION HERE].

% ferm vs non-ferm
Furthermore, non-fermentable carbon sources like pyruvate also lead to long-duration metabolic cycles \parencite{papagiannakisAutonomousMetabolicOscillations2017}.

% bulk addition and depletion
In addition, bulk depletion or addition of a carbon source can reset the phase of the YMC.
Chemostat studies say that an initial starvation phase is needed to generate long-lasting synchronous metabolic cycles \parencite{tuLogicYeastMetabolic2005} [OTHER CITATIONS PROBABLY USEFUL].
On the other hand, adding a bulk carbon source such as acetate, ethanol, or acetaldehyde can reset the phase of the YMC \citep{kuangMsn2RegulateExpression2017, krishnaMinimalPushPull2018}.

Perturbations in nitrogen sources are less well-studied.
% Maybe also because long OX stage as it's the stage for biomass building
But \textcite{baumgartnerFlavinbasedMetabolicCycles2018}, based on single-cell observations, suggest that decreasing nitrogen source concentration prolongs the YMC, as evidenced by [INSERT EXACTLY WHAT THEY DID HERE]

The fact that nutrient perturbations affect the metabolic cycle suggests a model in which cells adjust the metabolic cycle depending on nutrient availability, dynamically. % I say this SO MANY times.
However, the caveat is that creating nutrient perturbations in a chemostat affects so many aspects of the cell environment, that it is difficult to tease things apart.

% Reorganise this shit ----------------------------------------------------------------------
The phase difference between the YMC and cell cycle varies in different conditions % how?
% Confirmation that the YMC responds to nutrient availability?  How does phase
% resetting help the Cell Division Cycle then?
\citep{ewaldYeastCyclinDependentKinase2016}. % ok, but have I seen this phase shift with the different media (i.e. SC vs SM)?  What did Ewald have to say about this phase shift??
% 'bulk carbon source' -- could be adaptation to changing nutrient concentrations in general?
Furthermore, the YMC can oscillate multiple times per cell cycle, or even disappear in some conditions \citep{baumgartnerFlavinbasedMetabolicCycles2018}. % which conditions?  can these be tested?
% --> also see Causton et al. (2015)
% -----------------------------------------------------------------------------------------------

\subsubsection{Genetic perturbations}
\label{subsubsec:intro-ymc-perturbations-genetic}
Although the complete molecular basis of the yeast metabolic cycle is not well-characterised, key gene deletions shed light on the molecular mechanism of the yeast metabolic cycle.
% Mellor (2016) provide an awesome table.

Many of these deletions have been made for genes that control the cell division cycle and metabolism.

In particular, there are several deletions shown to remove the metabolic oscillations in chemostats: namely, \emph{ZWF1} \citep{tuCyclicChangesMetabolic2007}, \emph{GSY2}, \emph{GPH1} \parencite{oneillCircadianRhythmsPersist2011}.
\emph{ZWF1} is responsible for entry into the pentose phosphate pathway and subsequently a major source of NADPH generation, so deleting this gene may impair control of cellular redox;
however, because of its role, this gene has also been shown to wreck things like amino acid metabolism and auxotrophy [CONSULT MY ZWF1 NOTE, MAKE SURE WHAT I WRITE IS ACCURATE], so it might be a bit difficult to draw conclusions from this deletion in particular.
What's interesting is \emph{GSY2} and \emph{GPH1}, which both have roles in glucose/glycogen mobilisation and storage, thus suggesting that cycling of carbohydrate stores may be needed for the function of the metabolic cycle. % contradict with idea of lipid stores?  link with modelling?
In addition, \emph{MSN2} and \emph{MSN4} have been shown to regulate acetyl CoA accumulation in the reductive-charging phase, as evidenced by the lack of YMCs in deletion strains \citep{kuangMsn2RegulateExpression2017}.
This tells us that genes involved in signalling pathways play an important role in the integrity of the metabolic cycle too.

Additionally, other deletions have been shown to either change the frequency or the shape of the metabolic oscillations -- at least in terms of their dissolved oxygen traces.
\textcite{caustonMetabolicCyclesYeast2015} provide three examples: \emph{RIM11}, \emph{SWE1}, and \emph{TSA1 TSA2}. % they actually had more, but these are the three that pop in my head right now
\emph{RIM11} does this...
\emph{SWE1} does this...
\emph{TSA1 TSA2} does this...
Moreover, deletion of \emph{GTS1} shortens the oscillation periods \citep{lloydUltradianMetronomeTimekeeper2005},

Some discussion about peroxiredoxins and thioredoxins can fit here.
Cite \textcite{amponsahPeroxiredoxinsCoupleMetabolism2021}.

However, few genetic perturbation studies have been attempted in single-cell studies.
The most significant is in \textcite{baumgartnerFlavinbasedMetabolicCycles2018}, in which by deleting genes related to cell division cycle machinery [WHICH??], they showed that the metabolic cycle operates independently from the cell division cycle.

\subsection{Modelling the yeast metabolic cycle}
\label{subsec:intro-ymc-model}

Systems biology approaches have been used to develop models to explain features of biological rhythms.
An early model is \parencite{jonesCyberneticModelGrowth1999} [PLEASE ELABORATE].
There's also \textcite{krishnaMinimalPushPull2018} [PLEASE ELABORATE].
\citet{ozsezenInferenceHighLevelInteraction2019} use a deterministic Kuramoto model to explain the interaction between one metabolic oscillator and three cell cycle oscillators.
% [This is a bit out of place... probably nix it given that I've given up on using this.]
% Additionally, a data-driven stochastic modelling approach was used to describe the relationship between the circadian and cell cycle oscillators without assuming a prior relationship \citep{droin_low-dimensional_2019}.
% This strategy is yet to be applied to the metabolic and cell cycle oscillators.
% However, \citet{ozsezenInferenceHighLevelInteraction2019} assumed the same, specific coupling function for all interactions between the four oscillators.

\subsection{Big picture/Hypothesis: a nutrient sensor than entrains the cell division cycle?}
\label{subsec:intro-ymc-hypothesis}

From existing evidence, we can create a big picture of what the yeast metabolic cycle is.
Idea that the yeast metabolic cycle is a autonomous biological oscillator that operates at a range of frequencies in response to a range of (permissive) growth conditions, as evidenced by how extreme [BE MORE SPECIFIC, AND REFER TO A PREVIOUS SECTION] nutrient conditions throw the oscillator out of balance.

The yeast metabolic cycle creates windows of opportunities for the cell to commit to START if conditions are favourable: e.g. good stores.
Thus, this oscillator acts as a timing mechanism for cellular processes, most importantly the cell division cycle and biosynthetic/redox processes.
Though the relationship between the metabolic cycle and the cell division cycle is governed by the mathematical basis of coupled oscillators.
Most importantly, there is a small window of frequencies in which both oscillators can be phase-locked, and that other, complicated relationships exist: e.g. multiple metabolic cycles per cell division cycle -- in line in the window-of-opportunity idea above.

In addition, the yeast metabolic cycle responds dynamically and rapidly to nutrient availability.
Specifically, abrupt changes in availability may reset the phase of the cycle, and the concentration of nutrients in media changes the duration of the cycle -- again, within a window of frequencies.

% **Logical inconsistency**: previously I mentioned that this coupling is flexible &
% this idea is disputed.  This is an error!
% [COMMENTED-OUT MAIN TEXT]
% In turn, if the cell divides, it temporally partitions cell cycle events in relation to YMC events to prevent oxidative damage.

% *** I think chemostat vs single-cell discussions should be introduced WAY earlier, as I can't avoid discussing anything else about the YMC without going into the weeds of this.  But I can then dive into the dispute later.
\subsection{Disputes and unresolved questions with the yeast metabolic cycle}
\label{subsec:intro-ymc-unresolved}
% - Do single-cell flavin-based metabolic cycles from deletion strains recapitulate dissolved oxygen-based metabolic cycles in chemostats?  If not, what would be a likely explanation?

% (State the main unknowns of the yeast metabolic cycle:
% - molecular mechanisms, specifically...? -> deletions to investigate
% - whether things are the same in batch vs bulk)
There are several disputes and unknowns within the current literature pertaining yeast metabolic cycles.

\subsubsection{Chemostat vs single-cell studies}
\label{subsubsec:intro-ymc-unresolved-chemostat_singlecell}

% (Chemostat vs single-cell)
The most important dispute relates to how there are both studies arising from chemostat/bulk conditions and from single cell conditions,
although there are much fewer studies from single-cell set-ups.
Namely, whether metabolic cycles described in single-cell studies are equivalent to the metabolic cycles described in bulk studies.
This leads to definition issues, i.e. some authors \parencite{laxmanBehaviorMetabolicCycling2010, caustonMetabolicRhythmsFramework2018} only use the term metabolic cycle to refer to synchronised cycles of dissolved oxygen concentrations observed in chemostat cultures that must have gone through a starvation phase, while single-cell studies \parencite{baumgartnerFlavinbasedMetabolicCycles2018, zylstraMetabolicDynamicsCell2022} naturally have to expand that definition to include metabolite cycling and sequences of cellular events that are associated with the chemostat metabolic cycle.
As I work with budding yeast in a single-cell setup, I work with the latter definition.
Any conclusion from a single-cell study would be subject to the question: does it recapitulate dissolved oxygen-based metabolic cycles in chemostats?

Particularly when it uses a different readout such as metabolite redox state or ATP concentration... and it has to use a different readout because measuring the dissolved oxygen concentration does not make sense for single-cell studies.
Furthermore, using the chemostat imposes limitations: it obscures contributions from sub-populations of cells or individual cells, is not realistic compared to natural habit of yeast, and always imposes starvation.
The limitation of the imposition of starvation is highlighted by how single-cell studies have shown that metabolic oscillations occur when cells are cultured in high (20 g/L) glucose conditions.
Contributions from sub-populations of cells are highlighted by publications like \citet{burnettiCellCycleStart2016}, which [PUT WHAT THEY SAY HERE].
In addition, there is the question of whether cells individually generate the metabolic cycle or is a diffusible chemical responsible for synchrony (as proposed by e.g. \citet{krishnaMinimalPushPull2018}, which proposes that acetaldehyde does this; someone else proposed hydrogen sulfide).
Bulk culture set-ups, including chemostats, are not able to address questions about cell sub-populations and autonomy of the metabolic cycle; however, single-cell set-ups may fill in such a technical gap.
% NOTE: Sub-populations in single-cell: justify using Bagamery et al. (2020).  In addition to population in e.g. Burnetti et al. (2016).

Such limitations of the chemostat leads to disputes on the very phases the metabolic cycle exhibits, as discussed earlier.
The conditions of the chemostat may well force the population of cells to behave in a certain way.

It is unclear whether aspects of the YMC discovered in bulk culture apply to single cells.
There's only been a few single-cell studies of the YMC, and here I provide a brief review.
\citet{laxmanBehaviorMetabolicCycling2010} was an early attempt at using microfluidics to address the bulk vs single-cell issue [ELABORATE];
however, it lacks quantitative time-series analysis.
% This does seen like a key difference seen in single-cell, but the fact wasn't made clear
% in writing here.
\citet{papagiannakisAutonomousMetabolicOscillations2017} revealed that YMCs are an intrinsic feature of single cells and are autonomous with respect to the cell cycle, based on measurements of the combined level of NADH and NADPH in single cells in microfluidic devices.
% Also, Murray et al. (2011) -- flavins in BULK culture.
Furthermore, by measuring the level of flavins in the cell, \citet{baumgartnerFlavinbasedMetabolicCycles2018} demonstrated that YMCs persist in mutants deficient in oxidative phosphorylation, and that the cell cycle inhibitor rapamycin desynchronises the YMC and the cell cycle.
Additionally, \citet{ozsezenInferenceHighLevelInteraction2019} suggested that the YMC controls the early and late cell cycle independently, based on modelling the oscillators as a system of Kuramoto oscillators.

\subsubsection{Molecular and genetic mechanisms}
\label{subsubsec:intro-ymc-unresolved-molecular}
% (Unknown molecular mechanics)
Looking beyond the main disputes, there are several unknowns in the metabolic cycle.
There are unknowns in the molecular mechanism that drives YMCs.
Genome-wide transcript cycling has two superclusters that correspond to the oxidative and reductive-building phases \citep{machneYinYangYeast2012}.
However, there has been no genome-wide analysis of genes that influence cycling \citep{mellorMolecularBasisMetabolic2016}, though some genes seem to have key roles.
However, there are a number of questions remaining.
For example, without proteome analysis, it is unclear how protein levels and post-translation modifications are affected.
In terms of metabolite cycling...
[Write something about lipid stores \citep{campbellBuildingBlocksAre2020} vs carbohydrate stores \citep{ewaldYeastCyclinDependentKinase2016} here.  This is also a good place to add the figures I made from my investigation of the \citet{campbellBuildingBlocksAre2020} data and see if there is any periodicity.]
Additionally, the effects of perturbations to specific metabolic networks in single-cell conditions are yet to be determined [EVIDENCE NEEDED].

How YMCs are coordinated between cells is also unclear... and as I said above, whether such coordination is even needed is thrown into question by the presence of metabolic cycles in single-cell conditions.
% Need to re-read Murray et al. (2007) -- appreciate the evidence they give here
Experimental studies \citep{murrayRegulationYeastOscillatory2007} and mathematical models
% Actually, they aren't *that* explicit on H2S mediating cell-cell communication
% No synchronisation in single-cell conditions, so here's where simulating chemostat in microfluidics
% comes in (feast-and-famine).
% The different aspects found from bulk culture & single-cell experiments isn't written so
% clearly.  A proper review will be better.  These are just some aspects pasted together.
\citep{krishnaMinimalPushPull2018} suggest that acetaldehyde and hydrogen sulfide may mediate cell-to-cell communication.
The reason for this synchronisation between cells is still unclear. % citation needed
In addition, perhaps such cell-to-cell synchronisation may well be an artefact of chemostat activation [EVIDENCE/CITATION NEEDED -- SEE NOTES FROM KEVIN CORREIA].
It is entirely possible that observed cell-to-cell synchronisation is the effect of all cells independently responding to some condition, and chemostat studies cannot adequate say whether this is the case.

\subsection{Implications of the metabolic cycle}
\label{subsec:intro-ymc-implications}

%- campbell et al. 2020: suggests temporal regulation of biosynthesis for cell division cycle -- similar logic to ymc?  can put this in implications of ymc section


% Describe metabolic cycles in other organisms...
% - Other yeasts
% - Other organisms: bacteria, mammalian cells
% (I think Mellor probably talks about this)
Something something leads to the question of whether the metabolic cycle reflects something fundamental, perhaps an ancient biochemical adaptation to allow the cell to adapt its biochemistry and cell division cycle to external conditions.

%- Fundamental mechanistic basis for biological oscillators, such as the cell division cycle.  Could change what we know about the cell division cycle.
% (Significance of study)
Metabolic oscillations may be the origins of biological timekeeping mechanisms.
Indeed, \citet{lloydRedoxRhythmicityClocks2007} assert that ultradian oscillations form the basis of longer-period biological oscillators like the circadian rhythm or the cell cycle. % on what basis was this assertion made?
Circadian rhythms can occur in cells of multicellular eukaryotes without transcription \citep{oneillCircadianRhythmsPersist2011}, refuting the idea that gene circuits are responsible for these rhythms.
Additionally, the eukaryotic cell cycle evolved before cyclin-dependent kinases \citep{papagiannakisAutonomousMetabolicOscillations2017}, so metabolic oscillations may have served to regulate the cell cycle before cyclin-dependent kinases evolved.
Furthermore, YMCs share mechanisms with the circadian oscillator \citep{caustonMetabolicCyclesYeast2015,arataQuantitativeStudiesCellDivision2019}, suggesting a common evolutionary origin.
Thus, studying YMCs may shed light on the evolution of biological rhythms.

\section{Flavins and flavoproteins}
\label{sec:intro-flavin}
I'm writing this section because I'm using flavins to monitor the yeast metabolic cycle, as been done previously, and I'm going to discuss the background knowledege needed to make sense of the results.

\subsection{Introduction to cellular autofluorescence}
\label{subsec:intro-flavin-autofluo}
% - (Refer to reviews, e.g. Maslanka et al. 2018 -- is a goldmine)

Cellular autofluorescence is defined as the intrinsic fluorescence of a cell absent of any fluorescent tags, and this is caused by the autofluorescence of compounds in the cell that have light emission properties \parencite{maslankaAutofluorescenceYeastSaccharomyces2018}.
Such endogenous fluorophores include coenzymes, vitamins, and amino acids with aromatic chemical groups, flavins being one of them.
However, autofluorescence can pose difficulty in cellular microscopy because their wavelengths can overlap and therefore it is difficult to draw biochemical conclusions from the signal alone -- in some cases, certain treatments or conditions are needed to eliminate confounding factors [INSERT EXAMPLES AND EVIDENCE HERE].
Owing to the identity of those compounds, cellular autofluorescence can be a key indicator of the physiology and metabolism of the cell, and it has certainly been leveraged to give insight into the yeast metabolic cycle.

\subsection{Biochemical basis of flavins and flavoproteins}
\label{subsec:intro-flavin-biochem}
Flavins refer to a group of organic compounds that share a common moiety.
Its aromatic structure allows redox reactions.

[ADD CHEMICAL DIAGRAM: FLAVIN GROUP, FMN, FAD]

Specifically, the flavin moiety can exist in the oxidised, semiquinone, or reduced states.
Transitions between the states involve electron transfer.

Because flavins can exist in these states, they function as electron carriers in the cell.
In \emph{Saccharomyces cerevisiae}, flavin is present in FMN and FAD, which function as prosthetic groups in a group of proteins called flavoproteins that accounts for approximately 1\% of the proteome [FACT CHECK THE PERCENT NUMBER].
FMN and FAD can either be covalently bound to these proteins or be free.
  % [citation needed -- IIRC they don't exist as the free from for long?  Is the free form transient?].
FAD is a co-enzyme and has major roles in transferring electrons from the TCA cycle to the mitochondrial ETC.
FMN exists in both free and protein-bound forms.

% (Biosynthesis of flavin)
[ADD FIGURE: Schematic of riboflavin uptake and biosynthesis and their relationship with flavin-containing compounds in the yeast cell.]
% In more detail: https://www.genome.jp/pathway/map00740 (biosynthesis of riboflavin and derivatives, KEGG)

Broadly speaking, flavins in \emph{Saccharomyces cerevisiae} are derived from riboflavin.
Riboflavin can be synthesised in a \emph{de novo} pathway which derives from purine biosynthesis and the oxidative pentose phosphate pathway.
% Is there more to this than the KEGG diagram?

Architecture thus easily to understand and study -- very few processes are involved in making it, and nothing uses up flavin.
So it's clear that cell only makes new flavin for whatever purposes FMN and FAD are for.

[ADD FIGURE: SCHEMATIC OF DETECTING FLAVINS VIA MICROSCOPY, THAT I USE IN COUNTLESS POSTERS \& PRESENTATIONS]

From a technical standpoint, these redox states affect the emission/absorption of electromagnetic radiation of the flavin moiety.
Specifically, flavins

[ADD FIGURE(S): FLUORESCENCE EXCITATION \& EMISSION SPECTRUM OF RIBOFLAVIN]

From a technical standpoint, we can take advantage of cellular autofluorescence caused by the presence of flavins to monitor the redox state of the cell, and by extension, the yeast metabolic cycle.
FMN and FAD, in their oxidised forms, have a peak excitation frequency of 450 nm and a peak emission frequency of 535 nm;
in contrast, the reduced forms FMNH2 and FADH2 have negligible fluorescence [CHECK IF TRUE, REFER TO CITATIONS, PROBABLY BETTER TO USE NON-YMC CITATIONS AND PUBLICATIONS THAT DEAL WITH FMN/FAD DIRECTLY] \parencite{baumgartnerFlavinbasedMetabolicCycles2018, sasidharanTimeStructureYeastMetabolism2012}.
In addition, comparison of \emph{in vivo} autofluorescence in mammalian cells and the fluorescence spectrum of riboflavin in PBS confirm the fluorescence spectrum of flavins in the cell \parencite{aubinAutofluorescenceViableCultured1979}.
Therefore, we can detect flavin redox states using fluorescence microscopy, using excitation and emission wavelength windows that correspond to this spectrum.
Certainly, this has been performed both in chemostat-based \parencite{sasidharanTimeStructureYeastMetabolism2012, murrayRedoxRegulationRespiring2011} and single-cell microfluidic studies \parencite{baumgartnerFlavinbasedMetabolicCycles2018} of the yeast metabolic cycle.

\subsubsection{Description of key flavoproteins and their roles}
\label{subsubsec:intro-flavin-biochem-descriptions}
% - (Sort by abundance)
% - (On second thought, I'm not sure how useful a shopping list of flavoproteins is.
% probably better to pick a couple that are really key to what i study)

[ADD FIGURE: BAR CHART SHOWING ABUNDANCE OF MOST ABUNDANT FLAVOPROTEINS]

\citet{gudipatiFlavoproteomeYeastSaccharomyces2014} describe a family of flavoproteins in budding yeast and their roles. % Add some more notes from this publication -- a couple sentences
Many of the flavoproteins are poorly characterised with regard to their biochemical and enzymatic properties \citep{kochStructureBiochemicalKinetic2017}.

Focusing on the 12 most abundant \citep{hoUnificationProteinAbundance2018} flavoproteins...

There is a roughly equal division between FAD- and FMN-binding proteins.

The top dozen proteins or so on this list all catalyse redox reactions except for ilv2 -- though it has been hypothesised that an ancestral form catalysed a redox reaction; that said, it's a bit of a circular argument because this is inferred from the presence of FAD.
Indeed, the role in redox reactions is expected given the review and the redox capabilities of the flavin group afforded by its structure.
If they're all roughly in the same direction regarding redox, then the `cellular redox state' explanation makes sense: I would expect the equilibrium of these reactions to go roughly the same way at any given point of the YMC.

Specifically, these reactions include reduction of reactive chemical species/reactions involved in the response to oxidative stress (manifesting as increased abundances in response to DNA replication stress).
This is unsurprising given that these reactions need redox -- though not all enzymes involved in the response to reactive chemical species have flavin co-factors.
Additionally, the reactions include biosynthetic reactions.

Many, but not all, of these reactions require NADPH or NADH to donate electrons -- suggesting a link between flavins and NAD(P)H in regulating the so-called `cellular redox state'.
There is oye2 itself that catalyses the NADPH redox reaction; thus, this reaction should be \emph{the} main link between flavins and NAD(P)H.
It's no surprise then that flavin and NAD(P)H cycles are linked -- indeed, \citet{murrayRedoxRegulationRespiring2011} report that NAD(P)H and flavin both peak at the end of the HOC phase of the YMC.
This is reassuring as it provides some basis for looking at flavin, which plays well with PFS system.

And it makes logical sense for zwf1$\Delta{}$ cells to individually lack YMCs. Plus strong pressure to compensate using other pathways or selection pressure to evolve (and create suppressor mutants).

So far that none of these are `CDC proteins' in the strictest sense, but due to the interconnectedness of biochemistry in the cell, I'd expect e.g. fatty acid synthesis proteins to cycle along with the CDC as cell makes more membrane.
This is good because I need to rule out the possibility that what I'm seeing is just a function of the cell division cycle to make sure that I'm actually observing the metabolic cycle.

\subsection{Studies on global flavin changes in response to perturbations}
\label{subsec:intro-flavin-perturbations}
% - How oxygen and nutrient changes affect flavins and other autofluorescence (there are papers that specifically study this)

% (Corresponds to 'Responses of flavin to changing conditions' section in org note -- go through notes I made from each publication and synthesise.)
% \citet{sianoNADHFlavinFluorescence1989}
% \citet{podrazkyMonitoringGrowthStress2003}
% \citet{maslankaAutofluorescenceYeastSaccharomyces2018}
% \cite{horvathSituFluorescenceCell1993}

\subsection{Flavins and flavoproteins in the yeast metabolic cycle}
\label{subsec:intro-flavin-ymc}

Autofluorescence is usually something people would avoid in experiments, but we are taking advantage of it to monitor the yeast metabolic cycle.
This is because it is an easy-to-measure metabolic readout: it does not require additional engineering of strains e.g. fluorescence tags which may introduce burden, and all you need to do is adjust the excitation/emission wavelengths on the fluorescence microscope to the appropriate value.
Plus, depending on the fluorescent compound targeted, it can be linked to many biochemical processes; for example, flavins, as illustrated by the above review of flavin and flavoproteins in the cell.

I monitor flavin fluorescence and I have several justifications.
First, there is precedent.
Flavin fluorescence readings have been used to monitor metabolic oscillators both in chemostat \citep{murrayRedoxRegulationRespiring2011} and in single-cell \citep{baumgartnerFlavinbasedMetabolicCycles2018} experimental set-ups.

To extend on that precedent, there is further justification on the biological basis of flavins and flavoproteins.
They're linked to NAD(P)H via nitric oxide oxidoreductase (Yhb1), as discussed in section \ref{subsubsec:intro-flavin-biochem-descriptions}, and NAD(P)H cycles have been implicated in bulk-culture \citep{tuLogicYeastMetabolic2005} % and more
and single-cell \citep{papagiannakisAutonomousMetabolicOscillations2017} studies of the YMC, as discussed in section \ref{subsubsec:intro-ymc-definition-phases}.
The timing of flavin cycle peaks -- indicating oxidation -- coincides with the oxidative state of the YMC, as evidence by how flavin fluorescence peaks in-phase with oxygen uptake rates in the chemostat \citep{murrayRedoxRegulationRespiring2011,sasidharanTimeStructureYeastMetabolism2012}.
Riboflavin has been shown to peak in the oxidative state of the YMC, while FAD peaks in the reductive-building phase \parencite{tuCyclicChangesMetabolic2007}.
They took metabolic extracts at evenly-spaced time intervals and used a directed LC-MS/MS-based technique for metabolic profiling of the extracts.
Riboflavin and FAD are two (of many) metabolites they found oscillating.
Most abundant flavoproteins may have roles linked with the YMC.
The most abundant is Fas1 (fatty acid synthetase); because there is evidence that cycles of fatty acid stores are implicated in metabolic cycling in yeast \citep{campbellBuildingBlocksAre2020}, it is likely that fatty acid synthetase is heavily implicated.
Following this, the second most abundant is Yhb1, which may play a major role as discussed earlier.

So, for these reasons, I expect flavin autofluorescence to be oscillatory and be a useful read-out of the yeast metabolic cycle.

Reiterating the unknowns of the yeast metabolic cycle (section \ref{subsec:intro-ymc-unresolved}):
few studies have characterised how such oscillations respond to changing nutrient conditions or to gene deletions.
Furthermore,
few studies have used flavin oscillations to study how well features of the YMC found in bulk-culture experiments translate to the single cell.
Thus, filling in this knowledge gap is an avenue for further research.

Nevertheless, there are caveats of using flavin.
There could be other things fluorescing in that wavelength region, although [INSERT CITATIONS/EVIDENCE TO JUSTIFY THAT USE OF THIS WAVELENGTH IS OKAY].
And, this applies to other sources of autofluorescence too, including NAD(P)H.
Additionally, judging by flavin measurements alone, we cannot be sure if changes in the reading are due to changes in the `flavin pool' (the amount of flavin-derived moieties in the cells across all their redox states) or due to global changes in intracellular flavin redox state, as a function of intracellular redox state.
KEGG pathway (see section ...) suggests that there is a mechanism for the cell to increase its flavin pool by uptake of riboflavin from the media or from biosynthesis; certainly, the uptake route via Mch5p is confirmed by \textcite{gudipatiFlavoproteomeYeastSaccharomyces2014}.
However, the experimenter can eliminate the former possibility by using minimal media that does not contain riboflavin, which is a common component in yeast media.

Most studies [CITATIONS NEEDED] assume a constant flavin pool, based on [DISCUSS THEIR EVIDENCE HERE].
And therefore the oscillations can be seen as periodic shifts in redox equilibrium.
There is also the question of whether there is free-floating flavin in the cell, as opposed to flavins in the form of FAD/FMN bound to proteins.
Either way, flavin fluorescence is the sum of the activity of many proteins, and this leads to another caveat: you cannot conclude that flavin fluorescence is mostly Fas1 or any other protein, and the only biochemical conclusion you can make is conclusions about overall cellular redox state.
To close, I have to note that some of these caveats are not unique to flavin fluorescence, but a shared limitation of using fluorescence.

\subsubsection{Do flavin cycles suggest cycling of lipid stores?}
\label{subsubsec:intro-flavin-ymc-lipid_cycling}
% e.g. \textcite{campbellBuildingBlocksAre2020}
% Should this instead be in the bit where I discuss the molecular basis of the YMC?

[POTENTIAL FIGURES: MY ANALYSIS OF CAMPBELL ET AL. 2020 DATAFRAMES]

\textcite{campbellBuildingBlocksAre2020} showed that lipid biosynthesis in budding yeast is periodic with the cell division cycle and peaks during S phase, as evidenced by an increase in the number of metabolites implicated in lipid metabolism in such phases, based on metabolomics analysis of prototrophic cells with synchronised cell division cycles.
Furthermore, \textcite{ewaldYeastCyclinDependentKinase2016} also showed that lipid metabolism increased during S/G2/M, likely due to the synthesis of new cell membranes during bud growth, as evidenced by pathway enrichment analysis.
Taken together, such evidence suggests that cycling of cellular lipid stores may be key to the metabolic and cell division cycle oscillators in budding yeast.
This is further strengthened by the fact that a lipid-synthesis enzyme is the most abundant flavoprotein.
Then mention that a lipid-synthesis enzyme is one of the top flavoproteins.
But the evidence may be tenuous because of the `aggregate effects' of all flavoproteins as I discussed earlier in this section.

% Discuss ruling out other carbon metabolism, see Zylstra & Heinemann 2022

\section{Aims of thesis} % maybe rename this
\label{sec:intro-end}

This should be a short section on the questions I aim to answer, in the style of the last bit of introduction in a published paper.
% Though I'm not entirely sure if it would fit the logical flow of the introduction thus far; so see other theses to make sure.

%%% Local Variables:
%%% mode: latex
%%% TeX-master: "../thesis.tex"
%%% End:
