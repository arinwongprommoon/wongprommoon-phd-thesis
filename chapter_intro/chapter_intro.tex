% ROUGH DRAFT, based on 10-month report for now
% TODO:
% - Fact-checking
% - Add content (some from my org notes, and I refer to them in comments)

\chapter{Introduction}

\section{Motivation of thesis}

The motivation of this thesis is to understand how an organism adapts its metabolism and cellular processes in response to external conditions.
I do so by through using the the yeast metabolic cycle as a framework for other biological oscillators.
My reasons are twofold: (a) biological oscillators are important for coordination of responses and are present across kingdoms, (b) there are many unknowns about the yeast metabolic cycle, in particular the mechanistic basis and what happens in individual cells.
% might as well mention flavins here if i have to mention this in my description of chapter 1 in the list below.

Therefore, the general aim of my project is to study YMC regulation in isolated cells, especially mechanisms that ensure synchrony with the cell cycle, in different nutrient conditions.

% - describe the logic of the rest of the chapters
% TODO: replace chapter numbers with \ref{}
This thesis is divided into six chapters:
\begin{enumerate}
  \item Chapter 1 discusses the background behind the yeast metabolic cycle and the logic of using flavin autofluorescence as a way to monitor the yeast metabolic cycle.
  \item Chapter 2 discusses the methods used in my project, i.e. single-cell microfluidics of yeast cells, followed by an automated image analysis pipeline.
  \item Chapter 3 discusses the analysis of oscillatory time series; given the size of the datasets and the challenges of analysing noisy low-resolution time series, this deserves discussion in its own right.
  This chapter will step through the process of analysis and provide a review \& justification of the computational methods at each stage.
  \item Chapter 4 presents the biological results of my investigation of single-cell flavin-based yeast metabolic rhythms, employing the analysis methods discussed in chapter 3.
  In brief, I show that the metabolic cycle and cell division cycle are autonomous and synchronise in permissive conditions, while perturbations affect the relationship between these two biological oscillators.
  \item Chapter 5 discusses approaches to model the yeast metabolic cycle mathematically (so that predictions can be made) -- specifically addressing the questions of whether temporal partitioning of biosynthesis explains the timing of the yeast metabolic cycle and whether models of chemostat-based studies can be adapted to our understanding of single-cell metabolic cycles.
  \item Finally, chapter 6 ties together previous understanding of the metabolic cycle, experimental observations, and mathematical models to propose a coarse-grained, phenomenological model of the yeast metabolic cycle.
  And suggests further avenues of study.
\end{enumerate}


\section{Yeast metabolic cycle}
\label{sec:intro-ymc}

\subsection{Introduction to biological rhythms}
\label{subsec:intro-ymc-biological_rhythms}

\subsubsection{Biological basis of biological rhythms}
\label{subsubsec:intro-ymc-biological_rhythms-biological_basis}
% Physiological importance of biological rhythms
% including the circadian rhythm, cell division cycle, yeast metabolic cycle
% Basis, e.g. biochemical oscillators

% Literature:
% Mellor (2016) -- provides a good overview.
%I'm sure there are other citations I have lying around.

Biological rhythms are ... [INSERT SOME DEFINITION HERE].

Biological rhythms compartmentalise cellular processes and ensure that the cell prepares for sequential events.
Genetic oscillators, biochemical oscillators, and metabolic oscillators, all linked to a cellular redox cycle, govern biological rhythms \citep{mellorMolecularBasisMetabolic2016}.
Such biological rhythms include the circadian rhythm and the cell division cycle.

\subsubsection{Theoretical basis of biological rhythms}
\label{subsubsec:intro-ymc-biological_rhythms-theoretical_basis}
% - Mathematics of systems of coupled oscillators.

% Single oscillations
% Useful here: cell division cycle modelling literature, e.g. Novak/Tyson
Biological rhythms are modelled as ....
For example, the cell division cycle is modelled based on the molecular mechanisms (negative feedback loops) -- the most up-to-date and comprehensive one [REALLY?? CHECK CDC LITERATURE.  BUT DON'T GO TOO DEEP BECAUSE MY PHD ISN'T STRICTLY ABOUT CDCS] being \textcite{novakMitoticKinaseOscillation2022}, which employ ordinary differential equations to model negative feedback loops that sustain oscillations of kinases [MAKE SURE THIS SENTENCE IS ACTUALLY FACTUAL].
But people can get away with doing that with the cell division cycle because the molecular basis is well-characterised.
Less well-characterised biological rhythms such as glycolytic oscillations lead to more coarse-grained models, such as one driven by a positive feedback loop \parencite{goldbeterMultisynchronizationOtherPatterns} [EXPLAIN THE MODEL A BIT MORE HERE]. % Also add more examples

% Forced oscillators, coupled oscillators
% Useful here: \textcite{tysonTimekeepingDecisionmakingLiving} and related reviews
As biological rhythms are often coupled with each other, there is interest in modelling forced and coupled oscillators.
If an oscillator is forced, it means that it has a natural oscillation frequency when left to its own devices, but it is in a situation in which an external force is applied at a regular interval so as to force the oscillator to a certain frequency.
An example is the circadian clock, which is forced by being entrained to the light-dark cycle \parencite{goldbeterMultisynchronizationOtherPatterns}.
Closer to home, yeast glycolytic oscillations can also be entrained via a periodic input of substrate.
The concept of forced oscillators is closely linked to coupled oscillators, in which two oscillators are coupled to each other by certain activation/deactivation events.
% This is a good place to put an Arnold tongue figure to illustrate the point I'm trying to make in the next 3 sentences.
What tends to happen is that the two coupled oscillators oscillate at a compromise frequency if the natural frequencies of each are close enough to each other.
Otherwise, what can happen are complex oscillations in which the oscillators lock to a rational ratio of frequencies -- i.e. one oscillator goes through $p$ periods while the other goes through $q$ periods; the exact ratio depends on the ratio of the natural frequencies.
And in certain cases, chaos can happen.
There is a mathematical basis [CITATION NEEDED], and experimental observations support this [CITATIONS NEEDED, AND GO INTO SPECIFIC BIOLOGICAL EXAMPLES].
Certainly, the yeast metabolic cycle has been modelled as a system of coupled oscillators before \citep{papagiannakisAutonomousMetabolicOscillations2017,ozsezenInferenceHighLevelInteraction2019}, based on how it is linked to the cell division cycle -- but I will discuss this later when I finish defining what the metabolic cycle is.
% I can see this becoming relevant again in the modelling chapter

\subsection{Definition and description of the yeast metabolic cycle}
\label{subsec:intro-ymc-definition}
% Worth re-reading: Mellor 2016, Lloyd 2019

% Definition issue:
% Oscillations individually vs synchronised.
% Laxman et al. (2010) only calls it YMC if synchronised, while Papagiannakis et al. (2017)
% uses the term with single-cell NAD(P)H/ATP cycles.  I use the latter.
% There is this dispute in the scientific literature, which is worth discussing, it's central
% to my project, after all.

Heads-up: people don't really agree on what the yeast metabolic cycle (YMC) is, so I'm going to go with the bits that everyone seems to agree on first, and then discuss what people disagree on, and why.

% [BEGIN COMMENTED DRAFT TEXT]
% You can't discuss the definition of the yeast metabolic cycle (YMC) without addressing a certain dispute in the literature.
% Different authors have different standards to deem whether something is considered a yeast metabolic cycle or not.
% For example, most chemostat people \citep{laxmanBehaviorMetabolicCycling2010,caustonMetabolicRhythmsFramework2018} [CHECK WHETHER THEY ACTUALLY SAY THIS] only call it an YMC if the cycles are synchronised across cells.
% [END COMMENTED DRAFT TEXT]

The yeast metabolic cycle (YMC) is one such biological rhythm.
% 'aerobic' -- Oct 2020, Kevin did experiments in SC+pyruvate, degassed.
%   (I think the fluorescence killed cells)

% GOOD MAIN IDEA-THEN-EVIDENCE STRUCTURE HERE

% - Context of discovery: i.e. in chemostats, after glucose starvation, primary read-out being dissolved oxygen, start from Benjamin Tu in 2005.
It was initially described in \emph{Saccharomyces cerevisiae} cells cultured in chemostats; therefore, at high cell density and aerobic,
% 'nutrient-limited' -- Not necessarily.  Cycles are seen in SC+20 g/L glucose.
%   Glucose limitation is somewhere between 10--100 mg/L (Julian's data).
nutrient-limited conditions.
It comprises of oscillations in oxygen consumption, metabolite concentrations, and cellular events -- all coordinated by genome-wide transcriptional oscillations.

Several aspects of the YMC have been observed over decades.
\citet{nosohSYNCHRONIZATIONBUDDINGCYCLE1962} discovered that synchronised \emph{S. cerevisiae} cultures show oscillatory oxygen consumption.
\citet{kasparvonmeyenburgEnergeticsBuddingCycle1969} showed that gas metabolism and energy generation increase upon budding, while and \citet{mochanRespiratoryOscillationsAdapting1973} described a high-amplitude respiratory oscillation following a substrate shift from glucose to ethanol.
\citet{satroutdinovOscillatoryMetabolismSaccharomyces1992} were the first to describe the metabolic components of the short-phase YMC for cells in continuous culture.
\citet{tuLogicYeastMetabolic2005} first incorporated transcript cycling in the description of the YMC and defined the YMC events.

% Are these two sides of the same coin?
% I think I either have my own notes on this or notes from a meeting with Kevin Correia
In bulk culture, the YMC has been observed as a
% Murray camp
40-minute short-phase cycle [INSERT MURRAY CAMP CITATIONS HERE] or
% Tu camp
a 4- to 5-hour long-phase cycle synchronised to the cell cycle [INSERT TU CAMP CITATIONS HERE]
%\citep{mellorMolecularBasisMetabolic2016}.

% [START COMMENTED-OUT MAIN TEXT]
%The short-phase cycle occurs in the polyploid IFO0233 strain cultured under 2\% glucose w/v, while the long-phase cycle occurs in the diploid CEN.PK strain cultured under 1\% glucose w/v \citep{tu_chapter_2010}.
% [END COMMENTED-OUT MAIN TEXT]
% or is it cells independently responding to chemostat conditions?
% My experimental evidence (Apr 2021) strongly suggests this.  And also see Laxman et al. (2010)
These cycles are synchronised across cells in culture.
% 'Shitty reactor' idea: no short/long cycles in high-quality chemostats.
% I have an org note on this, and some related citations.
However, there is some debate in the literature as to what is responsible for the two phase lengths observed.
One possible explanation is that such short- or long-phase cycles are an artefact of chemostat culture conditions and truly high-quality chemostats do not produce such cycles [CITATION/SUPPORTING EVIDENCE NEEDED, especially that i'm making an argument possibly counter to the literature].
For example, accumulation of hydrogen sulfide \citep{oneillCircadianRhythmsPersist2011} and limitations imposed by [WHAT IS IT?? I CAN'T REMEMBER -- READ MY ORG NOTE].
This begs the question of whether yeast cells autonomously and individually generate these cycles -- a question bulk culture in chemostats cannot answer.  % probably good to tie all unsolved questions at the end and say 'look, this is why i do single-cell'.

Furthermore, the YMC is robust, with oscillations persisting for weeks to months \citep{lloydRedoxRhythmicityClocks2007}.
Additionally, the oscillation period is insensitive to temperatures from \SI{25}{\celsius} to \SI{35}{\celsius} and media pH values from
% also see O'Neill et al. (2020)
2.9 to 6.0 \citep{lloydUltradianMetronomeTimekeeper2005}.

% (In more detail... Describe the main features with brief references to key papers and experiments, as evidence.)

% INTEGRATE THIS STRUCTURE IN EACH PARAGRAPH THAT DISCUSSES EACH PHASE
% - Biochemical and metabolic aspects
%   - Metabolites: NADPH, ATP, flavins, etc.
%   - Mobilisation of macromolecules, storage molecules (carbohydates, lipids).
% - Transcription aspects
% - Cell biology aspects, including cell division cycle
% ---------------------------------------------------------------------
The YMC can be divided into two major phases: an oxidative, high-oxygen consumption (OX/HOC) phase and a reductive, low-oxygen consumption (RED/LOC) phase.
% Though Lloyd/Murray camp also describe these 3 phases
These oxidative and reductive phases are characterised by the rate of oxygen consumption \citep{mellorMolecularBasisMetabolic2016}.
\citet{krishnaMinimalPushPull2018} interpret the oxidative phase as a growth state, while the reductive phase is a quiescent state.
% Using unsupervised clustering, \citet{tuLogicYeastMetabolic2005} described the following events in each phase: --> WOULD LIKE TO AVOID THIS BECAUSE IT'S A REFERENCE TO JUST ONE PUBLICATION -- I'D RATHER WORK FROM THE CONSENSUS

[ADD DIAGRAM HERE -- MELLOR HAS A GOOD ONE, ADAPT HERS]

In the oxidative phase, cells consume oxygen at a high rate as respiration, fermentation, and
% Potentially may be the reason auxotrophs (hypothetically) don't have YMCs --
% but I showed that they do
energy-demanding processes
like biosynthesis and gene expression occur.
In chemostats, this manifests as a rapid decrease in measured dissolved oxygen.
Occurrence of biosynthesis and gene expression is confirmed by increased transcripts of [INSERT NAME OF TRANSCRIPTS HERE] \parencite{tuLogicYeastMetabolic2005}.
As the oxidative phase transitions to the reductive phase, ethanol and acetate concentrations in the medium peak as respiration finishes \citep{tuLogicYeastMetabolic2005}.
`Redox state' metabolites, including NADH, NADPH, glutathione \citep{lloydUltradianMetronomeTimekeeper2005}, and flavins (FMN and FAD) % check if true for flavins
\parencite{murrayRedoxRegulationRespiring2011} become most oxidised in this phase.
Here, 70\% of metabolite concentrations peak with the combined autofluorescence of NADH and NADPH \citep{murrayRegulationYeastOscillatory2007}.

In the reductive phase, cells consume oxygen at a low rate, showing up as an increase in dissolved oxygen in chemostats.
Some authors [INSERT CHAIN OF PUBLICATIONS HERE] identify a reductive-building (RB) phase and a reductive-charging (RC) phase within this main phase based on cellular and transcriptomic events.
However, others [INSERT CHAIN OF PUBLICATIONS HERE] debate the existence of the reductive-charging phase and argue that this phase is an artefact of cellular adaptation to glucose limitation or feast-and-famine conditions.
That said, I will discuss the reductive phase in terms of the reductive-building and reductive-charging phases, then discuss the evidence in favour of merging these phases.

During the so-called reductive-building phase, and activities linked to mitochondrial growth occur.
% as shown here.
There is evidence to suggest that activities linked to cell proliferation -- such as initiation of the cell division cycle, DNA replication, and spindle pole activity -- are gated to the reductive-building phase for both the short-period and long-period YMC.
Such evidence includes budding activity and the pattern of the expression of \emph{YOX1}, which encodes a cell division cycle repressor \citep{tuLogicYeastMetabolic2005}.
The hypothesis is that these links create a temporal separation between oxidative biochemical processes (in OX) and the cell division cycle so as to prevent reactive oxygen species generated by oxidative process from damaging DNA.
However, measuring DNA content and oxygen consumption in cells grown at different growth rates \citep{slavovCouplingGrowthRate2011} showed that the S phase of the cell cycle may occur in the oxidative phase if the cells have a slow growth rate.
% Not sure if they changed the growth rate of the cells though -- probably quite difficult
% to do in microfluidics setting.
This may be explained by the YMC gating the early and late cell cycle independently, as evidenced by single-cell evindence in [DESCRIBE THE EXACT LINE OF EVIDENCE HEINEMANN USED HERE] \parencite{papagiannakisAutonomousMetabolicOscillations2017}.
% What would be the function of such flexible gating?  Seems like community
% has no meaningful consensus yet.  Probably allocation of metabolites.
This evidence indicates that the gating between the YMC and the cell cycle is flexible,
though there is no meaningful consensus as to the function of such flexible gating.
% Discuss allocation of metabolites later -- and this can be linked to the modelling chapter of thesis.

% There are many strings to pull on this phase.
Finally, during the so-called reductive-charging phase,
non-respiratory metabolism and degradation processes occur to prepare the cell for the oxidative phase.
This non-respiratory metabolism includes glycolysis, ethanol and fatty acid metabolism, and nitrogen metabolism.
With these metabolic modes, under the regulation of the transcription factors Msn2p and Msn4p \citep{kuangMsn2RegulateExpression2017}, acetyl CoA accumulates for ATP production in the oxidative phase \citep{tuLogicYeastMetabolic2005}.
After acetyl CoA levels reach a threshold, it promotes histone acetylation and thus induces the oxidative phase.
These metabolic pathways also optimise production of NADPH -- based on the induction of \emph{GND2} -- to buffer against oxidative stress in the oxidative phase.
Genes associated with protein degradation, ubiquitinylation, peroxisomes, vacuoles, and the proteosome also peak in the reductive-charging phase.

% (Discuss the evidence in favour of merging the reductive-building and reductive-charging phases)
% This REALLY needs more detail: review literature related to the R/C phase ----- I don't think there's a lot though.

\subsection{Metabolic cycles in other organisms}
\label{subsec:intro-ymc-other_organisms}
% - Other yeasts
% - Other organisms: bacteria, mammalian cells
% (I think Mellor probably talks about this)
something something leads to the question of whether the metabolic cycle reflects something fundamental, perhaps an ancient biochemical adaptation to allow the cell to adapt its biochemistry and cell division cycle to external conditions.

\subsection{Yeast metabolic cycles under perturbations}
\label{subsec:intro-ymc-perturbations}
% - Nutrient perturbations
  % - Changing concentration or compositions of carbon sources.
  % - Changing concentration or compositions of nitrogen sources.
  % - Key deletion strains shed light on mechanism

Nutrient and genetic perturbations can affect the length of the metabolic cycle and its relationship with other cellular events.
The long-phase cycle may vary from 1.4 to 14 hours depending on the strain and culture conditions including the chemostat dilution rate \citep{caustonMetabolicRhythmsFramework2018}. % how?

The main nutrient perturbations people have studied so far are perturbations in carbon sources and in nitrogen sources.

Perturbations in carbon sources are well-documented.

% change glucose concentration
Lower glucose concentrations prolong the metabolic cycle, as evidenced by both chemostat [CITATION NEEDED, I currently just have Mellor 2016] and single-cell studies \parencite{papagiannakisAutonomousMetabolicOscillations2017}.
% 20 g/L --> 0.5 g/L doesn't seem to significantly prolong it, but maybe
% a greater effect will be seen if we get to the region of glucose limitation
% Though I am aware that this is my experimental observations -- need to see if literature confirms.
This effect is pronounced in the region of glucose limitation;
increasing the glucose concentration beyond a certain point does not produce an effect.
This experimental observations could by explained by models such as \textcite{jonesCyberneticModelGrowth1999}, which proposes that [INSERT WHAT THE MODEL PREDICTS IN TERMS OF GLUCOSE CONCENTRATION HERE].

% ferm vs non-ferm
Furthermore, non-fermentable carbon sources like pyruvate also lead to long-duration metabolic cycles \parencite{papagiannakisAutonomousMetabolicOscillations2017}.

% bulk addition and depletion
In addition, bulk depletion or addition of a carbon source can reset the phase of the YMC.
Chemostat studies say that an initial starvation phase is needed to generate long-lasting synchronous metabolic cycles \parencite{tuLogicYeastMetabolic2005} [OTHER CITATIONS PROBABLY USEFUL].
On the other hand, adding a bulk carbon source such as acetate, ethanol, or acetaldehyde can reset the phase of the YMC \citep{kuangMsn2RegulateExpression2017, krishnaMinimalPushPull2018}.

Perturbations in nitrogen sources are less well-studied.
% Maybe also because long OX stage as it's the stage for biomass building
But \textcite{baumgartnerFlavinbasedMetabolicCycles2018}, based on single-cell observations, suggest that decreasing nitrogen source concentration prolongs the YMC, as evidenced by [INSERT EXACTLY WHAT THEY DID HERE]

The fact that nutrient perturbations affect the metabolic cycle suggests a model in which cells adjust the metabolic cycle depending on nutrient availability, dynamically. % I say this SO MANY times.

Also, people have varied chemostat dilution rates and this broadly supports the above.
For example, increasing the growth rate increases the duration of the oxidative phase relative to the reductive phases \citep{slavovCouplingGrowthRate2011}, thus prolonging the metabolic cycle.
However, the caveat is that doing such a thing affects so many aspects of the cell environment, that it is difficult to tease things apart.

% Reorganise this shit ----------------------------------------------------------------------
The phase difference between the YMC and cell cycle varies in different conditions % how?
% Confirmation that the YMC responds to nutrient availability?  How does phase
% resetting help the Cell Division Cycle then?
\citep{ewaldYeastCyclinDependentKinase2016}. % ok, but have I seen this phase shift with the different media (i.e. SC vs SM)?  What did Ewald have to say about this phase shift??
% 'bulk carbon source' -- could be adaptation to changing nutrient concentrations in general?
Furthermore, the YMC can oscillate multiple times per cell cycle, or even disappear in some conditions \citep{baumgartnerFlavinbasedMetabolicCycles2018}. % which conditions?  can these be tested?
% --> also see Causton et al. (2015)
% -----------------------------------------------------------------------------------------------

Although the complete molecular basis of the yeast metabolic cycle is not well-characterised, key gene deletions shed light on the molecular mechanism of the yeast metabolic cycle.
% Mellor (2016) provide an awesome table.

Many of these deletions have been made for genes that control the cell division cycle and metabolism.

In particular, there are several deletions shown to remove the metabolic oscillations in chemostats: namely, \emph{ZWF1} \citep{tuCyclicChangesMetabolic2007}, \emph{GSY2}, \emph{GPH1} \parencite{oneillCircadianRhythmsPersist2011}.
\emph{ZWF1} is responsible for entry into the pentose phosphate pathway and subsequently a major source of NADPH generation, so deleting this gene may impair control of cellular redox;
however, because of its role, this gene has also been shown to wreck things like amino acid metabolism and auxotrophy [CONSULT MY ZWF1 NOTE, MAKE SURE WHAT I WRITE IS ACCURATE], so it might be a bit difficult to draw conclusions from this deletion in particular.
What's interesting is \emph{GSY2} and \emph{GPH1}, which both have roles in glucose/glycogen mobilisation and storage, thus suggesting that cycling of carbohydrate stores may be needed for the function of the metabolic cycle % contradict with idea of lipid stores?  link with modelling?
In addition, \emph{MSN2} and \emph{MSN4} have been shown to regulate acetyl CoA accumulation in the reductive-charging phase, as evidenced by the lack of YMCs in deletion strains \citep{kuangMsn2RegulateExpression2017}.
This tells us that genes involved in signalling pathways play an important role in the integrity of the metabolic cycle too.

Additionally, other deletions have been shown to either change the frequency or the shape of the metabolic oscillations -- at least in terms of their dissolved oxygen traces.
\textcite{caustonMetabolicCyclesYeast2015} provide three examples: \emph{RIM11}, \emph{SWE1}, and \emph{TSA1 TSA2}. % they actually had more, but these are the three that pop in my head right now
\emph{RIM11} does this...
\emph{SWE1} does this...
\emph{TSA1 TSA2} does this...
Moreover, deletion of \emph{GTS1} shortens the oscillation periods \citep{lloydUltradianMetronomeTimekeeper2005},

Some discussion about peroxiredoxins and thioredoxins can fit here.
Cite \textcite{amponsahPeroxiredoxinsCoupleMetabolism2021}.

However, few genetic perturbation studies have been attempted in single-cell studies.
The most significant is in \textcite{baumgartnerFlavinbasedMetabolicCycles2018}, in which by deleting genes related to cell division cycle machinery [WHICH??], they showed that the metabolic cycle operates independently from the cell division cycle.

\subsection{Modelling the yeast metabolic cycle}
\label{subsec:intro-ymc-model}

Systems biology approaches have been used to develop models to explain features of biological rhythms.
An early model is \parencite{jonesCyberneticModelGrowth1999} [PLEASE ELABORATE].
There's also \textcite{krishnaMinimalPushPull2018} [PLEASE ELABORATE].
\citet{ozsezenInferenceHighLevelInteraction2019} use a deterministic Kuramoto model to explain the interaction between one metabolic oscillator and three cell cycle oscillators.
% [This is a bit out of place... probably nix it given that I've given up on using this.]
% Additionally, a data-driven stochastic modelling approach was used to describe the relationship between the circadian and cell cycle oscillators without assuming a prior relationship \citep{droin_low-dimensional_2019}.
% This strategy is yet to be applied to the metabolic and cell cycle oscillators.
% However, \citet{ozsezenInferenceHighLevelInteraction2019} assumed the same, specific coupling function for all interactions between the four oscillators.

\subsection{Big picture/Hypothesis: a nutrient sensor than entrains the cell division cycle?}
\label{subsec:intro-ymc-hypothesis}

From existing evidence, we can create a big picture of what the yeast metabolic cycle is.
Idea that the yeast metabolic cycle is a autonomous biological oscillator that operates at a range of frequencies in response to a range of (permissive) growth conditions, as evidenced by how extreme [BE MORE SPECIFIC, AND REFER TO A PREVIOUS SECTION] nutrient conditions throw the oscillator out of balance.

The yeast metabolic cycle creates windows of opportunities for the cell to commit to START if conditions are favourable: e.g. good stores.
Thus, this oscillator acts as a timing mechanism for cellular processes, most importantly the cell division cycle and biosynthetic/redox processes.
Though the relationship between the metabolic cycle and the cell division cycle is governed by the mathematical basis of coupled oscillators.
Most importantly, there is a small window of frequencies in which both oscillators can be phase-locked, and that other, complicated relationships exist: e.g. multiple metabolic cycles per cell division cycle -- in line in the window-of-opportunity idea above.

In addition, the yeast metabolic cycle responds dynamically and rapidly to nutrient availability.
Specifically, abrupt changes in availability may reset the phase of the cycle, and the concentration of nutrients in media changes the duration of the cycle -- again, within a window of frequencies.

% **Logical inconsistency**: previously I mentioned that this coupling is flexible &
% this idea is disputed.  This is an error!
% [COMMENTED-OUT MAIN TEXT]
% In turn, if the cell divides, it temporally partitions cell cycle events in relation to YMC events to prevent oxidative damage.

\subsection{Implications of the metabolic cycle}
\label{subsec:intro-ymc-implications}
%- Fundamental mechanistic basis for biological oscillators, such as the cell division cycle.  Could change what we know about the cell division cycle.
% (Significance of study)

Metabolic oscillations may be the origins of biological timekeeping mechanisms.
Indeed, \citet{lloydRedoxRhythmicityClocks2007} assert that ultradian oscillations form the basis of longer-period biological oscillators like the circadian rhythm or the cell cycle. % on what basis was this assertion made?
Circadian rhythms can occur in cells of multicellular eukaryotes without transcription \citep{oneillCircadianRhythmsPersist2011}, refuting the idea that gene circuits are responsible for these rhythms.
Additionally, the eukaryotic cell cycle evolved before cyclin-dependent kinases \citep{papagiannakisAutonomousMetabolicOscillations2017}, so metabolic oscillations may have served to regulate the cell cycle before cyclin-dependent kinases evolved.
Furthermore, YMCs share mechanisms with the circadian oscillator \citep{caustonMetabolicCyclesYeast2015,arataQuantitativeStudiesCellDivision2019}, suggesting a common evolutionary origin.
Thus, studying YMCs may shed light on the evolution of biological rhythms.

% *** I think chemostat vs single-cell discussions should be introduced WAY earlier, as I can't avoid discussing anything else about the YMC without going into the weeds of this.  But I can then dive into the dispute later.
\subsection{Disputes and unresolved questions}
\label{subsec:intro-ymc-unresolved}
% - Do single-cell flavin-based metabolic cycles from deletion strains recapitulate dissolved oxygen-based metabolic cycles in chemostats?  If not, what would be a likely explanation?

% (State the main unknowns of the yeast metabolic cycle:
% - molecular mechanisms, specifically...? -> deletions to investigate
% - whether things are the same in batch vs bulk)
There are several disputes and unknowns within the current literature pertaining yeast metabolic cycles.

% (Chemostat vs single-cell)
The most important dispute relates to how there are both studies arising from chemostat/bulk conditions and from single cell conditions.
Namely, whether metabolic cycles described in single-cell studies are equivalent to the metabolic cycles described in bulk studies.
This leads to definition issues, i.e. some authors only use the term metabolic cycle to refer to cycles of dissolved oxygen concentrations observed in chemostat cultures that must have gone through a starvation phase, while single-cell studies naturally have to expand that definition to include metabolite cycling and sequences of cellular events that are associated with the chemostat metabolic cycle.
Any conclusion from a single-cell study would be subject to the question: does it recapitulate dissolved oxygen-based metabolic cycles in chemostats?
Particularly when it uses a different readout such as metabolite redox state or ATP concentration... and it has to use a different readout because measuring the dissolved oxygen concentration does not make sense for single-cell studies.

Furthermore, using the chemostat imposes limitations: it obscures contributions from sub-populations of cells or individual cells, is not realistic compared to natural habit of yeast, and always imposes starvation.
Contributions from sub-populations of cells are highlighted by publications like \citet{burnettiCellCycleStart2016}, which [PUT WHAT THEY SAY HERE].
In addition, there is the question of whether cells individually generate the metabolic cycle or is a diffusible chemical responsible for synchrony (as proposed by e.g. \citet{krishnaMinimalPushPull2018}, which proposes that acetaldehyde does this; someone else proposed hydrogen sulfide).
Bulk culture set-ups, including chemostats, are not able to address questions about cell sub-populations and autonomy of the metabolic cycle; however, single-cell set-ups may fill in such a technical gap.

Such limitations of the chemostat leads to disputes on the very phases the metabolic cycle exhibits, as discussed earlier.
The conditions of the chemostat may well force the population of cells to behave in a certain way.

Current knowledge of the YMC derives mostly from studies done on batch or continuous yeast cultures, but it is unclear whether aspects of the YMC discovered in bulk culture apply to single cells.
There's only been a few single-cell studies of the YMC, and here I provide a brief review.
\citet{laxmanBehaviorMetabolicCycling2010} was an early attempt at using microfluidics to address the bulk vs single-cell issue [ELABORATE];
however, it lacks quantitative time-series analysis.
% This does seen like a key difference seen in single-cell, but the fact wasn't made clear
% in writing here.
\citet{papagiannakisAutonomousMetabolicOscillations2017} revealed that YMCs are an intrinsic feature of single cells and are autonomous with respect to the cell cycle, based on measurements of the combined level of NADH and NADPH in single cells in microfluidic devices.
% Also, Murray et al. (2011) -- flavins in BULK culture.
Furthermore, by measuring the level of flavins in the cell, \citet{baumgartnerFlavinbasedMetabolicCycles2018} demonstrated that YMCs persist in mutants deficient in oxidative phosphorylation, and that the cell cycle inhibitor rapamycin desynchronises the YMC and the cell cycle.
Additionally, \citet{ozsezenInferenceHighLevelInteraction2019} suggested that the YMC controls the early and late cell cycle independently, based on modelling the oscillators as a system of Kuramoto oscillators.

% (Unknown molecular mechanics)
Looking beyond the main disputes, there are several unknowns in the metabolic cycle.
There are unknowns in the molecular mechanism that drives YMCs.
Genome-wide transcript cycling has two superclusters that correspond to the oxidative and reductive-building phases \citep{machneYinYangYeast2012}.
However, there has been no genome-wide analysis of genes that influence cycling \citep{mellorMolecularBasisMetabolic2016}, though some genes seem to have key roles.
However, there are a number of questions remaining.
For example, without proteome analysis, it is unclear how protein levels and post-translation modifications are affected.
In terms of metabolite cycling...
[Write something about lipid stores (Campbell et al. 2020) vs carbohydrate stores (Ewald et al. 2016) here.]
Additionally, the effects of perturbations to specific metabolic networks in single-cell conditions are yet to be determined [EVIDENCE NEEDED].

How YMCs are coordinated between cells is also unclear... and as I said above, whether such coordination is even needed is thrown into question by the presence of metabolic cycles in single-cell conditions.
% Need to re-read Murray et al. (2007) -- appreciate the evidence they give here
Experimental studies \citep{murrayRegulationYeastOscillatory2007} and mathematical models
% Actually, they aren't *that* explicit on H2S mediating cell-cell communication
% No synchronisation in single-cell conditions, so here's where simulating chemostat in microfluidics
% comes in (feast-and-famine).
% The different aspects found from bulk culture & single-cell experiments isn't written so
% clearly.  A proper review will be better.  These are just some aspects pasted together.
\citep{krishnaMinimalPushPull2018} suggest that acetaldehyde and hydrogen sulfide may mediate cell-to-cell communication.
The reason for this synchronisation between cells is still unclear. % citation needed
In addition, perhaps such cell-to-cell synchronisation may well be an artefact of chemostat activation [EVIDENCE/CITATION NEEDED -- SEE NOTES FROM KEVIN CORREIA].
It is entirely possible that observed cell-to-cell synchronisation is the effect of all cells independently responding to some condition, and chemostat studies cannot adequate say whether this is the case.

\section{Flavins and flavoproteins}
\label{sec:intro-flavin}
% USE ORG NOTE ABOUT FLAVINS/FLAVOPROTEINS

\subsection{Introduction to cellular autofluorescence}
\label{subsec:intro-flavin-autofluo}
% - (Refer to reviews, there are good ones in the 2000s)
% \citet{sianoNADHFlavinFluorescence1989}
% \citet{podrazkyMonitoringGrowthStress2003}
% \citet{maslankaAutofluorescenceYeastSaccharomyces2018}
% \cite{horvathSituFluorescenceCell1993}

\subsection{Biochemical basis of flavins and flavoproteins}
\label{subsec:intro-flavin-biochem}
% - Flavins are molecules with a certain aromatic moiety than can undergo redox reactions.  Show chemical structures.  These include FAD and FMN.
% - Flavoproteins have FAD and FMN as co-factors.

\subsubsection{Description of key flavoproteins and their roles}
\label{subsubsec:intro-flavin-biochem-descriptions}
% - (Sort by abundance)
% - Most in biosynthesis and redox.

\subsection{Studies on global flavin changes in response to perturbations}
\label{subsec:intro-flavin-perturbations}
% - How oxygen and nutrient changes affect flavins and other autofluorescence (there are papers that specifically study this)

\subsection{Flavins and flavoproteins in the yeast metabolic cycle}
\label{subsec:intro-flavin-ymc}
% - Autofluorescence is usually something people would avoid in experiments, but we are taking advantage of it.
% - Easy-to-measure metabolic readout (does not require additional engineering of strains), linked to many biochemical processes

\subsubsection{Review of chemostat and single-cell studies that use flavin}
\label{subsubsec:intro-flavin-ymc-precedent}
% - To justify my use, e.g. \textcite{murrayRedoxRegulationRespiring2011}, \textcite{baumgartnerFlavinbasedMetabolicCycles2018}

\subsubsection{Do flavin cycles suggest cycling of lipid stores?}
\label{subsubsec:intro-flavin-ymc-lipid_cycling}
% e.g. \textcite{campbellBuildingBlocksAre2020}

%%% Local Variables:
%%% mode: latex
%%% TeX-master: "../thesis.tex"
%%% End:
