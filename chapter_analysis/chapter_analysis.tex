\chapter{Analysis of oscillatory time series}

% Introduction: overview of the 'pipeline'
\section{Overview}
% - Zielinski et al. (2014) is a good starting point
% - Titles of the section lends themselves to a punchy opener paragraph that
%   describes the process

\section{Cleaning: choosing data, filtering, missing time points}

\section{Classification: is my time series oscillatory?}

% Literature review subsection
\subsection{Rhythmicity detection for biological data}
% - Compare and contrast methods
% - Highlight challenges with large datasets of noisy biological data

\subsection{Machine learning approaches to classification}
% Could be useful: Results section of 10-month report

\section{Characterisation: I have one time series -- what properties does it have?}

% Literature review subsection
\subsection{Periods, phases, amplitudes}

\subsection{Combining methods to get a picture of periodicity}

\section{Clustering: I have many time series (of the same signal) from many cells -- what are their relationships to each other?}

% Literature review subsection
\subsection{(Literature)}

\subsection{Machine learning approaches to clustering}
% - Featurisation -- decisions to make
% - Clustering approaches and algorithms -- compare and contrast

\section{Correlation: I have two signals from the same cell -- what are their relationships to each other?}
% - Cross-correlation: start from the basic definitions, then extend to population-level cross-correlation as used by Kiviet et al. (2014)

\section{Summary}
% - Justification of each steps of my pipeline
